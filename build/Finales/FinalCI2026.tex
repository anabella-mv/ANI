\documentclass[10pt]{article}

% Bibliotecas necesarias

\usepackage[a4paper, top=0.5cm, left=2cm, right=2cm,bottom=2cm]{geometry}
\usepackage{amsfonts}
\usepackage{amsmath, scalerel}
\usepackage{mathtools}
\usepackage{amssymb}
\usepackage{multicol}
\usepackage{fullpage}
\usepackage{graphicx}
\usepackage{float}

\geometry{top=1.2cm}
% Datos del documento

\title{\small Facultad de Matemática, Astronomía, Física y Computación}
\author{Simulacro Final\\ GURI - La Bisagra, conducción del CEIMAF}
\date{28 de Febrero de 2026}
\begin{document}
\maketitle

%Datos del alumno 

\noindent\textbf{Apellido y Nombre:}\\
\textbf{Comisión:}\\
\textbf{DNI:}
\begin{itemize}
        \item Leé cuidadosamente todas las consignas antes de comenzar.
        \item No está permitido el uso de calculadoras y/o celulares.
        \item Toda respuesta debe estar justificada, asegurate de acompañarla con su procedimiento y cuentas que realices, es evaluado como se llega a ella.
\end{itemize}

%Inicio de enumeración de las consignas a realizar 

\begin{enumerate}
    
    %Punto 1: Propiedades de los numeros reales
    \item Calcular:
    \begin{equation*}
        \frac{2026^2-2024^2}{2026+2024}+\left(\frac{3}{2}\right)^{44} \cdot \left(\frac{3}{2}\right)^{-46}\cdot 3^2 - \sqrt[16]{\left(\frac{2^4\cdot 3}{15 \cdot 5^2 \cdot 5^{-3}}\right)^4}
    \end{equation*}   
    
    %Punto 2: Sistema de ecuaciones
    \item Dado el siguiente enunciado:\\\\
    \textit{Nahuel encargó nuevos stickers de cada carrera de FAMAF, como compró por plantillas y sin recortar, Franco quiere saber cuantos stickers son por cada plantilla para ver si se necesita más ayuda para recortarlos. Nahuel sabe que en total pidió 66 stickers por carrera que vienen en 3 plantillas grandes y 2 pequeñas y también que el local al que encargó tiene promos de 25 stickers por 1 plantilla grande y 1 plantilla pequeña.}
    \begin{enumerate}
        \item Escribir el sistema de ecuaciones que representa la situación.
        \item Utilizando alguno de los métodos de igualación, sustitución o reducción, resolver el sistema de ecuaciones para encontrar cuantos stickers tiene cada plantilla.
    \end{enumerate}

    %Punto 3: Circunferencias, lógica proposicional y división de polinomios
    \item Dadas las siguientes proposiciones:
    \begin{itemize} 
        \item \textit{c}: El resto de la división entre $P(x)=3x^4+6x^3+6x-3$ y $Q(x)=3x^2+3$ es $R(x)=0$
        \item \textit{f}: La ecuación $(x-3)^2+(y-6)^2-25=24$  describe una circunferencia con centro en el punto $A=(3,6)$ y radio $9$
        \item \textit{k}: $\lnot (\exists x \in \mathbb{R} / x^0\neq 1) \equiv \forall x \in \mathbb{R} , x^0=1$
    \end{itemize}
    \begin{enumerate}
        \item Dar el valor de la verdad de cada proposición.
        \item Conociendo que la proposición $(\lnot t \iff q)$ es falsa y que $(p \implies q)$ es falsa, determinar el valor de la verdad de:\\\\
        \begin{tabular}{cccc}
             (i) $p$ & (ii) $q$ & (iii) $t$ & (iv) $(t \vee p)$ \\
        \end{tabular}
    \end{enumerate}

    %Punto 4: Factorización y dominio con sus restricciones
    \item 
    \begin{enumerate}
        \item Determinar el dominio de la función:
        \begin{equation*}
            f(x)=\frac{1}{x^2-3x} - \frac{x^2-4}{-x^2+5x-6}
        \end{equation*}
        \item Factorizar, simplificar y resolver la siguiente ecuación:
        \begin{equation*}
            \frac{1}{x^2-3x} = \frac{x^2-4}{-x^2+5x-6}
        \end{equation*}
    \end{enumerate}

    %Punto 5: Operaciones entre conjuntos y sus representaciones
    \item Sea $\mathcal{U}=(-\infty, 8]$ el conjunto universal. Sean $A=\{x \in \mathcal{U}/ x \geq 3\}$, $B=\{x-1/x \in \mathcal{U} \wedge x \in \mathbb{N} \wedge x>10\}$ y $C=(-\infty,7)$
    \begin{enumerate}
        \item Expresar el conjunto $A$ como intérvalo, al conjunto $B$ por extensión y al conjunto $C$ por compresión.
        \item Calcular las siguientes operaciones entre conjuntos utilizando la recta real, expresar el conjunto solución como intervalo(o unión de intervalos), y realizar los diagramas de Venn que representan las operaciones:\\\\
        \begin{tabular}{cccc}
             (i) $D=B^{c}$ & (ii) $E=A \cap C$ & (iii) $F=A-C$ \\
        \end{tabular}
    \end{enumerate}
    
    %Punto 6: Intersección de rectas, distancia entre puntos y pendiente.
    \item Sean las rectas $L_1$ y $L_2$ definidas por las ecuaciones:
    \[
        L_1: y = ax + b, \qquad \qquad L_2: y = \frac{3}{4}x + c
    \]
    Sabiendo que cortan en el punto $P=(3,-2)$ y son perpendiculares entre sí.
    \begin{enumerate}
        \item Calcular los valores de $a$, $b$ y $c$. Encontrar las coordenadas donde las rectas $L_1$ y $L_2$ intersecan con los ejes coordenados. (Eje x y eje y)
        \item Determinar si los puntos $A=(\frac{1}{2},\frac{4}{3})$ pertenece a la recta $L_1$ y si $B=(\frac{2}{3},\frac{4}{3})$ pertenece a la recta $L_2$ y calcular la distancia entre los puntos $A$ y $B$.
    \end{enumerate}

    %Punto 7: Parábolas.
    \item La función cuadrática $f(x) = 3x^2+bx-6$ que tiene eje de simetría en $x=\frac{1}{2}$.
    \begin{enumerate}
        \item Calcular el valor del coeficiente $b$.
        \item Calcular las coordenadas $(x_v,y_v)$ del vértice de la parábola.
        \item Calcular las coordenadas de la intersección de la parábola con el eje de ordenadas y con el eje de abscisas (raíces).
        \item Utilizando los datos encontrados en los incisos anteriores, esbozar el gráfico de la parábola (no utilizar tabla de valores).
    \end{enumerate}

    %Punto 8: identidades trigonométricas y circunferencia unitaria
    \item El punto $P(t)$ se ubica sobre la circunferencia unitaria en el primer cuadrante y tiene coordenadas $(x,\frac{\sqrt{5}}{3})$
    \begin{enumerate}
        \item Calcular el valor de las seis funciones trigonométricas del ángulo $t$.
        \item Ubicar en la circunferencia unitaria el punto $P(s)$ con $s=16\frac{\pi}{6}$ y dar sus coordenadas $(x,y)$.
        \item Calcular $cos(s+\frac{\pi}{2})$
    \end{enumerate}

    %Punto 9: Trigonometría y aplicación.
    \item Marcio vio que era ideal agregar más cuerdas para sostener la bandera que colgaba $60\sqrt{3} cm.$ en linea recta desde sólo dos ventanas con cuerdas, entonces decide ir a comprar más cuerdas con el fin de usarlas desde la primer y cuarta ventana, se forma un ángulo de $60^{\circ}$ desde la bandera al borde de la primera ventana y se forma un ángulo de $60^{\circ}$ desde el borde de la cuarta ventana a la bandera. Calcular la cantidad de cuerda que necesitaría Marcio. \textbf{(Ver figura)}
    \begin{figure}[htb]
        \centering
        \includegraphics[scale=0.15]{Imagenes/Trigonometria.png}
        \label{fig:punto9}
    \end{figure}

    %Punto 10: IVU
    \item En cada uno de los siguientes incisos, seleccione en esta hoja la casilla para indicar cúal es la recpuesta correcta:
    \begin{enumerate}
        \item Actualmente la conducción del CEIMAF la tiene el GURI, ¿Cómo está compuesto el CEIMAF?
        \begin{itemize}
            \item[$\square$] Consejerxs, titulares y suplentes
            \item[$\square$] Presidentx, secretarix general y vocalías.
            \item[$\square$] Decanato y secretarias.
        \end{itemize}
        \item ¿Cuál es el único requisito académico para poder ser elegido como representante estudiantil en el Consejo Directivo?
        \begin{itemize}
            \item[$\square$] Tener promedio mayor a 8 en al menos el $30\%$ de la carrera.
            \item[$\square$] Tener aprobado por lo menos 1/3 del número de años de la carrera o un tercio 1/3 del número total de materias establecidas en el plan de estudio, indistintamente.
            \item[$\square$] No existe requisito previo para poder ser electo Consejero Estudiantil.
        \end{itemize}
    \end{enumerate}
\end{enumerate}
\end{document}