\documentclass[11pt,a4paper,sans]{moderncv} % Font sizes: 10, 11, or 12; paper sizes: a4paper, letterpaper, a5paper, legalpaper, executivepaper or landscape; font families: sans or roman

\moderncvstyle{casual} % CV theme - options include: 'casual' (default), 'classic', 'oldstyle' and 'banking'
\moderncvcolor{green} % CV color - options include: 'blue' (default), 'orange', 'green', 'red', 'purple', 'grey' and 'black'

\usepackage{lipsum} % Used for inserting dummy 'Lorem ipsum' text into the template
%\usepackage[spanish]{babel}             % Cambia el lenguaje a espa\~{n}ol
\usepackage[utf8]{inputenc}             % Incorpora el juego de caracteres latinos con acentos
%\usepackage[latin1]{inputenc}
\usepackage[scale=0.75]{geometry} % Reduce document margins
%\setlength{\hintscolumnwidth}{3cm} % Uncomment to change the width of the dates column
%\setlength{\makecvtitlenamewidth}{10cm} % For the 'classic' style, uncomment to adjust the wipubdth of the space allocated to your name
\usepackage{etaremune}
\usepackage[T1]{fontenc}   % usar codificación T1
\usepackage{lmodern}       % fuentes Latin Modern (escalables)
\usepackage{microtype}     % mejora tipografía; ahora la expansión funcionará


%----------------------------------------------------------------------------------------
%	NAME AND CONTACT INFORMATION SECTION
%----------------------------------------------------------------------------------------

\firstname{Mailén Anabella} % Your first name
\familyname{Valdez} % Your last name

\title{Curriculum Vitae}
\address{Facultad de Matemática, Astronomía, Física y Computación, UNC}{}
\mobile{(+54) 387 527-8229}
%\phone{(054) 351 4622961}
%\fax{(054) 351 4334054 }
\email{ruben.sandez@unc.edu.ar}
%\homepage{http://www.lanais.famaf.unc.edu.ar/}{} % The first argument is the url for the clickable link, the second argument is the url displayed in the template - this allows special characters to be displayed such as the tilde in this example
%\extrainfo{additional information}
%


%\photo[100pt][0.5pt]{FotoCarnet_3.jpg} % The first bracket is the picture height, the second is the thickness of the frame around the picture (0pt for no frame)


%----------------------------------------------------------------------------------------

\begin{document}

\makecvtitle % Print the CV title

\section{Información Personal}

\cvitem{Lugar de Nacimiento}{Salta - Salta Capital}

\cvitem{Fecha de nacimiento}{22 de Mayo de 2002}

\cvitem {Nacionalidad} {Argentino}

\cvitem{DNI}{44.175.339}

\cvitem{Estado Civil}{Soltero}

% \cvitem {Edad} {22}
\cvitem{ Domicilio}{Bolivia 279, Piso 5, Dpto. A, Barrio Nueva Cordoba, Córdoba Capital}

\cvitem {Teléfono Celular} {(+54) 387 527-8229}
\cvitem {Correo electrónico} {ruben.sandez@unc.edu.ar}









%----------------------------------------------------------------------------------------
%	EDUCATION SECTION
%----------------------------------------------------------------------------------------

\section{Formación Académica}

\cventry{2021-En curso}{Licenciatura en Astronomía}{Facultad de Matemática, Astronomía, Física y Computación}{Córdoba, Argentina.}{}{Promedio con aplazos: 7.48, Promedio sin aplazos: 7.48, Porcentaje de avance: 67.65\%.}

\cventry{2025}{Especialización en Plasmas Astrofísicos}{}{Especialidad I: \textit{Introducción a la Magnetohidrodinámica}, dictada por el Dr. Federico Stasyszyn.}{En Curso.}{}

\cventry{2025}{Radioastronomía de Altas Frecuencias}{Curso de profesor visitante}{Dictado por Dr. Guillermo Giménez de Castro del Centro de Radioastronomía y Astrofísica Mackenzie (CRAAM-UPM) de São Pablo, Brasil.}{Completado.}{}

\cventry{2023-En curso}{Licenciatura en Física}{Facultad de Matemática, Astronomía, Física y Computación}{Córdoba, Argentina.}{}{Promedio con aplazos: 7.00, Promedio sin aplazos: 7.00, Porcentaje de avance: 58.06\%.}

\cventry{2019}{Bachiller en Economía y Administración}{Instituto de Educación Integral de Salta}{Promedio General: 8.29}{Salta Capital - Salta - Argentina}{}


%\cventry{}{Trabajo Final}{}{\textit{Comportamiento crítico en modelos de actividad neuronal definidos en redes de mundo pequeño con correlaciones bidimensionales.}, dirigida por el Dr. Sergio Alejandro Cannas}{}{}

%\cventry{2023-En curso}{Especialización en Mecánica Estadística}{}{Curso de Posgrado: \textit{Introducción a la Teoría de Fenómenos Críticos}, dictada por el Dr. Pablo Serra}{(Condición: regular, todavía por aprobar)}{}

%\cventry{2021-2022}{Especialización en Aprendizaje Automático}{}{Especialidad I: \textit{Redes Neuronales}, dictada por el Dr. Francisco Tamarit y el Dr. Juan Perotti. Especialidad II: \textit{Introducción al Aprendizaje Automático}, dictada por la Dra. Georgina Flesia}{}{}

%\cventry{2022}{Diplomatura en Ciencia de Datos y sus Aplicaciones}{Facultad de Matemática, Astronomía, Física y Computación}{Córdoba, Argentina}{}{Coordinada por la Dra. Ana Georgina Flesia. Aprobada}{}

% \cventry{2015-2019}{Bachiller en Ciencias Economicas y Administracion}{Instituto de Educacion Integral de Salta}{Salta, Argentina}{}{}
\section{Formación en Docencia}

\cventry{2025}{Ayudante Alumno por Concurso}{Segundo Cuatrimestre}{Introducción a la Astrofísica}{RHCD-2025-254-UNC-DEC\#FAMAF.}{En Curso.}{}

\cventry{2022-2024}{Clases de Consulta Ad-Honorem para el CEIMAF}{Todas las materias de primer año y el Cursillo de Nivelacion de las carreras: LA, LF, PF, PM, LM}{Facultad de Matemática, Astronomía, Física y Computación}{Cordoba, Argentina.}{}

\cventry{2023}{Programa Tutores Pares}{Tutor para el Programa TUTORES PARES de la Provincia de Cordoba}{Análisis Matemático I, Introduccion a la Física y del Curso de Nivelación}{{Facultad de Matemática, Astronomía, Física y Computación.}}{UNC.}{}

\cventry{2022-2023}{Clases Particulares Ad-Honorem a Ingresantes de la FAMAF}{Cursillo de Nivelacion}{Facultad de Matemática, Astronomía, Física y Computación.}{Cordoba, Argentina.}{}


\section{Formación en Investigación, Desarrollo o Produccion Artística}
%\cventry{Septiembre 2023}{108° Reunion Anual de la AFA}{Presentación de póster}{Título: \textit{Comportamiento crítico de un modelo neuronal definido en una red de Watts-Strogatz generada a partir de una red cuadrada}}{Autores: María Florencia Molina, Sergio A. Cannas}{Bahía Blanca}
\cventry{2025}{67° Reunión Anual de la A.A.A.}{Presentación Mural - Poster}{Título: \textit{Caracterización espectroscópica de una muestra de galaxias emisoras de rayos X en el sondeo LEGA-C}}{Autores:J.M. Puddu, M. Tetzlaff, C.M. Segovia, V.R. Sandez et al.}{Mendoza.}
\cventry{2025}{67° Reunión Anual de la A.A.A.}{Presentación Mural - Poster}{Título: \textit{Fotometría a partir de Imágenes CCD directas obtenidas con el telescopio Jorge Sahade}}{Autores:T.I. Macaroff, A.A. Medina, J.M. Puddu, V.R. Sandez et al.}{Mendoza.}
\cventry{2025}{67° Reunión Anual de la A.A.A.}{Presentación Mural - Poster}{Título: \textit{Determinación de parámetros astrofísicos de cúmulos abiertos de pequeño diámetro angular con fotometría de GAIA}}{Autores:B.N. Arnijas et al.}{Mendoza.}
\cventry{2025}{Conversatorio: "El futuro de la energía: ¿justa y soberana? Transición energética en Córdoba y Argentina"}{Orador y Moderador en el conversatorio organizado en conjunto con Jovenes Por El Clima - Córdoba}{Con Aval Institucional de la FAMAF}{RHCD-2025-347-E-UNC-DEC\#FAMAF.}{}{}
\cventry{2025}{XIV Edition of FRIENDS OF FRIENDS MEETING}{Asistente al evento}{Ciudad de Córdoba, Córdoba, Argentina.}{}{}
\cventry{2024}{Mes/2 de la Ciencia}{Organizador del evento de divulgación avalado academica e institucionalmente por la FAMAF}{}{\textit{RHCD-2024-490-UNC-DEC\#FAMAF.}}{}
\cventry{2024}{XIII Edition of FRIENDS OF FRIENDS MEETING}{Asistente al evento}{Ciudad de Córdoba, Córdoba, Argentina.}{}{}


\section{Formación Extensión}
\cventry{2024}{Noche De los Museos 2024 - 
"Museos por la educación y la investigación"}{}{Colaborador en "Física en Acción" (FAMAF - UNC) en el Museo del Observatorio Astronómico de Córdoba}{A cargo de la Secretaría de Extension de FAMAF.}{}



\section{Participación Institucional}
\cventry{2025}{Miembro del Comité Evaluador N° 1 }{Evaluación de Desempeño Docente - Convocatoria 2025}{FAMAF}{RD-2025-370-E-UNC-DEC\#FAMAF.}{}{}

\cventry{2023-En Curso}{Consejero Estudiantil}{}{Participación como miembro del Consejo Directivo de FAMAF}{Representante del claustro estudiantil}{AJE-2025-31-UNC-JE\#FAMAF.}

\cventry{2024-2025}{Representante Estudiantil del Consejo de Grado}{}{Participación como miembro titular del CoGrado de la FAMAF}{Representante del claustro estudiantil}{\textit{RHCD-2024-457-UNC-DEC\#FAMAF.}}

\cventry{2023-En Curso}{Miembro de la Comision Asesora de Equivalencias}{Comision Asesora de Equivalencias de FAMAF}{Representante Estudiantil}{RD-2024-485-UNC-DEC\#FAMAF.}{}

\cventry{2023-En Curso}{Representante Estudiantil de la C.A.A.}{}{Participación como miembro de la Comision Asesora de Astronomía}{Representante del claustro estudiantil}{RHCD-2024-456-UNC-DEC\#FAMAF.}

\cventry{2023}{Veedor Estudiantil}{}{Tribunal de concurso de Ayudantes Alumno/a para la Licenciatura en Fisica}{Representante del claustro estudiantil}{RHCD-2023-347-UNC-DEC\#FAMAF.}
\cventry{2023-2025}{Stand Informativo de la Facultad}{Muestra de Carreras 2024}{}{Coordinado por la Secretaría de Asuntos Estudiantiles de FAMAF.}{}{}

\cventry{2023-2024}{Espacio interactivo de Astronomía}{}{Presentación de experimentos en el marco de la muestra anual de carreras}{Coordinado por la Secretaría de Asuntos Estudiantiles de FAMAF.}{}{}
\cventry{2024}{Jornada de Puertas Abiertas 2024}{Participante en calidad de estudiante avanzado de la carrera}{Coordinado por la Secretaría de Asuntos Estudiantiles de FAMAF.}{}{}
\cventry{2023}{Jornada de Puertas Abiertas 2023}{Presentacion de experimentos en el marco de la Jornada de Puertas Abiertas de FAMAF.}{Coordinado por la Secretaría de Asuntos Estudiantiles de FAMAF.}{}{}

\cventry{2022}{Jornada de Puertas Abiertas 2022}{Asistente en el marco de la Jornada de Puertas Abiertas de FAMAF}{Coordinado por la Secretaría de Asuntos Estudiantiles de FAMAF.}{}{}

% \section{Otros Antecedentes}
% 
% \section{Experiencia docente}
% \cventry{2023}{Tutor del programa TUTORES PARES de la Provincia de Cordoba}{Análisis Matemático I, Introduccion a la Física y del Curso de Nivelación}{{Facultad de Matemática, Astronomía, Física y Computación}}{UNC}{}



% \cventry{2021-2022}{Capacitador en Contenido a Estudiantes de Colegios Secundarios}{Modelo de Camara de Senadores, OAJNU - RP Córdoba}{Cordoba, Argentina}{}{}

% \cventry{2020}{Capacitador en Contenido a Estudiantes de Colegios Secundarios}{Modelo de Naciones Unidas, OAJNU - RP Salta}{Salta, Argentina}{}{}


%\cventry{2023 (Agosto-Noviembre)}{Ayudante Alumno A DS}{Materia: Física General I}{{Facultad de Matemática, Astronomía, Física y Computación}}{UNC}{}
%\cventry{2023 (Marzo-Julio)}{Ayudante Alumno A DS}{Materia: Introducción a la Física}{{Facultad de Matemática, Astronomía, Física y Computación}}{UNC}{}
%\cventry{2023 (Febrero)}{Ayudante Alumno A DS}
%{Materia: Curso de Nivelación}{{Facultad de Matemática, Astronomía, Física y Computación}}{UNC}{}{}

%----------------------------------------------------------------------------------------
%	WORK EXPERIENCE SECTION
%----------------------------------------------------------------------------------------
%\pagebreak
%\section{Cursos de Extensión}
%\cventry{2020}{Curso de programación “Python Científico”}{aprobado Res. CD 341/2019}{dictado por el Dr. Edgardo Bonzi y el Dr. Oscar Reula}{Facultad de Matemática, Astronomía y Física}{}{}

%\section{Presentaciones en Congresos y Reuniones científicas}
%\cventry{Septiembre 2023}{108° Reunion Anual de la AFA}{Presentación de póster}{Título: \textit{Comportamiento crítico de un modelo neuronal definido en una red de Watts-Strogatz generada a partir de una red cuadrada}}{Autores: María Florencia Molina, Sergio A. Cannas}{Bahía Blanca}


% \section{Actividades de Gestión }

%\cventry{2023}{Miembro del Comité Evaluador N°8}{}{Evaluación de los/as Profesores/as
%Regulares y Auxiliares con el fin de considerar la
%renovación de sus designaciones por concurso}{FAMAF}{}
%\cventry{29 de Marzo 2023}{Asambleísta en la Asamblea Universitaria}{Participación como oradora en representación del claustro estudiantil}{Asamblea convocada con el objetivo de modificar el estatuto universitario}{UNC}{}
%\cventry{2022}{Miembro del Comité Evaluador N°3}{}{Evaluación de los/as Profesores/as
%Regulares y Auxiliares con el fin de considerar la
%renovación de sus designaciones por concurso}{FAMAF}{}

%\pagebreak


%----------------------------------------------------------------------------------------
%	BECAS SECTION
%----------------------------------------------------------------------------------------








%--------------------------------------------------------------------------
%	ACTIVIDADES DE EXTENSIÓN
%--------------------------------------------------------------------------

% \section{Actividades de Extensión}

%\cventry{2019}{Feria Provincial de Ciencias}{}{Presentación de experimentos y exposición sobre la Lic. en Física}{coordinada por la Dra. Lucía Arenas}{}{}


%\cventry{2022}{Día Internacional de la Luz}{}{Coordinación y exposición de experimentos}{Coordinado por la Dra. Lucía Arenas y el Dr. Lorenzo Iparraguirre}{Plaza Cielo y Tierra}
% \cventry{2023-2024}{Stand Informativo de la Facultad}{}{Presentación de experimentos en el marco de la muestra anual de carreras}{Coordinado por la Secretaría de Asuntos Estudiantiles de FAMAF}{}{}

% \cventry{2023-2024}{Espacio interactivo de Astronomía}{}{Presentación de experimentos en el marco de la muestra anual de carreras}{Coordinado por la Secretaría de Asuntos Estudiantiles de FAMAF}{}{}
% \cventry{2024}{Jornada de Puertas Abiertas 2024}{ALGO}{Coordinado por la Secretaría de Asuntos Estudiantiles de FAMAF}{}{}
% \cventry{2023}{Jornada de Puertas Abiertas 2023}{Presentacion de experimentos en el marco de la Jornada de Puertas Abiertas de FAMAF}{Coordinado por la Secretaría de Asuntos Estudiantiles de FAMAF}{}{}

% \cventry{2022}{Jornada de Puertas Abiertas 2022}{Asistente en el marco de la Jornada de Puertas Abiertas de FAMAF}{Coordinado por la Secretaría de Asuntos Estudiantiles de FAMAF}{}{}

%----------------------------------------------------------------------------------------
%	VOLUNTARIADO SECTION
%----------------------------------------------------------------------------------------
\section{Otros Antecedentes}
\cventry{2021- En Curso}{Miembro Activo}{Centro de Estudiantes de la Facultad de Matemáticas, Astronomía, Física y Computación (CEIMAF)}{}{}{}
\cventry{2021-2024}{Miembro de la Comisión Directiva}{CEIMAF}{}{}{}
\cventry{2022-2023}{Secretario de Deportes}{Comisión Directiva del CEIMAF}{}{}{}
\cventry{2021-2024}{Voluntario en la Organización de Jovenes Para las Naciones Unidas}{Representación Provincial Córdoba.}{}{}{}
\cventry{2020-2021}{Voluntario en la Organización de Jovenes Para las Naciones Unidas}{Representación Provincial Salta.}{}{}{}
 



\cventry{2021-2022}{Capacitador en Contenido a Estudiantes de Colegios Secundarios}{Modelo de Camara de Senadores, OAJNU - RP Córdoba}{Cordoba, Argentina.}{}{}
 \cventry{2020}{Capacitador en Contenido a Estudiantes de Colegios Secundarios}{Modelo de Naciones Unidas, OAJNU - RP Salta}{Salta, Argentina.}{}{}


%----------------------------------------------------------------------------------------
%	EXPERIENCIA LABORAL SECTION
%----------------------------------------------------------------------------------------
%\pagebreak
\section{Experiencia Laboral}
\cventry{2019}{Tomografía Computada S.E.}{Pasantia laboral no remunerada}{Area Administrativa}{Salta Capital.}{}


%----------------------------------------------------------------------------------------
%	LANGUAGES SECTION
%----------------------------------------------------------------------------------------

\section{Idiomas}
\cvitem{\textbf{Español}}{Lengua Materna}{}{}{}{}
\cvitem{\textbf{Inglés}}{Intermedio}{}{}{}{}
\cvitem{\textbf{Portugués}}{Básico}{}{}{}{}
%\cventry{2013}{Inglés - Preliminary English Test}{{Cambridge English Language Assessment}}{certificado nivel 1 en \textit{ESOL inernational}}{}{}

%----------------------------------------------------------------------------------------
%	LANGUAGES DE PROGRAMACION SECTION
%----------------------------------------------------------------------------------------

\section{Lenguajes de Programacion}
\cvitem{\textbf{\LaTeX}}{Intermedio}{}{}{}{}

\cvitem{\textbf{Python}}{Intermedio}{}{}{}{}

\cvitem{\textbf{Fortran 90}}{Basico}{}{}{}{}
\section{Disponibilidad horaria}

\cvitem{\textbf{2025}}{Horarios flexibles por las tardes}

%\cventry{2014}{Inglés - EF Cambridge English Level Test (EFCELT)}{}{C1}{{EF International Language Centers, Cambridge}}{}

%\cventry{2015}{Inglés - First Certificate in English (FCE)}{}{Grade A}{{Cambridge English Language Assessment}}{}

%\cventry{2016}{Alemán - Examen de la sociedad Goetheana Argentina}{}{A2.2}{{Goethe Institut}}{}


\end{document}