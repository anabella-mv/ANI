\documentclass[12pt]{article}

\usepackage[a4paper, top=0.5cm, left=2cm, right=2cm,bottom=2cm]{geometry}
\usepackage{amsfonts}
\usepackage{amsmath}
\usepackage{mathtools}
\usepackage{amssymb}
\usepackage{multicol}
\usepackage{fullpage}
\usepackage{graphicx}

\geometry{top=1.2cm}
\title{\small Facultad de Matemática, Astronomía, Física y Computación}
\author{Simulacro Segundo Parcial\\ GURI - La Bisagra, conduccion del CEIMAF}
\date{}
\begin{document}
\maketitle
\begin{enumerate}
    \item Sea $m(x)$ la funcion dada por:
    \begin{equation*}
    m(x)=\frac{6x}{x^2+3x-10}
    \end{equation*}
    \begin{enumerate}
        \item Para ver donde está bien definido el dominio de la función dada se debe tener en cuenta las posibles restricciones, en este caso evitar el 0 en el denominador, por lo tanto buscamos las raíces del denominador:
        \begin{align*}
            x_{1,2}=&\frac{-3\pm \sqrt{3^2-4\cdot 1 \cdot (-10)}}{2\cdot 1}\\
            =&\frac{-3\pm \sqrt{49}}{2}=\frac{-3\pm 7}{2}\\
            =&\{-5,2\}
        \end{align*}
    \end{enumerate}
\end{enumerate}

\end{document}