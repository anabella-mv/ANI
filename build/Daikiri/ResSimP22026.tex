\documentclass[12pt]{article}

\usepackage[a4paper, top=0.5cm, left=2cm, right=2cm,bottom=2cm]{geometry}
\usepackage{amsfonts}
\usepackage{amsmath}
\usepackage{mathtools}
\usepackage{amssymb}
\usepackage{multicol}
\usepackage{fullpage}
\usepackage{graphicx}
\usepackage{float}

\geometry{top=1.2cm}
\title{\small Facultad de Matemática, Astronomía, Física y Computación}
\author{Simulacro Segundo Parcial\\ GURI - La Bisagra, conduccion del CEIMAF}
\date{}
\begin{document}
\maketitle
\begin{enumerate}
    \item Sea $m(x)$ la funcion dada por:
    \begin{equation*}
    m(x)=\frac{6x}{\sqrt{x^2+3x-10}}
    \end{equation*}
    \begin{enumerate}
        \item Para ver donde está bien definido el dominio de la función dada se debe tener en cuenta las posibles restricciones, en este caso evitar el 0 en el denominador y evitar que el radicando sea negativo pues estamos trabajando en los reales, por lo tanto, primero buscamos las raíces del denominador:
        \begin{align*}
            x_{1,2}=&\frac{-3\pm \sqrt{3^2-4\cdot 1 \cdot (-10)}}{2\cdot 1}\\
            =&\frac{-3\pm \sqrt{49}}{2}=\frac{-3\pm 7}{2}\\
            =&\{-5,2\}
        \end{align*}
        Ahora, viendo que el coeficiente principal es positivo, tenemos una idea del comportamiento de la cuadrática lo que resulta en que sus valores negativos sean alcanzados en medio de sus raíces, por lo tanto es un intervalo que debemos evitar para el dominio. Resulta $Dom(m(x))=\{x \in \mathbb{R} : x < -5 \vee x > 2\}$.
        \item Para determinar la imágen que pasa por $x=4$ se debe evaluar la función en esos puntos:
        \begin{align*}
            m(4)=&\frac{6\cdot 4}{\sqrt{4^2+3\cdot 4-10}}=\frac{24}{\sqrt{16+12-10}}=\frac{24}{\sqrt{18}}=\frac{24}{\sqrt{2\cdot 3^2}} =\frac{24}{3\sqrt{2}}=\frac{8}{\sqrt{2}}
        \end{align*}
            Por lo tanto $m(4)=\frac{8}{\sqrt{2}}$.
    \end{enumerate}
    \item Considerando la función del gráfico y teniendo en cuenta sus puntos cerrados y abiertos podemos resolver los incisos:
    \begin{enumerate}
        \item 
        \begin{itemize}
            \item el dominio es de la forma $Dom(f(x))=[-4,-1)\cup (0,7]$
            \item La imagen es de la forma $Im(f(x))=[-4,3]\cup [4,5)$
            \item Las raices son en $x=-4$, $x=-3$ y $x=-2$ en la parte donde la funcion es senoidal ya que en $-1$ es un punto que no pertenece al dominio.
        \end{itemize}
        \item Los valores que cumplen tal condición son los que respectan a la parte senoidal y cuadrática, entonces ${x \in \mathbb{R}: f(x)\leq 3}=[-4,-1)\cup(1,4]$
        \item Las transformaciones que se aplican para graficar $g(x)=-f(x+2)$ son:
        \begin{itemize}
            \item Traslación horizontal de $-2$ unidades, es decir, hacia la izquierda.
            \begin{figure}[H]
                \centering
                \includegraphics[scale=0.5]{Imagenes/f2izquierda.png}
                \label{fig:funcion desplazada}
            \end{figure}
            \item Reflexión respecto al eje $x$.
            \begin{figure}[H]
                \centering
                \includegraphics[scale=0.5]{Imagenes/f2reflejada.png}
                \label{fig:funcion reflejada}
            \end{figure}
        \end{itemize}
    \end{enumerate}
    \item 
    \begin{enumerate}
        \item Para determinar la ecuación de la recta que pasa por los puntos $A=(-5,\frac{3}{2})$ y $B=(1,\frac{15}{2})$ se debe calcular la pendiente de la recta y luego utilizar la forma punto-pendiente para escribir su ecuación:
        \begin{align*}
            m=&\frac{\frac{15}{2}-\frac{3}{2}}{1-(-5)}=\frac{\frac{12}{2}}{6}=1\\
            y-y_1=&m(x-x_1)\\
            y-\frac{3}{2}=&1(x-(-5))\\
            y=&x+\frac{13}{2}
        \end{align*}
        \item Retomamos lo que conocemos de las rectas perpendiculares, sabemos que la pendiente de la recta $L$ es $m_L=\frac{3}{2}$ (esto es de distribuir y escribir la ecuación de forma polinómica $L: y=\frac{3}{2}x+\frac{9}{2}$), entonces la pendiente de la recta $P$ es $m_P=-\frac{1}{m_L}=-\frac{2}{3}$, ahora utilizando la forma punto-pendiente con el punto $C=(-7,7)$ se puede escribir la ecuación de la recta $P$:
        \begin{align*}
            y-y_1=&m(x-x_1)\\
            y-7=&-\frac{2}{3}(x-(-7))\\
            y=&-\frac{2}{3}x-\left(-\frac{2}{3}\right) \cdot (-7)+7\\
            y=&-\frac{2}{3}x-\frac{14}{3}+7\\
            =&-\frac{2}{3}x+\frac{7}{3}
        \end{align*}
        Entonces $P: y=-\frac{2}{3}x+\frac{7}{3}$.
        \item Para calcular las coordenadas del punto de intersección entre las rectas $L$ y $P$ se debe igualar sus ecuaciones:
        \begin{align*}
            \frac{3}{2}x+\frac{9}{2}=&-\frac{2}{3}x+\frac{7}{3}\\
            \frac{3}{2}x+\frac{2}{3}x=&\frac{7}{3}-\frac{9}{2}\\
            \frac{9+4}{6}x=&\frac{14-27}{6}\\
            x=&\frac{-13}{6}\cdot \frac{6}{13}=-1
        \end{align*}
        Ahora para calcular la coordenada $y$ se puede utilizar cualquiera de las dos rectas, por ejemplo utilizando $L$:
        \begin{align*}
            y=&\frac{3}{2}(-1)+\frac{9}{2}=\frac{6}{2}=3
        \end{align*}
        Por lo tanto las coordenadas del punto de intersección entre las rectas $L$ y $P$ son $(-1,3)$.
    \end{enumerate}
    \item Ayuda mucho reconocer que es la forma canónica de la parábola y aprovechar los datos que proporciona dicha forma.
    \begin{enumerate}
        \item Para calcular el valor del coeficiente $k$ se debe utilizar el punto $Q=(3,8)$ que pertenece a la parábola, entonces se debe evaluar la función en ese punto y despejar $k$:
        \begin{align*}
            8=&2\left(3-\frac{1}{2}\right)^2+k\\
            =&2\left(\frac{5}{2}\right)^2+k\\
            =&\frac{25}{2}+k\\
            k=&8-\frac{25}{2}=-\frac{9}{2}
        \end{align*}
        Por lo tanto $k=-\frac{9}{2}$ y la parábola en su forma polinómica es:
        \begin{align*}
        P: y=&2\left(x-\frac{1}{2}\right)^2-\frac{9}{2}\\
        =&2\left(x^2-x+\frac{1}{4}\right)-\frac{9}{2}\\
        =&2x^2-2x+\frac{1}{2}-\frac{9}{2}\\
        =&2x^2-2x-\frac{8}{2}\\
        =&2x^2-2x-4
        \end{align*} 
        Por lo tanto $P: y= 2x^2-2x-4$
        \item Como pide calcular las coordenadas, conozco que el eje de las ordenadas es el eje $y$ y lo intersecciona la parábola en su ordenada al origen, en la forma polinómica es $c=-4$, es decir, interseccionan en $(0,-4)$, y en el eje de las abscisas es el eje $x$ y lo intersecciona la parábola en sus raíces, basta usar Baskara:
        \begin{align*}
            x_{1,2}=&\frac{2\pm\sqrt{4-4\cdot 2 \cdot (-4)}}{2\cdot 2}\\
            =&\frac{2\pm\sqrt{4+32}}{4}\\
            =&\frac{2\pm\sqrt{36}}{4}\\
            =&\frac{2\pm 6}{4}
        \end{align*}
        Por lo tanto $x_1=\frac{8}{4}=2$ y $x_2=\frac{-4}{4}=-1$, entonces las coordenadas de los puntos de intersección con el eje de las abscisas son $(-1,0)$ y $(2,0)$.
        \item Para calcular las coordenadas del vértice de la parábola, se puede usar la fórmula del vértice en la forma canónica: $x_v=-\frac{b}{2a}$, donde $a=2$, $b=-2$, y $c=-4$. Entonces:
        \begin{align*}
            x_v=&-\frac{-2}{2\cdot 2}=\frac{1}{2}\\
            y_v=&P(x_v)=P\left(\frac{1}{2}\right)=2\left(\frac{1}{2}\right)^2-2\left(\frac{1}{2}\right)-4= \frac{1}{2}-1-4=-\frac{9}{2}
        \end{align*}
        Si bien eran datos ya conocidos por su forma canónica, está bueno verificarlo para cersiorar que nuestros cálculos tengan sentido.
        \item El gráfico es:
        \begin{figure}[H]
                \centering
                \includegraphics[scale=0.5]{Imagenes/cuadraticaG.png}
                \label{fig:funcion Cuadratica}
        \end{figure}
    \end{enumerate}
    \item 
        \begin{enumerate}
            \item Como coseno nos suele indicar las coordenadas en $x$ de algún punto de un ángulo, es posible intuir que al ser negativo tenemos dos opciones, que el coseno se encuentre en el segundo o el tercer cuadrante. Además con las identidades trigonométricas podemos obtener otras operaciones necesarias y sabiendo al cuadrante al que pertenecen también podemos intuir el signo que tienen.
            \begin{align*}
                1=&cos^2(t)+sen^2(t)\\
                1=&\left(-\frac{2}{3}\right)^2+sen^2(t)\\
                1-\frac{4}{9}=&sen^2(t)\\
                \sqrt{\frac{5}{9}}=&sen(t)\\
                \pm\frac{\sqrt{5}}{3}=&sen(t)
            \end{align*}
            Como se encuentra en el tercer cuadrante, entonces seno también es negativo, por lo tanto $sen(t)=-\frac{\sqrt{5}}{3}$
            \begin{itemize}
                \item $tg(t)=\frac{sen(t)}{cos(t)}=\frac{\sqrt{5}}{2}$
                \item $cosec(t)=\frac{1}{sen(t)}=-\frac{3}{\sqrt{5}}$
                \item $sec(t)=\frac{1}{cos(t)}=-\frac{3}{2}$
                \item $cotg(t)=\frac{1}{tg(t)}=\frac{2}{\sqrt{5}}$
            \end{itemize}
            \item Tratamos de usar propiedades de los radianes en funciones trigonometricas para ponerlo en valores que conocemos:
            \begin{equation*}
                \frac{11}{3}\pi=\frac{2}{3}\pi+\pi+2\pi
            \end{equation*}
            \begin{align*}
                P\left(\frac{11}{3}\pi\right)=&\left(cos\left(\frac{11}{3}\pi\right),sen\left(\frac{11}{3}\pi\right)\right)\\
                =&\left(cos\left(\frac{2}{3}\pi+\pi+2\pi\right),sen\left(\frac{2}{3}\pi+\pi+2\pi\right)\right)\\
                =&\left(cos\left(\frac{2}{3}\pi+\pi\right),sen\left(\frac{2}{3}\pi+\pi\right)\right)\\
                =&\left(-cos\left(\frac{2}{3}\pi\right),-sen\left(\frac{2}{3}\pi\right)\right)\\
            \end{align*}
            De acá podemos ver que $sen\left(\frac{2}{3}\pi\right)$ está a la misma altura que $sen\left(\frac{1}{3}\pi\right)$ y que $cos\left(\frac{2}{3}\pi\right)$ está a la misma distancia del origen que $cos\left(\frac{1}{3}\pi\right)$ sólo que con signo cambiado
            \begin{align*}
                \left(-cos\left(\frac{2}{3}\pi\right),-sen\left(\frac{2}{3}\pi\right)\right)=&\left(-\left(-cos\left(\frac{1}{3}\pi\right)\right),-sen\left(\frac{1}{3}\pi\right)\right)\\
                =&\left(-\left(-\frac{1}{2}\right),-\left(\frac{\sqrt{3}}{2}\right)\right)\\
                =&\left(\frac{1}{2},-\frac{\sqrt{3}}{2}\right)
            \end{align*}
            Entonces las coordenadas de $P\left(\frac{11}{3}\pi\right)$ son $\left(\frac{1}{2},-\frac{\sqrt{3}}{2}\right)$
            \item Arranquemos con la interpretación gráfica, $210^{\circ}$ desde el eje de abscisas en sentido horario termina en un ángulo en el tercer cuadrante:
            \begin{figure}[H]
                \centering
                \includegraphics[scale=0.5]{Imagenes/angulop2.png}
            \end{figure}
            Además $210^{\circ}=180^{\circ}+30^{\circ} \implies \pi+\frac{1}{6}\pi=\frac{7}{6}\pi$ es su equivalente en radianes y conocemos que la tangente del ángulo que se forma es la pendiente de una recta. Tengo dos opciones, considerar este ángulo o hallar otro para calcular la tangente de ese nuevo valor. Continuemos con este:
            \begin{align*}
                a=tg(210^{\circ})=&tg\left(\frac{7}{6}\pi\right)=\frac{sen\left(\frac{7}{6}\pi\right)}{cos\left(\frac{7}{6}\pi\right)}\\
                =&\frac{sen\left(\frac{1}{6}\pi+\pi\right)}{cos\left(\frac{1}{6}\pi+\pi\right)}\\
                =&\frac{-sen\left(\frac{1}{6}\pi\right)}{-cos\left(\frac{1}{6}\pi\right)}\\
                =&\frac{-\frac{1}{2}}{-\frac{\sqrt{3}}{2}}=\frac{1}{2}\cdot \frac{2}{\sqrt{3}}=\frac{1}{\sqrt{3}}
            \end{align*}
            Por lo tanto la pendiente de la recta es $a=\frac{1}{\sqrt{3}}$
        \end{enumerate}
        \item Por interpretación tenemos lo siguiente
        \begin{enumerate}
            \item los triangulo son de la forma:\\
            Tomi:
            \begin{figure}[H]
                \centering
                \includegraphics[scale=0.5]{Imagenes/triangulo tomi.png}
            \end{figure}
            Juli:
            \begin{figure}[H]
                \centering
                \includegraphics[scale=0.5]{Imagenes/triangulo juli.png}
            \end{figure}
            \item Vemos en los triángulos que comparten un lado y parte del otro que vienen a ser los catetos de los triangulos. Por razones trigonométricas obtenemos las siguientes ecuaciones:
            \begin{enumerate}
                \item Tomi:
                \begin{itemize}
                    \item $cos(60^{\circ})=\frac{D}{H_1}$
                    \item $sen(60^{\circ})=\frac{h}{H_1}$
                    \item $tg(60^{\circ})=\frac{h}{D}$
                \end{itemize}
                \item Juli:
                \begin{itemize}
                    \item $cos(30^{\circ})=\frac{D+\frac{4\sqrt{3}}{3}}{H_2}$
                    \item $sen(30^{\circ})=\frac{h}{H_2}$
                    \item $tg(30^{\circ})=\frac{h}{D+\frac{4\sqrt{3}}{3}}$
                \end{itemize}
            \end{enumerate}
            Las ecuaciones que comparten incógnitas para hacer sistema de ecuaciones son:
            \begin{equation*}
        \left\{
            \begin{array}{ll}
                tg(60^{\circ})=\frac{h}{D}\\
                tg(30^{\circ})=\frac{h}{D+\frac{4\sqrt{3}}{3}}
            \end{array}
        \right.
        \implies
        \left\{
            \begin{array}{ll}
                tg(\frac{\pi}{3})=\frac{h}{D}\\
                tg(\frac{\pi}{6})=\frac{h}{D+\frac{4\sqrt{3}}{3}}
            \end{array}
        \right.
        \implies
        \left\{
            \begin{array}{ll}
                \sqrt{3}=\frac{h}{D}\\
                \frac{1}{\sqrt{3}}=\frac{h}{D+\frac{4\sqrt{3}}{3}}
            \end{array}
        \right.
        \end{equation*}
        Resolvemos:
        \begin{equation*}
        \left\{
            \begin{array}{ll}
                \sqrt{3}=\frac{h}{D}\\
                \frac{1}{\sqrt{3}}=\frac{h}{\frac{3D+4\sqrt{3}}{3}}
            \end{array}
        \right.
        \implies
        \left\{
            \begin{array}{ll}
                \sqrt{3}\cdot D=h\\
                \frac{1}{\sqrt{3}}=\frac{h}{\frac{3D+4\sqrt{3}}{3}}
            \end{array}
        \right.
        \implies
        \left\{
            \begin{array}{ll}
                \sqrt{3}\cdot D=h\\
                \frac{1}{\sqrt{3}}=h\cdot\frac{3}{3D+4\sqrt{3}}          \end{array}
        \right.
        \end{equation*}
        Luego de uso sustitución:
        \begin{align*}
            \frac{1}{\sqrt{3}}=&D\sqrt{3}\cdot\frac{3}{3D+4\sqrt{3}}\\
            1=&\frac{D\sqrt{3}\sqrt{3}\cdot 3}{3D+4\sqrt{3}}\\
            3D+4\sqrt{3}=&D3\cdot 3\\
            4\sqrt{3}=&9D-3D\\
            4\sqrt{3}=&6D\\
            \frac{4\sqrt{3}}{6}=&D=\frac{2\sqrt{3}}{3}
        \end{align*}
        Ahora reemplazo en I:
        \begin{align*}
            \sqrt{3}\cdot D=&h\\
            \sqrt{3}\cdot \frac{2\sqrt{3}}{3}=&h\\
            \frac{2\cdot 3}{3}=&h\\
            2=h
        \end{align*}
        Y así llegamos a que la altura a la que se encuentra la bandera es de 2 metros.
        \end{enumerate}
    \end{enumerate}

\end{document}