\documentclass[a4paper]{article}
\usepackage{amsfonts}
\usepackage{amsmath}
\usepackage{mathtools}
\usepackage{amssymb}
\usepackage{multicol}
\usepackage{fullpage}
\usepackage{graphicx}
\title{Facultad de Matemática, Astronomía, Física y Computación}
\author{Curso de Nivelación 2025\\Simulacro Primer Parcial\\ GURI - La Bisagra, conduccion del CEIMAF}
\date{10 de Octubre de 2025}
\begin{document}
\maketitle
\noindent\textbf{Apellido y Nombre:}\\
\textbf{Comisión:}\\
\textbf{DNI:}
\begin{itemize}
        \item Leé cuidadosamente todas las consignas antes de comenzar.
        \item No está permitido el uso de calculadoras y/o celulares.
        \item Toda respuesta debe estar justificada, asegurate de acompañarla con su procedimiento y cuentas que realices, es evaluado como se llega a ella.
\end{itemize}

\scalebox{1.5}{
\centering
        \begin{tabular}{|c|c|c|c|c|c|c|c|c|c|c|c|c|c|c|}
                \hline
                \multicolumn{2}{|c|}{1} & 2 & \multicolumn{2}{|c|}{3} & \multicolumn{3}{|c|}{4} & \multicolumn{2}{|c|}{5} & \multicolumn{2}{|c|}{6} & \multicolumn{2}{|c|}{7} & total \\ \hline
                a & b & a & a & b & a & b & c & a & b & a & b & a & b & \ \\ \hline
                \ & \ & \ & \ & \ & \ & \ & \ & \ & \ & \ & \ & \ & \ & \ \\ \hline
        \end{tabular}
}
\begin{enumerate}
        \item (a) (Xpts) Resolver utilizando propiedades de los números y de sus operaciones:
        \begin{equation*}
                \frac{1996^2-2004^2}{2000}-\sqrt[15]{\left(\frac{12\cdot 8\cdot 2^{28} \cdot 2^{-30}}{3}\right)^5} + \frac{\left(\frac{14}{15}\right)^{-1}}{\frac{3}{2}-\frac{3}{7}}
        \end{equation*}
        (b) (Xpts) Plantear una ecuación que describa el siguiente enunciado y resolver:\\\\
         \textit{Martina tiene 66 años y sus dos nietas Lucía y Julieta tienen 8 y 9 años respectivamente. ¿Cuantos años tienen que pasar para que el doble de la suma de las edades de las niñás sea igual a la edad de Martina?}
        \item (Xpts) Calcular el cociente y resto de la división entre los polinomios $P(x)=x^5+5x^4+x^3-2x^2-10x+2$ y $Q(x)=x^4-2x$
        \item (a) (Xpts) Plantear un sistema de ecuaciones que describa el siguiente problema:\\\\
        \textit{Para evaluar un exámen multiplechoice de 100 preguntas, se considera que cada respuesta correcta suma un punto y las incorrectas o sin respuesta restan medio punto, Daniel obtuvo una nota de 76 sobre 100. ¿Cuantas respuestas obtuvo como correctas?¿Y como incorrectas o sin responder?}\\\\
        (b) (Xpts) Resolver el sistema planteado en el inciso anterior e indicar si es compatible o incompatible, determinado o indeterminado.
        \item (a) (Xpts)La siguiente ecuación cuadrática tiene una única raíz real doble: $k x^2 - 9x + k = 0$, calcular el valor de $k$ ¿Es único?.\\\\
        (b) (Xpts) Calcular \textbf{todas} las raíces de la ecuación $x^4 -8x^2 +16=0$.\\\\
        (c) (Xpts) Considerar la ecuación $ax^2+5x+c$. Determinar los valores de $a$ y $c$ sabiendo que la suma de sus raíces es $-5$ y el producto de sus raíces es $6$.
        \item Considerar la siguiente ecuación fraccionaria:
        \begin{equation*}
                \frac{x^4-16}{x^3-4x}=5
        \end{equation*}
        (a) (Xpts) ¿En que valores de $x$ no está definida la ecuación?¿Por qué?\\\\
        (b) (Xpts) Simplificar la ecuación y encontrar todas sus soluciones.
        \item (a) (Xpts) Indicar el valor de la verdad de la siguiente proposición justificando la respuesta, luego escribir la negación sin usar $\neg$ 
        \begin{equation*}
                \exists x \in \mathbb{R} / 2x-3=3x+1
        \end{equation*}
        (b) (Xpts) Sabiendo que $(\neg p \wedge q)$ es \textit{Falso}, $p \implies r$ es \textit{Falso} y $\neg p \iff q$ es \textit{Verdadero}. Determinar el valor de verdad de:
        \begin{itemize}
                \item $p$, $q$ y $r$.
                \item $(p \vee r) \implies (\neg q)$
        \end{itemize}
        \item Considere los siguientes conjuntos: $\mathcal{U}=\mathbb{R}$, $A=\{x \in \mathcal{U}/(4x^2-16)\cdot x=0\}$, $B=[3, \infty)$ y $C=\{x\in \mathcal{U}/-3< x \wedge 8\geq x\}$\\\\
        (a) (Xpts) Expresar por extensión al conjunto $A$, por compresión al conjunto $B$ y como intervalo al conjunto $C$.\\\\
        (b) (Xpts) Calcular $C^c$, $A \cap B$, $B-C$, expresar los resultados como intérvalos y dibujar los conjuntos solución en la recta real.
\end{enumerate}
\end{document}