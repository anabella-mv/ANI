\documentclass[a4paper]{article}

%paquetes necesarios
\usepackage[a4paper, top=1.5cm, left=2cm, right=2cm,bottom=2cm]{geometry}
\usepackage{amsfonts}
\usepackage{amsmath, scalerel}
\usepackage{mathtools}
\usepackage{amssymb}
\usepackage{multicol}
\usepackage{graphicx}
%texto del  documento
\title{Facultad de Matemática, Astronomía, Física y Computación}
\author{Curso de Nivelación 2025\\Simulacro Segundo Parcial\\ GURI - La Bisagra, conduccion del CEIMAF}
\date{21 de Noviembre de 2025}

\begin{document}
\maketitle

%Inicio parcial

%Información
\noindent\textbf{Apellido y Nombre:}\\
\textbf{Comisión:}\\
\textbf{DNI:}
\begin{itemize}
        \item Leé cuidadosamente todas las consignas antes de comenzar.
        \item No está permitido el uso de calculadoras y/o celulares.
        \item Toda respuesta debe estar justificada, asegurate de acompañarla con su procedimiento y cuentas que realices, es evaluado como se llega a ella.
\end{itemize}

%Punto 1
\begin{enumerate}
    \item Sea $f(x)$ la función definida por partes que se muestra en el siguiente gráfico:
    \begin{figure}[htb]
        \centering
        \includegraphics[scale=0.5]{gráfico.png}
        \label{fig:funcion}
    \end{figure}
    \begin{enumerate}
        \item Determinar el dominio y la imagen de la función.
        \item Determinar el conjunto de valores de x para los cuales se cumple que $f(x) \geq -1$
        \item Utilizando las transformaciones adecuadas, graficar $g(x) = -f(x + 1)$. Justificar indicando cuántas y cuáles transformaciones se aplicaron.
    \end{enumerate}
    \item Escriba el dominio de la siguiente función:
    \[
        \scaleto{f(x)=\frac{\sqrt{x+1}}{x^2-1}}{25pt}
    \]
    \item 
    \begin{enumerate}
        \item Determinar la ecucación de la recta $L$ que pasa por los puntos $A=(8,-2)$ y $B=(-4,2)$
        \item Determinar la ecuación de la recta $P$ que es perpendicular a la recta del inciso anterior que pasa por el punto $C=\left(0,-1\right)$
        \item Representar gráficamente a las rectas L y P en un único sistema de coordenadas. Calcular analíticamente las coordenadas del punto de intersección entre ambas rectas.
    \end{enumerate}
    \item Sabiendo que el gráfico de la función cuadrática $f(x)=ax^2+bx+\frac{5}{2}$ pasa por los puntos $M=\left(14,6\right)$ y $N=\left(-2,6\right)$
    \begin{enumerate}
        \item Determinar el eje de simetría de la parábola.
        \item Calcular los valores de $a$ y $b$.
        \item Calcular las coordenadas $(x_v,y_v)$ del vértice de la parábola.
        \item Encuentre las raíces de la función y donde corta el eje de las ordenadas.
        \item Utilizando la información obtenida de los incisos anteriores, esbozar el gráfico de la función.
    \end{enumerate}
    \item Considerar la función $f:\mathbb{R} \rightarrow \mathbb{R}$ definida por:
    \begin{equation*}
        f(x)=    
        \left\{
            \begin{array}{ll}
                \frac{5}{2}+2x &\text{si } x\geq 1\\
                2x^2-x-1 &\text{si } x<1
            \end{array}
            \right.
    \end{equation*}
    \begin{enumerate}
        \item Calcular $f(1)$, $f(-\frac{1}{2})$ y $f(3)$
        \item Sin utilizar tabla de valores, realice el gráfico de $f(x)$
    \end{enumerate}
    \item Sea el ángulo $t$ tal que $P(t)$ está en el cuarto cuadrante y $sen(t)=-\frac{\sqrt{3}}{2}$
    \begin{enumerate}
        \item Encuentre dentro del círculo unitario las coordenadas $(x,y)$ del punto $P(t)$, $P(t+\frac{\pi}{3})$ y $P(t+\pi)$
        \item Calcular el valor de $sec(t)$ y $cotan(t)$
        \item Dada la función $g(x)=\frac{1}{2}sen\left(2x\right)$, realice su gráfico y especifique su periodo y amplitud.
        \item La recta $S: y=ax+b$ hace un ángulo de $30^{\circ} $ con la dirección negativa del eje de las abscisas. Calcular el valor de la pendiente de la recta.
    \end{enumerate}
    \item Dos barcos veleros $A$ y $B$ se encuentran queriendo volver a la costa en la cual los espera un faro, los mismos se encuentran formando un ángulo con el mar y su vista al faro, siendo que el barco $B$ se encuentra formando un ángulo de $30^{\circ}$ mientras que el barco $A$, uno de $60^{\circ}$. A ambos los separa una distancia de 40 metros como se muestra en la figura.
    \begin{enumerate}
        \item Calcular la distancia que separa al barco $A$ del faro (d) y la altura del faro (H)
        \item Calcular la distancia entre el barco $A$ y los focos del faro (L)
    \end{enumerate}
    \begin{figure}[htb]
        \centering
        \includegraphics[scale=0.18]{punto final.png}
        \label{fig:barcos}
    \end{figure}
    \begin{figure}[htb]
                \centering
                \includegraphics[scale=0.45]{an.png}
                \label{fig:notables}
            \end{figure}
\end{enumerate}
\end{document}