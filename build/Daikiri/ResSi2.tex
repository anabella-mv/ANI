\documentclass[a4paper]{article}
\usepackage[a4paper, top=1.5cm, left=2cm, right=2cm,bottom=2cm]{geometry}
\usepackage{amsfonts}
\usepackage{amsmath}
\usepackage{mathtools}
\usepackage{amssymb}
\usepackage{multicol}
\usepackage{fullpage}
\usepackage{graphicx}
\usepackage{polynom}
\title{Facultad de Matemática, Astronomía, Física y Computación}
\author{Curso de Nivelación 2025\\Simulacro Segundo Parcial\\ GURI - La Bisagra, conduccion del CEIMAF}
\date{21 de Noviembre de 2025}
\begin{document}
\maketitle
\section{Resolución}
\begin{enumerate}
    \item
    \begin{enumerate}
        \item Visualmente se aprecia una función por partes en la cual hay puntos abiertos y cerrados, vemos como el dominio es $[-4,4)$, abierto al final ya que en $4$ hay un punto abierto y la imagen es $[-2,2]$.
        \item El conjunto de valores de $x$ para los que se cumple que $f(x)\leq-1$ son $[-4,-3] \cup [-1,4)$
        \item Por como está definida la función $g(x)=-f(x+1)$, podemos ver que en primer lugar tiene un desplazamiento a la izquierda de $1$ y luego se refleja respecto al eje $x$ dado por el $-1$ que multiplica a $f$.
        \begin{enumerate}
            \item Desplazamiento a la izquierda de $1$:
            \begin{figure}[htb]
                \centering
                \includegraphics[scale=0.5]{izquierda.png}
                \label{fig:izquierda}
            \end{figure}
            \item Reflexión respecto al eje $x$:
            \begin{figure}[htb]
                \centering
                \includegraphics[scale=0.35]{reflejo.png}
                \label{fig:reflejo}
            \end{figure}
        \end{enumerate}
    \end{enumerate}
    \item Podemos ver que en el denominador no puede tener valores que provoquen el $0$, por lo cual $x^2-1\neq 0$:
        \begin{gather*}
            x^2-1\neq0\\
            x^2\neq1\\
            x\neq\pm 1
        \end{gather*}
        o bien tambien puede verse como diferencia de cuadrados:
        \begin{gather*}
            x^2-1\neq0\\
            (x-1)(x+1)\neq 0\\
            x\neq 1, x\neq -1
        \end{gather*}
        Otro conflicto yace en el radicando en el numerador, sabemos que debe ser siempre positivo o igual a 0:
        \begin{gather*}
            x+1\geq 0\\
            x\geq -1
        \end{gather*}
        Conociendo estas restricciones podemos definir el dominio como $(-1,1) \cup (1,\infty)$
        \item 
        \begin{enumerate}
            \item Para obtener la ecuación de una recta que pasa por dos puntos podemos en primer lugar buscar la pendiente con $A=(x_1,y_1)=(8,-2)$ y $B=(x_2,y_2)=(-4,2)$:
            \begin{gather*}
                m=\frac{y_2-y_1}{x_2-x_1}\\
                m=\frac{2-(-2)}{-4-8}\\
                m=\frac{4}{-12}=-\frac{1}{3}
            \end{gather*}
            La ecuación de una recta es de la forma $y=mx+b$, es decir, que una vez obtenida la pendiente podemos evaluar en cualquier punto que conozcamos para poder despejar $b$, evalúo en $A$:
            \begin{gather*}
            y_1=-\frac{1}{3}x_1+b\\
            -2=-\frac{1}{3}8+b\\
            -2+\frac{8}{3}=b\\
            \frac{-6+8}{3}=b\\
            b=\frac{2}{3}
            \end{gather*}
            \begin{figure}[htb]
                \centering
                \includegraphics[scale=0.35]{L.png}
                \label{fig:lineal}
            \end{figure}
            \item La rectas perpendiculares tienen pendientes negativas recíprocas entre sí, es decir, $m_p=-\frac{1}{m}$ por lo que la pendiente de la recta perpendicular es $m_p=-\frac{1}{-\frac{1}{3}}=3$ a lo cual sólo nos falta saber donde ubicar tal recta a lo largo de toda la recta $L$, para eso consideramos el punto que pertenece a la recta $C=(0,-1)$:
            \begin{gather*}
            y_3=3x_3+b\\
            -1=3\cdot 0+b\\
            -1=b\\
            \end{gather*}
            \item Para el punto de intersección bastaría igualar las ecuaciones con el metodo de igualación con $y$:
            \begin{gather*}
            -\frac{1}{3}x+\frac{2}{3}=3x-1\\
            \frac{2}{3}+1=3x+\frac{1}{3}x\\
            \frac{2+3}{3}=\frac{9+1}{3}x\\
            -\frac{5}{3}=\frac{10}{3}x\\
            -\frac{5}{3}\cdot \frac{3}{10}=x\\
            \frac{5}{10}=x\\
            x=\frac{1}{2}
            \end{gather*}
            Ahora puedo reemplazar ese valor de $x$ en cualquiera de las funciones de la recta y obtener la coordenada $y$:
            \begin{gather*}
            y=-\frac{1}{3}\cdot \frac{1}{2}+\frac{2}{3}\\
            y=-\frac{1}{6}+\frac{2}{3}\\
            y=\frac{-1+4}{6}\\
            y=\frac{3}{6}\\
            y=\frac{1}{2}
            \end{gather*}
            La coordenada del punto intersección es en $\left(\frac{1}{2},\frac{1}{2}\right)$
            \begin{figure}[htb]
                \centering
                \includegraphics[scale=0.35]{Imagenes/P.png}
                \label{fig:perpendicular}
            \end{figure}
        \end{enumerate}
        \item
        \begin{enumerate}
            \item Podemos ver que ambos puntos dados se encuentran a la misma altura, por lo tanto el eje de simetría se encuentra en medio de ellos, por lo tanto:
            \begin{gather*}
                x_v=\frac{x_M+x_N}{2}\\
                x_v=\frac{-2+14}{2}\\
                x_v=\frac{12}{2}\\
                x_v=6
            \end{gather*}
            \item Tanto como para $a$ como para $b$ podemos plantear un sistema de ecuaciones con los puntos dados:
            \begin{gather*}
                f(-2)=a(-2)^2+b(-2)+\frac{5}{2}=6\\
                4a-2b+\frac{5}{2}=6\\
                4a-2b=6-\frac{5}{2}\\
                4a-2b=\frac{12-5}{2}\\
                4a-2b=\frac{7}{2} \hspace{1cm} (1)
            \end{gather*}
            Ahora con el punto $N$:
            \begin{gather*}
                f(14)=a(14)^2+b(14)+\frac{5}{2}=6\\
                196a+14b+\frac{5}{2}=6\\
                196a+14b=6-\frac{5}{2}\\
                196a+14b=\frac{12-5}{2}\\
                196a+14b=\frac{7}{2}\\
                28a+2b=\frac{1}{2} \hspace{1cm} (2)
            \end{gather*}
            Usamos algún metodo para resolver el sistema:
            \begin{equation*}
                \left\{
                \begin{array}{l}
                4a-2b=\frac{7}{2}\\
                28a+2b=\frac{1}{2}
                \end{array}
                \right.
                \implies
                \left\{
                \begin{array}{l}
                2b=4a-\frac{7}{2}\\
                2b=\frac{1}{2}-28a
                \end{array}
                \right.
            \end{equation*}
            Finalmente igualo las ecuaciones:
            \begin{gather*}
                4a-\frac{7}{2}=\frac{1}{2}-28a\\
                28a+4a=\frac{1}{2}+\frac{7}{2}\\
                24a=\frac{8}{2}\\
                32a=4\\
                a=\frac{4}{32}=\frac{1}{8}
            \end{gather*}
            Reemplazo $a$ en las ecuaciones para así obtener $b$:
            \begin{gather*}
                2b=4a-\frac{7}{2}\\
                2b=4\cdot \frac{1}{8}-\frac{7}{2}\\
                2b=\frac{1}{2}-\frac{7}{2}\\
                b=\frac{-6}{2}\cdot \frac{1}{2}=-\frac{3}{2}
            \end{gather*}
            \item Como obtuve $a$ y $b$, mi función ahora se ve así: $f(x)=\frac{1}{8}x^2-\frac{3}{2}+\frac{5}{2}$, es decir que puedo evaluar cualquier punto de la parábola para obtener su coordenada en el eje de las abscisas, conociendo el eje de simetría, evalúo en $x_v$:
            \begin{gather*}
                y_v=f(6)=\frac{1}{8}(6)^2-\frac{3}{2}(6)+\frac{5}{2}\\
                y_v=\frac{36}{8}-9+\frac{5}{2}\\
                y_v=4.5-9+2.5\\
                y_v=-2
            \end{gather*}
            Las coordenadas del vértice son $(6,-2)$
            \item Para las raices de la función igualamos a $0$:
            \begin{gather*}
                \frac{1}{8}x^2-\frac{3}{2}x+\frac{5}{2}=0\\
                x^2-12x+20=0\\
                (x-10)(x-2)=0\\
                x_1=10, x_2=2
            \end{gather*}
            O bien podemos aplicar Bhaskara:
            \begin{gather*}
                x=\frac{-b\pm \sqrt{b^2-4ac}}{2a}\\
                x=\frac{3/2\pm \sqrt{(-3/2)^2-4\cdot 1/8 \cdot 5/2}}{2\cdot 1/8}\\
                x=\frac{3/2\pm \sqrt{9/4-20/16}}{1/4}\\
                x=\frac{3/2\pm \sqrt{36/16-20/16}}{1/4}\\
                x=\frac{3/2\pm \sqrt{16/16}}{1/4}\\
                x=\frac{3/2\pm 1}{1/4}\\
                x_1=\frac{5/2}{1/4}=10, x_2=\frac{1/2}{1/4}=2
            \end{gather*}
            Y para saber donde corta el eje de las ordenadas, evaluamos la función en $x=0$:
            \begin{gather*}
                f(0)=\frac{1}{8}(0)^2-\frac{3}{2}(0)+\frac{5}{2}\\
                f(0)=\frac{5}{2}
            \end{gather*} 
            \item Se ve graficado de la siguiente manera:
            \begin{figure}[htb]
                \centering
                \includegraphics[scale=0.35]{Imagenes/cuadratica.png}
                \label{fig:cuadratica}
            \end{figure}
        \end{enumerate}
        \item 
        \begin{enumerate}
            \item Evalúo en esos puntos según como está definida la función:
            \begin{itemize}
                \item $f(1)=\frac{5}{2}+2(1)=\frac{5}{2}+2=\frac{9}{2}$
                \item $f(-\frac{1}{2})=2(-\frac{1}{2})^2-(-\frac{1}{2})-1=\frac{1}{2}+\frac{1}{2}-1=0$
                \item $f(3)=\frac{5}{2}+2(3)=\frac{5}{2}+6=\frac{17}{2}$
            \end{itemize}
            \item Esbozo el gráfico conociendo los puntos evaluados y los datos que aportan la función como las ordenadas al origen en ambos intérvalo y el comportamiento de las funciones segun el coeficiente que acompaña. Además, evalúo en el punto $x=1$ para ver la continuidad de la función:
            \begin{equation*}
                f(1)=2(1)^2-1-1=0
            \end{equation*}
            Pude encontrar la otra raíz de la parte cuadrática (Podría calcular con Bhaskara como alternativa)
            \begin{figure}[htb]
                \centering
                \includegraphics[scale=0.45]{Imagenes/fpp.png}
                \label{fig:porpartes}
            \end{figure}
        \end{enumerate}
        \item Como está en el cuarto cuadrante, conozco que el coseno es positivo, además recordando ángulos notables, quien provoca en $sen(t)=\frac{\sqrt{3}}{2}$ es $60^{\circ}$ o $\frac{\pi}{3}$, lo cual hace que el ángulo que le corresponde en el cuarto cuadrante sea $360^{\circ}-60^{\circ}=300^{\circ}$ o $2\pi-\frac{\pi}{3}=\frac{5\pi}{3}$.
        \begin{enumerate}
            \item Ahora conociendo el valor de $t$, basta evaluarlo en el círculo unitario (a las ordenadas las determina el seno y a las abscisas el coseno) e interpretando el signo de las funciones:
            \begin{gather*}
                P(t)=P\left(\frac{5}{3}\pi\right)\left(cos\left(\frac{5}{3}\pi\right),-\frac{\sqrt{3}}{2}\right)=(\frac{1}{2},-\frac{\sqrt{3}}{2})\\\\
                P\left(t+\frac{\pi}{3}\right)=P\left(2\pi\right)=(1,0)\\\\    
                P(t+\pi)=P\left(\frac{8\pi}{3}\right)=P\left(\frac{2\pi}{3}\right)=\left(cos\left(\frac{2\pi}{3}\right),sen\left(\frac{2\pi}{3}\right)\right)=\left(-\frac{1}{2},\frac{\sqrt{3}}{2}\right)
            \end{gather*}
            \item \begin{gather*}
                sec(t)=\frac{1}{cos(t)}=\frac{1}{\frac{1}{2}}=2\\\\
                cotan(t)=\frac{cos(t)}{sen(t)}=\frac{\frac{1}{2}}{-\frac{\sqrt{3}}{2}}=-\frac{1}{\sqrt{3}}
            \end{gather*}
            \item Sabemos que la función $g(x)=\frac{1}{2}sen(2x)$ tiene una amplitud de $\frac{1}{2}$ (dado por el coeficiente que multiplica al seno) y un periodo de $\pi$ (dado por $\frac{2\pi}{k}$, siendo $k=2$):
            \begin{figure}[htb]
                \centering
                \includegraphics[scale=0.5]{Imagenes/seno.png}
                \label{fig:seno}
            \end{figure}
            \item En trigonometría se ve que la recta $S$ es paralela a otra recta de la misma pendiente que pasa por el origen, tal recta corta la circunferencia unitaria en dos puntos: $P(t)$ y $P(t+\pi)$, ($P(t)=(cos(t),sen(t))$), el punto pertenece a la recta mencionada, ($L:y=ax$), por lo cual:
            \begin{equation*}
                sen(t)=a\cdot cos(t) \implies a=\frac{sen(t)}{cos(t)}=tan(t)
            \end{equation*}
            Como hace un ángulo de $30^{\circ}$ con la dirección negativa del eje de las abscisas, el ángulo que forma con la dirección positiva es de $180^{\circ}-30^{\circ}=150^{\circ}=\frac{11}{6}\pi$, por lo cual:
            \begin{gather*}
                a=tan\left(\frac{11}{6}\pi\right)=-\frac{1}{\sqrt{3}}
            \end{gather*}
            Se llega pensando en seno y coseno en el circulo unitario y los ángulos notables.
        \end{enumerate}
        \item podemos ver que se forman 2 triangulos rectángulos, uno con el barco $A$ y el faro y el otro con el barco $B$ y el faro, ambos comparten la altura del faro $H$, y tienen como catetos las distancias que separan a cada barco del faro ($d$ y $d+40$ respectivamente). Podemos plantear las siguientes ecuaciones:
        \begin{enumerate}
            \item 
            \begin{gather*}
                tan(60^{\circ})=\frac{H}{d} \implies H=d\cdot tan(60^{\circ})=d\cdot \sqrt{3}\\
                tan(30^{\circ})=\frac{H}{d+40} \implies H=(d+40)\cdot tan(30^{\circ})=(d+40)\cdot \frac{1}{\sqrt{3}}
            \end{gather*}
            Igualando ambas ecuaciones:
            \begin{gather*}
                d\cdot \sqrt{3}=(d+40)\cdot \frac{1}{\sqrt{3}}\\
                3d=d+40\\
                2d=40\\
                d=20
            \end{gather*}
            Reemplazando $d$ en alguna de las ecuaciones para obtener $H$:
            \begin{gather*}
                H=20\cdot \sqrt{3}=20\sqrt{3} \text{ metros}
            \end{gather*}
            \item Para calcular la distancia entre el barco $A$ y los focos del faro ($L$), podemos usar el teorema de Pitágoras:
            \begin{gather*}
                L^2=d^2+H^2\\
                L^2=20^2+(20\sqrt{3})^2\\
                L^2=400+1200\\
                L^2=1600\\
                L=40 \text{ metros}
            \end{gather*}
        \end{enumerate} 
    \end{enumerate}
\end{document}