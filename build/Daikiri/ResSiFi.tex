\documentclass[a4paper]{article}

%paquetes necesarios
\usepackage[a4paper, top=1.5cm, left=2cm, right=2cm,bottom=2cm]{geometry}
\usepackage{amsfonts}
\usepackage{amsmath, scalerel}
\usepackage{mathtools}
\usepackage{amssymb}
\usepackage{multicol}
\usepackage{graphicx}
\usepackage{polynom}
%texto del  documento
\title{Facultad de Matemática, Astronomía, Física y Computación}
\author{Curso de Nivelación 2025\\Simulacro Segundo Parcial\\ GURI - La Bisagra, conduccion del CEIMAF}
\date{18 de Diciembre de 2025}

\begin{document}
\maketitle

\begin{enumerate}
    \item Resolución:
    \begin{gather*}
        -\sqrt[9]{\left(\frac{16}{5^{23}\cdot 10 \cdot 5^{-21}}\right)^3} - \frac{722^2-710^2}{722+710} + \left(\frac{2}{7}\right)^{21} \cdot \left(\frac{2}{7}\right)^{-23}\\
        = -\sqrt[9]{\left(\frac{2^4}{5^{23}\cdot 2 \cdot 5 \cdot 5^{-21}}\right)^3} - \frac{(722+710)(722-710)}{722+710} + \left(\frac{2}{7}\right)^{21-23}\\
        = -\left(\frac{2^3}{5^{23+1-21}}\right)^{\frac{3}{9}} - (722-710) + \left(\frac{2}{7}\right)^{-2}\\
        = -\left(\frac{2^3}{5^3}\right)^{\frac{1}{3}} - 12 + \left(\frac{7}{2}\right)^{2}\\
        = -\frac{2}{5} - 12 + \frac{49}{4}\\
        = \frac{-8 - 240 + 245}{20} = - \frac{3}{20}
    \end{gather*}
    \item Planteo:
    \begin{enumerate}
        \item Sistema de ecuaciones:
        \begin{equation*}   
        \left\{
            \begin{array}{ll}
                4V +2G = 86 & (1)\\
                V+3+G=37 & (2)
            \end{array}
        \right.
        \end{equation*}
        \item Resolución:
        \begin{equation*}   
        \left\{
            \begin{array}{ll}
                4V +2G = 86\\
                V+3+G=37 
            \end{array}
        \right.
        \implies
        \left\{
            \begin{array}{ll}
                2V + G = 43 \\
                V + 3 + G = 37
            \end{array}
        \right.
        \implies
        \left\{
            \begin{array}{ll}
                G = 43 - 2V\\
                V + 3 + 43 -2V = 37
            \end{array}
        \right.
        \end{equation*}
        \begin{equation*}
        \left\{
            \begin{array}{ll}
                G = 43 - 2V\\
                46 - 37 = V
            \end{array}
        \right.
        \implies
        \left\{
            \begin{array}{ll}
                G = 43 - 18 = 25\\
                V = 9
            \end{array}
        \right.
        \end{equation*}
    \end{enumerate}
    \item 
    \begin{enumerate}
        \item 
        \begin{itemize}
            \item $q:$ Falso, es rápido ver que el centro de la circunferencia es $A=(3,2)$
            \item $p:$ Falso
            \begin{center}
                \polylongdiv[style=D]{x^4+x^3+2x-4}{x^3-3x}
            \end{center}
            \item $r:$ Verdadero, la negación de la primera proposición es equivalente y además ambas proposiciones son verdaderas.
        \end{itemize}
        \item $V \implies F$ es falso y $V \wedge V$ es Verdadero
        \begin{enumerate}
            \item $V \equiv \lnot a$ entonces $a$ es falso.
            \item $b$ es falso.
            \item $c$ es verdadero.
            \item $c \vee a$ es $V \vee F$ entonces es verdadero.
        \end{enumerate}
    \end{enumerate}
\item 
    \begin{enumerate}
        \item Dominio de la función:
        \begin{equation*}
            x \neq 0, \quad x \neq 3
        \end{equation*}
        Resulta entonces el dominio $D_f = \mathbb{R} - \{0,3\}$
        \item Resolución de la ecuación:
        \begin{gather*}
            \frac{2x^2-18}{x-3} = \frac{x^2+x-6}{x} \\
            \frac{2(x^2-9)}{x-3} = \frac{(x+3)(x-2)}{x} \\
            \frac{2(x+3)(x-3)}{x-3} = \frac{(x+3)(x-2)}{x} \\
            2(x+3)x = (x+3)(x-2)\\
            (x+3)(2x-(x-2))=0\\
            (x+3)(x+2)=0\\
            x_1 = -3, \quad x_2 = -2
        \end{gather*}

    \end{enumerate}
\item \begin{enumerate}
    \item \begin{itemize}
        \item $A=(-3,-1]$
        \item $B=\emptyset$ Pues ser negativo y natural a la vez no es posible.
        \item $C=\{x \in \mathcal{U}/ x \geq 1\}$
    \end{itemize}
    \item \begin{itemize}
        \item $D=A^{c}=(-1,\infty)$
        \begin{figure}[htb]
        \centering
        \includegraphics[scale=0.4]{Acom.png}
        \label{fig:AC}
    \end{figure}
        \item $E=A \cup C = (-3,-1] \cup [1,\infty)$
        \begin{figure}[htb]
        \centering
        \includegraphics[scale=0.4]{AUC.png}
        \label{fig:AUC}
    \end{figure}
        \item $F=A-C = (-3,-1]=A$
        \begin{figure}[htb]
        \centering
        \includegraphics[scale=0.3]{AmenosC.png}
        \label{fig:AmenosC}
    \end{figure}
    \end{itemize}
\end{enumerate}
\item Sean las rectas $L_1$ y $L_2$ definidas por las ecuaciones:
    \[
        L_1: y = ax + b, \qquad \qquad L_2: y = \frac{1}{3}x + c
    \]
\begin{enumerate}
    \item Evalúo el punto P en ambas rectas y además conozco que son perpendiculares los que no da que la pendiente de $L_1$ es $-3$.
    \begin{equation*}
        \left\{
            \begin{array}{ll}
                \frac{2}{3} = -3(-1) + b\\
                \frac{2}{3} = \frac{1}{3}(-1) + c
            \end{array}
        \right.
        \implies
        \left\{
            \begin{array}{ll}
                b = -\frac{7}{3}\\
                c = 1\\
                a = -3
            \end{array}
        \right.
    \end{equation*}
    Las intersecciones con los ejes coordenados son:
    \begin{itemize}
        \item Eje x:
        \begin{equation*}
            L_1: 0 = -3x - \frac{7}{3} \implies x = -\frac{7}{9}
        \end{equation*}
        \begin{equation*}
            L_2: 0 = \frac{1}{3}x + 1 \implies x = -3
        \end{equation*}
        \item Eje y:
        \begin{equation*}
            L_1: y = -3(0) - \frac{7}{3} = -\frac{7}{3}
        \end{equation*}
        \begin{equation*}
            L_2: y = \frac{1}{3}(0) + 1 = 1
        \end{equation*}
    \end{itemize}
    \item Verifico si $A$ pertenece a la recta $L_2$:
    \begin{equation*}
        A: \frac{1}{3} \neq \frac{1}{3}\left(-\frac{1}{9}\right) + 1 = -\frac{1}{27} + 1 = \frac{26}{27} \quad \textbf{No pertenece}
    \end{equation*}
    Ahora verifico si $B$ pertenece a la recta $L_1$
    \begin{equation*}
        B: -2 = -3\left(-\frac{1}{9}\right) - \frac{7}{3}=\frac{1}{3} - \frac{7}{3} =-2\quad \textbf{Pertenece}
    \end{equation*}
    La distancia entre ambos puntos se obtiene mediante la fórmula de distancia:
    \begin{gather*}
        d = \sqrt{(x_2 - x_1)^2 + (y_2 - y_1)^2} = \sqrt{\left(-\frac{1}{9} + \frac{1}{9}\right)^2 + (-2 - \frac{1}{3})^2} = \sqrt{0^2 + \left(-\frac{7}{3}\right)^2} = \sqrt{\frac{49}{9}} = \frac{7}{3}
    \end{gather*}
\end{enumerate}
\item Sea la parábola $P: y = 2(x-3)^2 + k$ que pasa por el punto $Q=(2,-6)$.
\begin{enumerate}
    \item Calculo el valor de $k$:
        \begin{equation*}
            -6= 2(2-3)^2 + k \implies -6 = 2(1) + k \implies k = -6 -2 = -8
        \end{equation*}
    \item Coordenadas del vértice:
        \begin{equation*}
            x_v = 3, \quad y_v = 2(3-3)^2 -8 =-8 \implies (x_v,y_v) = (3,-8)
        \end{equation*}
        Esto sale de la ecuacion de la parábola en su forma canónica: $f(x)=a(x-x_v)^2+y_v$
    \item Intersección con el eje de ordenadas:
        \begin{equation*}
            y = 2(0-3)^2 -8 = 2\cdot 9 -8 = 18-8=10 \implies (0,10)
        \end{equation*}
        Intersección con el eje de abscisas:
        \begin{equation*}
            0 = 2(x-3)^2 -8 \implies 2(x-3)^2 = 8 \implies (x-3)^2 =4 \implies x-3 = \pm 2
        \end{equation*}
        \begin{equation*}
            x_1 = 3+2=5, \quad x_2 = 3-2=1 \implies (5,0) \text{ y } (1,0)
        \end{equation*}
    \item Gráfico:
    \begin{figure}[htb]
        \centering
        \includegraphics[scale=0.4]{parabola7.png}
        \label{fig:punto7}
    \end{figure}
    \end{enumerate}
\item El punto al encontrarse sobre la circunferencia unitaria, sabemos que la primera coordenada hace referencia a $cos(t)$, podemos usar el teorema de pitágoras para encontrar el valor del $sen(t)$
\begin{enumerate}
    \item \begin{itemize}
        \item $cos(t) = -\frac{\sqrt{8}}{3}$
        \item \begin{equation*}
        1=sen^2(t) + cos^2(t) = sen^2(t) + \left(-\frac{\sqrt{8}}{3}\right)^2 = sen^2(t) + \frac{8}{9} \implies sen^2(t) = 1 - \frac{8}{9} = \frac{1}{9} \implies sen(t) = -\frac{1}{3}
        \end{equation*}
        \item \begin{equation*}
            tg(t) = \frac{sen(t)}{cos(t)} = \frac{-\frac{1}{3}}{-\frac{\sqrt{8}}{3}} = \frac{1}{\sqrt{8}}
        \end{equation*}
        \item \begin{equation*}
            sec(t) = \frac{1}{cos(t)} = \frac{1}{-\frac{\sqrt{8}}{3}} = -\frac{3}{\sqrt{8}}
        \end{equation*}
        \item \begin{equation*}
            csc(t) = \frac{1}{-\frac{1}{3}}= -3
        \end{equation*}
        \item \begin{equation*}
            cotg(t) = \frac{1}{tg(t)} = \sqrt{8}
        \end{equation*}
        \end{itemize}
    En el caso del seno, aseguramos que es en signo negativo por estar en el tercer cuadrante.
    \item El punto $P(s)$ se encuentra en el tercer cuadrante, por lo que sus coordenadas son negativas y consideramos que están dadas por $cos$ y $sen$:
    \begin{equation*}
        P(s) = \left(cos\left(7\frac{\pi}{6}\right), sen\left(7\frac{\pi}{6}\right)\right) = \left(-\frac{\sqrt{3}}{2}, -\frac{1}{2}\right)
    \end{equation*}
    \item \begin{equation*}
        sen\left(7\frac{\pi}{6} + \frac{\pi}{2}\right) = sen\left(\frac{7\pi + 3\pi}{6}\right) = sen\left(\frac{10\pi}{6}\right) = sen\left(\frac{5\pi}{3}\right) = -sen\left(\frac{\pi}{3}\right) = -\frac{\sqrt{3}}{2}
    \end{equation*}
\end{enumerate}
\item Podemos ver dos triangulos formados por la altura del edificio y las distancias de las plazas a él, como los ángulos son en depresión, basta considerar su complementario como parte del tríangulo rectángulo que se puede formar, así tenemos el triangulo 1 con un lado de 100 metros, la altura del edificio y un ángulo de $45^{\circ}$, y el triangulo 2 con la altura del edificio, la distancia entre las plazas $(100m + d)$ y un ángulo de $60^{\circ}$.
\begin{itemize}
    \item En el primer triángulo:
    \begin{equation*}
        tan(45^{\circ}) = \frac{h}{100} \implies h = 100 \cdot tan(45^{\circ}) = 100 \cdot 1 = 100m
    \end{equation*}
    Sabiendo la altura del edificio, podemos usar el segundo triángulo para encontrar la distancia entre las plazas, pues sabemos comparten este lado, además de ya conocer el ángulo antes dado.\\\\
    \item En el segundo triángulo:
    \begin{equation*}
        tan(30^{\circ}) = \frac{h}{100 + d} \implies 100 + d = \frac{h}{tan(30^{\circ})} = \frac{100}{\frac{1}{\sqrt{3}}} = 100\sqrt{3} \implies d = 100\sqrt{3} - 100 = 100(\sqrt{3} - 1)m
    \end{equation*}
    Con esto determinamos que la altura del edificio es de 100 metros y la distancia entre las plazas es de $100(\sqrt{3} - 1)$ metros.
\end{itemize}
\item \begin{enumerate}
    \item La respuesta es el Consejo Superior.
    \begin{figure}[htb]
        \centering
        \includegraphics[scale=0.8]{CS.png}
        \label{fig:consejo}
    \end{figure}
    \item
    \begin{figure}[htb]
        \centering
        \includegraphics[scale=0.8]{CE.png}
        \label{fig:consejere}
    \end{figure}
\end{enumerate}
\end{enumerate}
\end{document}