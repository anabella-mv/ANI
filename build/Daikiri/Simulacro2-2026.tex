\documentclass[12pt]{article}

\usepackage[a4paper, top=0.5cm, left=2cm, right=2cm,bottom=2cm]{geometry}
\usepackage{amsfonts}
\usepackage{amsmath}
\usepackage{mathtools}
\usepackage{amssymb}
\usepackage{multicol}
\usepackage{fullpage}
\usepackage{graphicx}

\geometry{top=1.2cm}
\title{\small Facultad de Matemática, Astronomía, Física y Computación}
\author{Simulacro Segundo Parcial\\ GURI - La Bisagra, conduccion del CEIMAF}
\date{}
\begin{document}
\maketitle
\noindent\textbf{Apellido y Nombre:}\\
\textbf{Comisión:}
\begin{itemize}
        \item Leé cuidadosamente todas las consignas antes de comenzar.
        \item No está permitido el uso de calculadoras y/o celulares.
        \item Toda respuesta debe estar justificada, asegurate de acompañarla con su procedimiento y cuentas que realices, es evaluado como se llega a ella.
\end{itemize}
\begin{enumerate}
    \item Sea $m(x)$ la funcion dada por:
    \begin{equation*}
    m(x)=\frac{6x}{x^2+3x-10}
    \end{equation*}
    \begin{enumerate}
        \item Expresar por comprensión el dominio de $m(x)$.
        \item Determinar la imagen de $m(x)$ cuando $x=\frac{2}{5}$ y $x=4$.
    \end{enumerate}

\end{enumerate}

\end{document}