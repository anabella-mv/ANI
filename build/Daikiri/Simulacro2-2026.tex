\documentclass[12pt]{article}

\usepackage[a4paper, top=0.5cm, left=2cm, right=2cm,bottom=2cm]{geometry}
\usepackage{amsfonts}
\usepackage{amsmath}
\usepackage{mathtools}
\usepackage{amssymb}
\usepackage{multicol}
\usepackage{fullpage}
\usepackage{graphicx}

\geometry{top=1.2cm}
\title{\small Facultad de Matemática, Astronomía, Física y Computación}
\author{Simulacro Segundo Parcial\\ GURI - La Bisagra, conduccion del CEIMAF}
\date{}
\begin{document}
\maketitle
\noindent\textbf{Apellido y Nombre:}\\
\textbf{Comisión:}
\begin{itemize}
        \item Leé cuidadosamente todas las consignas antes de comenzar.
        \item No está permitido el uso de calculadoras y/o celulares.
        \item Toda respuesta debe estar justificada, asegurate de acompañarla con su procedimiento y cuentas que realices, es evaluado como se llega a ella.
\end{itemize}
\begin{enumerate}
    \item Sea $m(x)$ la funcion dada por:
    \begin{equation*}
    m(x)=\frac{6x}{x^2+3x-10}
    \end{equation*}
    \begin{enumerate}
        \item Expresar por comprensión el dominio de $m(x)$.
        \item Determinar la imagen de $m(x)$ cuando $x=\frac{2}{5}$ y $x=4$.
    \end{enumerate}
\item Sea $f(x)$ la función definida por partes que se muestra en el siguiente gráfico:
    \begin{figure}[htb]
        \centering
        \includegraphics[scale=0.5]{Imagenes/f2.png}
        \label{fig:funcion por partes}
    \end{figure}
\begin{enumerate}
    \item Determinar el dominio, la imágen y las raíces de la función.
    \item Determinar los valores de $x$ para los cuales $f(x)\leq 3$
    \item Determinar el valor de $x$ tal que $f(x)=4$ y determinar $f(-3)$ y $f(4)$
    \item Utilizando las transformaciones adecuadas, graficar $g(x)=-f(x+2)$. Justificar indicando cuántas y cuáles transformaciones se aplicaron.
\end{enumerate}
\item 
    \begin{enumerate}
        \item Determinar la ecuación de la recta que pasa por los puntos $A=(-5,\frac{3}{2})$ y $B=(1,\frac{15}{2})$.
        \item Sea $L:y=2+\frac{1}{2}(3x+5)$. Determinar la recta $P$ que es perpendicular a la recta $L$ y que pasa por el punto $C=(-7,7)$.
        \item Calcular \textbf{analíticamente} las coordenadas $(x_i,y_i)$ del punto de intersección entre las rectas $L$ y $P$ del inciso anterior.
    \end{enumerate}
\item Sea la parábola $P: y=2(x-\frac{1}{2})^2+k$ que pasa por el punto $Q=(3,8)$.
    \begin{enumerate}
        \item Calcular el valor del coeficiente $k$ y escribir $P$ en su forma polinómica.
        \item Calcular las coordenadas de los puntos de intersección de la parábola con el eje de las ordenadas y con el eje de las abscisas.
        \item Calcular las coordenadas $(x_v,y_v)$ del vértice de la parábola.
        \item Utilizando todos los datos obtenidos en los incisos anteriores, esbozar el gráfico de la función.
    \end{enumerate}
\item 
    \begin{enumerate}
        \item Sabiendo que $cos(t)=-\frac{\sqrt{3}}{3}$ y que el ángulo $t$ está en el tercer cuadrante, calcular el valor de las otras 5 funciones trigonométricas del ángulo $t$.
        \item Ubicar en la circunferencia unitaria el punto $P\left(\frac{11}{3}\pi\right)$ y dar sus coordenadas $(x,y)$ explicando qué argumentos utilizaste para encontrarlas.
        \item La recta $L: y=ax+b$ hace un ángulo de $210^{\circ}$ con la dirección positiva del eje de las ascisas. Calcular el valor de la pendiente de la recta.
    \end{enumerate}
    \item Tomi decide colgar la bandera del GURI en la entrada de FAMAF y siente curiosidad de saber a que altura está la bandera del piso una vez colgada y junto a Juli deciden observarla desde fuera de la facultad, Tomi estando a $\frac{4\sqrt{3}}{3} mts.$ de Juli tiene un ángulo de observación respecto del suelo a la bandera de $60^{\circ}$ mientras que Juli tiene de $30^{\circ}$ (ver figura).
    \begin{enumerate}
        \item Identificar en tu hoja gráficamente los triángulos (lados y ángulos) que son utilizados para resolver la situación problemática.
        \item Calcular la altura $(x)$ de la bandera respecto al suelo.
    \end{enumerate}
    \begin{figure}[htb]
        \centering
        \includegraphics[scale=0.1]{Imagenes/Tom y Juli.png}
        \label{fig:tomi y juli los amooooo}
    \end{figure}
\end{enumerate}

\end{document}