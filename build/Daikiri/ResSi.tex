\documentclass[a4paper]{article}
\usepackage{amsfonts}
\usepackage{amsmath}
\usepackage{mathtools}
\usepackage{amssymb}
\usepackage{multicol}
\usepackage{fullpage}
\usepackage{graphicx}
\title{Facultad de Matemática, Astronomía, Física y Computación}
\author{Curso de Nivelación 2025\\Simulacro Primer Parcial\\ GURI - La Bisagra, conduccion del CEIMAF}
\date{10 de Octubre de 2025}
\begin{document}
\maketitle
\section{Resolución}
\begin{enumerate}
        \item (a)
        \begin{gather*}
                \frac{(1996-2004)(1996+2004)}{2000} - \left(\frac{3\cdot 2^2\cdot 2^3 \cdot 2^{28} \cdot 2^{-30}}{3}\right)^{\frac{5}{15}}+\frac{\frac{15}{14}}{\frac{21-6}{14}}\\
                \frac{-8\cdot 4000}{2000}-\left(2^{2+3+28-30}\right)^{\frac{1}{3}}+\frac{\frac{15}{14}}{\frac{15}{14}}\\
                \frac{-8 \cdot 2}{1}-\sqrt[3]{2^{3}}+\frac{15}{14}\cdot\frac{14}{15}\\
                -16-2+1=-17
        \end{gather*}
        (b) Martine tiene 67 años, Lucía 8 y Julieta 9.\\\\
        El \textbf{doble de la suma} de las edades que están avanzando con el paso de esos $x$ años se puede ver como:
        \begin{equation*}
                2\cdot((8+x)+(9+x))=66+x
        \end{equation*}
        Se puede resolver de la siguiente manera:
        \begin{gather*}
                2\cdot(8+x)+2\cdot (9+x)=67+x\\
                16+2x+18+2x=67+x\\
                4x-x=67-34\\
                3x=33\\
                x=\frac{33}{3}\\
                x=11
        \end{gather*}
\end{enumerate}
\end{document}