\documentclass[a4paper]{article}
\usepackage{amsfonts}
\usepackage{amsmath}
\usepackage{mathtools}
\usepackage{amssymb}
\usepackage{multicol}
\usepackage{fullpage}
\usepackage{graphicx}
\usepackage{polynom}
\title{Facultad de Matemática, Astronomía, Física y Computación}
\author{Curso de Nivelación 2025\\Simulacro Primer Parcial\\ GURI - La Bisagra, conduccion del CEIMAF}
\date{10 de Octubre de 2025}
\begin{document}
\maketitle
\section{Resolución}
\begin{enumerate}
        \item (a)
        \begin{gather*}
                \frac{(1996-2004)(1996+2004)}{2000} - \left(\frac{3\cdot 2^2\cdot 2^3 \cdot 2^{28} \cdot 2^{-30}}{3}\right)^{\frac{5}{15}}+\frac{\frac{15}{14}}{\frac{21-6}{14}}\\
                \frac{-8\cdot 4000}{2000}-\left(2^{2+3+28-30}\right)^{\frac{1}{3}}+\frac{\frac{15}{14}}{\frac{15}{14}}\\
                \frac{-8 \cdot 2}{1}-\sqrt[3]{2^{3}}+\frac{15}{14}\cdot\frac{14}{15}\\
                -16-2+1=-17
        \end{gather*}
        (b) Martine tiene 67 años, Lucía 8 y Julieta 9.\\\\
        El \textbf{doble de la suma} de las edades que están avanzando con el paso de esos $x$ años se puede ver como:
        \begin{equation*}
                2\cdot((8+x)+(9+x))=67+x
        \end{equation*}
        Se puede resolver de la siguiente manera:
        \begin{gather*}
                2\cdot(8+x)+2\cdot (9+x)=67+x\\
                16+2x+18+2x=67+x\\
                4x-x=67-34\\
                3x=33\\
                x=\frac{33}{3}\\
                x=11
        \end{gather*}
        \item Resolvemos la división como se conoce.
        \begin{center}
            \polylongdiv[style=D]{x^5+5x^4+x^3-2x^2-10x+2}{x^4-2x}
        \end{center}
        Como podemos ver:
        \begin{itemize}
            \item \textbf{Cociente:} $x+5$
            \item \textbf{Resto:} $x^3+2$
        \end{itemize}
        \item (a) El total de preguntas es 100, si llamamos $x$ a las que se respondieron correctamente e $y$ a las incorrectas o en blanco, obtengo que $x+y=100$, ahora si hablamos del puntaje podemos determinar que $76$ es el total de sumar todas las correctas y restar el medio punto que da de sanción la incorrecta, es decir, $x-\frac{1}{2}y=76$
        \begin{equation*}
            \left\{
            \begin{array}{l}
                x+y=100\\
                x-\frac{1}{2}y=76
            \end{array}
            \right.
        \end{equation*}
        (b) Resolver el sistema:
        \begin{equation*}
            \left\{
            \begin{array}{l}
                x+y=100\\
                x-\frac{1}{2}y=76
            \end{array}
            \right.
            \implies
            \left\{
            \begin{array}{l}
                x=100-y\\
                x-\frac{1}{2}y=76
            \end{array}
            \right.
        \end{equation*}
        Uso sustitución en $x$ en la segunda ecuación:
        \begin{equation*}
            \left\{
            \begin{array}{l}
                x=100-y\\
                100-y-\frac{1}{2}y=76
            \end{array}
            \right.
            \implies
            \left\{
            \begin{array}{l}
                x=100-y\\
                -\frac{3}{2}y=76-100
            \end{array}
            \right.
            \implies\left\{
            \begin{array}{l}
                x=100-y\\
                y=-24 : \left(-\frac{3}{2}\right)=16
            \end{array}
            \right.
        \end{equation*}
        Por último reemplazo el valor de $y$ en la primera ecuación:
        \begin{equation*}
            \left\{
            \begin{array}{l}
                x=100-16=84\\
                y=16
            \end{array}
            \right.
        \end{equation*}
        Como llegué a una única solución podemos asegurar que el sistema de ecucaiones es \textbf{compatible} y \textbf{determinado}, con $84$ respuestas correctas y $16$ incorrectas.
        \item (a) Al hablar de una única raíz real doble, aseguramos que el discriminante sea 0:
        \begin{gather*}
            \Delta=b^2-4ac=0\\
            \Delta=(-9)^2-4\cdot k \cdot k=0\\
            \Delta=81-4k^2=0
        \end{gather*}
        Podemos notar una diferencia de cuadrados ya que $4k^2=(2k)^2$
        \begin{gather*}
            (9-2k)(9+2k)=0\\
            9-2k=0 \vee 9+2k=0
        \end{gather*}
        Despejamos k en ambos casos:
        \begin{equation*}
                k=\frac{9}{2} \vee k=-\frac{9}{2}
        \end{equation*}
        Por lo tanto, $k$ no es único.\\\\
        (b) Para calcular las raíces de la ecuación, tomemos $y=x^2$
        \begin{gather*}
            x^4-8x^2+16=0\\
            y^2-8y+16=0
        \end{gather*}
        Aplico la fórmula de Bhaskara:
        \begin{gather*}
            y=\frac{-b\pm\sqrt{b^2-4ac}}{2a}\\
            y=\frac{8\pm\sqrt{(-8)^2-4\cdot 1 \cdot 16}}{2\cdot 1}\\
            y=\frac{8\pm\sqrt{64-64}}{2}\\
            y=\frac{8\pm 0}{2}\\
            y=4
        \end{gather*}
        Ahora deshacemos el cambio de variable:
        \begin{gather*}
            x^2=4\\
            x=\pm\sqrt{4}\\
            x=\pm 2
        \end{gather*}
        (c) Sabemos que la suma de las raíces es $-5$ y el producto $6$, por lo tanto:
        \begin{gather*}
            x_1+x_2=-\frac{b}{a}=-5\\
            x_1 \cdot x_2=\frac{c}{a}=6
        \end{gather*}
        De allí podemos trabajar despejando porque conocemos $b$, pero a su vez, sabemos que sabiendo sus raíces podemos reescribir a la ecuación de la siguiente manera:
        \begin{gather*}
            ax^2+bx+c=a(x-x_1)(x-x_2)=0\\
            ax^2+5x+c=a(x^2-(x_1+x_2)x+x_1x_2)=0\\
            ax^2+5x+c=a(x^2+5x+6)=0 
        \end{gather*}
        De acá podemos concluir que $5x=a5x$ lo que implica que $a=1$ y de allí ya obtenemos $c$.
        \item (a) La ecuación no está definida cuando el denominador es 0, es decir, cuando $x^3-4x=0$. En particular podemos notar que podemos factorizarlo como $x(x^2-4)=0$, donde es más obvio cuales son las raíces del denominador: $x=0$, $x=2$ y $x=-2$. La segunda parte es usando diferencia de cuadrados ($(x^2-4)=(x-2)(x+2)$).\\\\
        (b) Simplificamos la ecuación:
        \begin{gather}
            \frac{x^4-16}{x^3-4x}=5\\
            \frac{(x^2-4)(x^2+4)}{x(x^2-4)}=5\\
            \frac{x^2+4}{x}=5\\
            x^2+4=5x\\
            x^2-5x+4=0
        \end{gather}
        Acá podemos buscar sus raíces de la manera que conocemos por ser una ecuación de segundo grado:
        \begin{gather*}
            x=\frac{-b\pm\sqrt{b^2-4ac}}{2a}\\
            x=\frac{5\pm\sqrt{(-5)^2-4\cdot 1 \cdot 4}}{2\cdot 1}\\
            x=\frac{5\pm\sqrt{25-16}}{2}\\
            x=\frac{5\pm\sqrt{9}}{2}\\
            x=\frac{5\pm 3}{2}
        \end{gather*}
        Por lo tanto, las raíces son: $x_1=\frac{8}{2}=4$ y $x_2=\frac{2}{2}=1$.
        \item (a) La proposición es \textit{Verdadera}, ya que si resolvemos la ecuación $2x-3=3x+1$ obtenemos:
        \begin{gather*}
            2x-3=3x+1\\
            2x-3x=1+3\\
            -x=4\\
            x=-4
        \end{gather*}
        Por lo tanto, existe un $x$ en los reales que cumple la ecuación. La negación de la proposición nos cambia el cuantificador y niega la función proposicional:
        \begin{equation*}
            \forall x \in \mathbb{R}, 2x-3 \neq 3x+1
        \end{equation*}
        (b) Sabemos que $(\neg p \wedge q)$ es \textit{Falso}, por lo que una o ambas son falsas, con $p \implies r$ tenemos que es falsa si $p$ es verdadera y $r$ falsa, ahí obtuve dos valores de la verdad y me queda $q$ que podemos resolverla con $\neg p \iff q$ que es verdadera únicamente cuando ambas son falsas o verdaderas, como $\neg p$ es falsa, $q$ es falsa.
        \begin{itemize}
            \item $p$ es \textit{Verdadero}, $q$ es \textit{Falso} y $r$ es \textit{Falso}.
            \item 
            \begin{gather*}
                    (V \vee F)\implies (\neg F)\\
                    V \implies V\\
                    V
            \end{gather*}
        \end{itemize}
        \item (a)
        \begin{itemize}
            \item Resuelvo la ecuación que condiciona a $A$: Identifico a $0$ como raíz y luego diferencia de cuadrados donde las raíces son las que provocan $0$ en $(2x-4)$ y $(2x+4)$, es decir, \textit{por extensión} $A=\{0,2,-2\}$
            \item $B$ \textit{por compresión} se ve como $B=\{x\in \mathbb{R}: 3\leq x\}$
            \item $C$ \textit{como intervalo} es $C=(-3,8]$
        \end{itemize}
        (b)
        \begin{itemize}
            \item $C^c=\mathcal{U}-C=(-\infty,-3]\cup(8,\infty)$
            \item $A\cap B=\{-2,0,2\} \cap [3,\infty)=\varnothing$
            \item $B-C=[3,\infty)-(-3,8]=(8,\infty)$
        \end{itemize}
    \end{enumerate}
\end{document}