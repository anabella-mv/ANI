\documentclass[12pt]{article}

\usepackage[a4paper, top=0.5cm, left=2cm, right=2cm,bottom=2cm]{geometry}
\usepackage{amsfonts}
\usepackage{amsmath}
\usepackage{mathtools}
\usepackage{amssymb}
\usepackage{multicol}
\usepackage{fullpage}
\usepackage{graphicx}

\geometry{top=1.2cm}
\title{\small Facultad de Matemática, Astronomía, Física y Computación}
\author{Simulacro Primer Parcial\\ GURI - La Bisagra, conduccion del CEIMAF}
\date{}
\begin{document}
\maketitle
\noindent\textbf{Apellido y Nombre:}\\
\textbf{Comisión:}
\begin{itemize}
        \item Leé cuidadosamente todas las consignas antes de comenzar.
        \item No está permitido el uso de calculadoras y/o celulares.
        \item Toda respuesta debe estar justificada, asegurate de acompañarla con su procedimiento y cuentas que realices, es evaluado como se llega a ella.
\end{itemize}
\begin{enumerate}
    \item 
        \begin{enumerate}
            \item Resolver las siguientes operaciones:
            \begin{enumerate}
                \item \[\frac{\left(\frac{3}{4}\right)^{-2}}{\frac{1}{9}-\frac{10}{18}}\]
                \item \[i^{8}\cdot \left(\overline{-i+2}\right)\]
                \item \[\frac{\sqrt{98}\cdot 2^{\frac{1}{2}}}{\left(\frac{14}{3}\right)^{7}\cdot \left(\left(\frac{14}{3}\right)^{-1}\right)^{5}} \cdot \frac{14}{27}\]
            \end{enumerate}
            \item  Completar la tabla indicando con una cruz a que conjunto/s pertenece cada uno de los resultados obtenidos en las operaciones del ítem anterior.
            \begin{table}[h!]
                \centering
                \resizebox{0.98\textwidth}{!}{
                \begin{tabular}{|c|c|c|c|c|c|c|}
                \hline
                \ & $\mathbb{N}$(Naturales) & $\mathbb{Z}$(Enteros) & $\mathbb{Q}$(Racionales)& $\mathbb{I}$(Irracionales)& $\mathbb{R}$(Reales) & $\mathbb{C}$(Complejos) \\ \hline
                Resultado operación I & \ & \ & \ & \ & \ & \ \\ \hline
                Resultado operación II & \ & \ & \ & \ & \ & \ \\ \hline
                Resultado operación III & \ & \ & \ & \ & \ & \ \\ \hline
            \end{tabular}
                }
            \end{table}
            \item Despejar la incógnita $G$ de la siguiente ecuación:
            \begin{equation*}
                \frac{4}{d}=\sqrt{\frac{4}{G-2}-(x+3n)}
            \end{equation*}
            \end{enumerate}
        \item A partir de los siguientes polinomios:
        \begin{equation*}
            P(x)=x^4-2x^3+5x^2-7x+3, \quad Q(x)=x^2-4x+1
        \end{equation*}
        \begin{enumerate}
            \item Resolver la operación y dar el grado del polinomio resultante:
            \begin{equation*}
                P(x)-2Q(x)-x^4-3x+1
            \end{equation*}
            \item Determinar el cociente y el resto de la división de $P(x)$ por $Q(x)$.
            \item Utilizar el teorema del resto, determinar si $P(x)$ es divisible por $x-1$.
        \end{enumerate}
        \item A partir de la siguiente situación:
        \begin{quote}
            \textit{Sara decide mandar por delivery a Cantina del CEIMAF unas galletitas y alfajores para que no falten ahora que hay mucha gente yendo a estudiar. Daiki preocupado de que sea poco pregunta por la cantidad de cada cosa, a lo que Sara dice que no se fijó pero sí en los precios: las galletitas a $\$1500$ cada una y los alfajores a $\$2500$ cada uno. Sara además recuerda que en total gastó $\$205000$ y que compró 100 unidades entre galletitas y alfajores.}
        \end{quote}
        \begin{enumerate}
            \item Plantear un sistema de ecuaciones que permita determinar cuantas galletitas y cuántos alfajores compró Sara.
            \item Resolver el sistema utilizando alguno de los métodos vistos en clase.
            \item Clasificar el sistema según el tipo de solución que presente.
        \end{enumerate}
        \item 
        \begin{enumerate}
            \item Determinar el valor de $k$ y $r$ en la ecuación $2x^2+kx+r=0$ sabiendo que la suma de sus raíces es igual a 7 y el producto de las mismas es igual a 10.
            \item Dada la siguiente ecuación $2x^4-6x^2+4=0$
            \begin{enumerate}
                \item Realizar el cambio de variable para reducir el grado de la ecuación.
                \item Encontrar las soluciones de la ecuación original.
            \end{enumerate}
        \end{enumerate}
        \item Resolver la siguiente ecuación fraccionaria, mostrando todos los pasos realizados. Para ello, factorizar los polinomios cuando sea posible, simplificar la expresión y encontrar la solución.
        \begin{equation*}
            \frac{x^2-5x+6}{x^2-4} - \frac{x-2}{x+2} = \frac{2}{x-2}
        \end{equation*}
        \item 
        \begin{enumerate}
            \item Se sabe que $\lnot s \implies a$ es verdadero y que $b \vee a$ es falsa. Determinar el valor de verdad de la proposición:
            \begin{equation*}
                \lnot (s \wedge b) \implies a
            \end{equation*}
            \item Dados los siguientes conjuntos:
            \begin{gather*}
                \mathcal{U}=\mathbb{R}\\
                 A=\{x \in \mathcal{U} \mid x^2-5x+6=0\}, B=\{x \in \mathcal{U} \mid -1<x<1\}, C=(-2,5]
            \end{gather*}
            \begin{enumerate}
                \item Expresar el conjunto $A$ por extensión, el conjunto $B$ como intérvalo y el conjunto $C$ por comprensión.
                \item Definir por intervalos $\left(A \cup B^{c}\right)^{c}$
                \item Representar en la recta numérica $A \cap C$
                \item Listar los pares ordenados de $A \times A$
            \end{enumerate}
            \item Indicar el valor de la verdad de la siguiente proposición, justificando la respuesta:
            \begin{equation*}
                \forall x \in \mathbb{Z}, \quad \exists y \in \mathbb{Z} \mid x + y = 0
            \end{equation*}
        \end{enumerate}
        \end{enumerate}
\end{document}