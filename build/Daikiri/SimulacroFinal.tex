\documentclass[a4paper]{article}

%paquetes necesarios
\usepackage[a4paper, top=1.5cm, left=2cm, right=2cm,bottom=2cm]{geometry}
\usepackage{amsfonts}
\usepackage{amsmath, scalerel}
\usepackage{mathtools}
\usepackage{amssymb}
\usepackage{multicol}
\usepackage{graphicx}
%texto del  documento
\title{Facultad de Matemática, Astronomía, Física y Computación}
\author{GURI - La Bisagra, conduccion del CEIMAF}
\date{18 de Diciembre de 2025}

\begin{document}
\maketitle

%Inicio final

%Información
\noindent\textbf{Apellido y Nombre:}\\
\textbf{Comisión:}\\
\textbf{DNI:}
\begin{itemize}
        \item Leé cuidadosamente todas las consignas antes de comenzar.
        \item No está permitido el uso de calculadoras y/o celulares.
        \item Toda respuesta debe estar justificada, asegurate de acompañarla con su procedimiento y cuentas que realices, es evaluado como se llega a ella.
\end{itemize}

\begin{enumerate}
    \item Calcular:
    \begin{equation*}
                -\sqrt[9]{\left(\frac{16}{5^{23}\cdot 10 \cdot 5^{-21}}\right)^3}+ \frac{722^2-710^2}{722+710} + \left(\frac{6}{13}\right)^{21} \cdot \left(\frac{6}{13}\right)^{-23}
    \end{equation*}
    \item Dado el siguiente enunciado:\\\\
    \textit{En una granja para la visita de turistas hay vacas y gallinas. Si se cuentan todas las patas hay un total de 86 patas, además, luego cuando se sumaron 3 vacas, pasaron a ser un total de 37 animales.}
    \begin{enumerate}
        \item Escribir el sistema de ecuaciones que representa la situación.
        \item Utilizando alguno de los métodos de igualación, sustitución o reducción, resolver el sistema de ecuaciones para encontrar cuantos animales había de cada especie si cuando se agregaron 3 conejos mas, se cuentan en total 46 animales.
    \end{enumerate}
    \item Dadas las siguientes proposiciones:
    \begin{itemize}
        \item \textit{q}: La ecuación $(x-3)^2+(y-2)^2+8=24$  describe una circunferencia con centro en el punto $A=(2,3)$ y radio $4$
        \item \textit{p}: El resto de la división entre $P(x)=x^4+x^3+2x-4$ y $Q(x)=x^3-3x$ es $R(x)=4x^2-8$
        \item \textit{r}: $\lnot (\exists x \in \mathbb{R} / x^3+1<0) \equiv \forall x \in \mathbb{R} / $$x^3+1\geq0$
    \end{itemize}
    
    \begin{enumerate}
        \item Dar el valor de la verdad de cada proposición.
        \item Conociendo que la proposición $(\lnot a \implies b)$ es falsa y que $(c \wedge \lnot b)$ es verdadera, determinar el valor de la verdad de:\\\\
        \begin{tabular}{cccc}
             (i) $a$ & (ii) $b$ & (iii) $c$ & (iv) $(c \vee a)$ \\
        \end{tabular}
    \end{enumerate}
    \item 
    \begin{enumerate}
        \item Determinar el dominio de la función:
        \begin{equation*}
            f(x)=\frac{2x^2-18}{x-3} - \frac{x^2+x-6}{x}
        \end{equation*}
        \item Factorizar, simplificar y resolver la siguiente ecuación:
        \begin{equation*}
            \frac{2x^2-18}{x-3} = \frac{x^2+x-6}{x}
        \end{equation*}
    \end{enumerate}
    \item Sea $\mathcal{U}=(-3, \infty)$ el conjunto universal. Sean $A=\{x \in \mathcal{U}/ x \leq -1\}$, $B=\{x+1/x \in \mathcal{U} \wedge x \in \mathbb{N} \wedge x<0\}$ y $C=[1,\infty)$
    \begin{enumerate}
        \item Expresar el conjunto $A$ como intérvalo, al conjunto $B$ por extensión y al conjunto $C$ por compresión.
        \item Calcular las siguientes operaciones entre conjuntos utilizando la recta real, expresar el conjunto solución como intervalo(o unión de intervalos), y realizar los diagramas de Venn que representan las operaciones:\\\\
        \begin{tabular}{cccc}
             (i) $D=A^{c}$ & (ii) $E=A \cup C$ & (iii) $F=A-C$ \\
        \end{tabular}
    \end{enumerate}
    \item Sean las rectas $L_1$ y $L_2$ definidas por las ecuaciones:
    \[
        L_1: y = ax + b, \qquad \qquad L_2: y = \frac{1}{3}x + c
    \]
    Sabiendo que cortan en el punto $P=(-1,\frac{2}{3})$ y son perpendiculares entre sí.
    \begin{enumerate}
        \item Calcular los valores de $a$, $b$ y $c$. Encontrar las coordenadas donde las rectas $L_1$ y $L_2$ intersecan con los ejes coordenados. (Eje x y eje y)
        \item Determinar si los puntos $A=(2,\frac{1}{3})$ pertenece a la recta $L_1$ y si $B=(-\frac{1}{9},-2)$ pertenece a la recta $L_2$ y calcular la distancia entre los puntos $A$ y $B$.
    \end{enumerate}
    \item Sea la parábola $P: y = 2(x-3)^2 + k$ que pasa por el punto $Q=(2,-6)$.
    \begin{enumerate}
        \item Calcular el valor del coeficiente $k$.
        \item Calcular las coordenadas $(x_v,y_v)$ del vértice de la parábola.
        \item Calcular las coordenadas de la intersección de la parábola con el eje de ordenadas y con el eje de abscisas (raíces).
        \item Utilizando los datos encontrados en los incisos anteriores, esbozar el gráfico de la parábola (no utilizar tabla de valores).
    \end{enumerate}
    \item El punto $P(t)$ se ubica sobre la circunferencia unitaria en el tercer cuadrante y tiene coordenadas $(-\frac{\sqrt{8}}{3},x)$
    \begin{enumerate}
        \item Calcular el valor de las seis funciones trigonométricas del ángulo $t$.
        \item Ubicar en la circunferencia unitaria el punto $P(s)$ con $s=7\frac{\pi}{6}$ y dar sus coordenadas $(x,y)$.
        \item Calcular $sen(s+\frac{\pi}{2})$
    \end{enumerate}
    \item Desde la terraza de un edificio se observan dos plazas a la distancia, en un ángulo de depresión de $45^{\circ}$ la primera mientras que la otra a un ángulo de depresión de $30^{\circ}$. Si la plaza más cercana se encuentra a 100 metros del edificio. Calcular la altura del edificio y la distancia entre ambas plazas.
    \begin{figure}[htb]
        \centering
        \includegraphics[scale=0.2]{edificio.png}
        \label{fig:punto9}
    \end{figure}
    \item En cada uno de los siguientes incisos, seleccione en esta hoja la casilla para indicar cúal es la recpuesta correcta:
    \begin{enumerate}
        \item ¿Cúal es el máximo órgano de co-gobierno cotidiano de la Universidad Nacional de Córdoba?
        \begin{itemize}
            \item[$\square$] El Consejo Directivo.
            \item[$\square$] El Rectorado.
            \item[$\square$] El Consejo Superior.
        \end{itemize}
        \item ¿Cuál es el único requisito académico para poder ser elegido como representante estudiantil en el Consejo Directivo?
        \begin{itemize}
            \item[$\square$] Tener promedio mayor a 8 en al menos el $30\%$ de la carrera.
            \item[$\square$] Tener aprobado por lo menos 1/3 del número de años de la carrera o un tercio 1/3 del número total de materias establecidas en el plan de estudio, indistintamente.
            \item[$\square$] No existe requisito previo para poder ser electo Consejero Estudiantil.
        \end{itemize}
    \end{enumerate}
\end{enumerate}
\end{document}