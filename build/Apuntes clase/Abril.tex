\documentclass[a4paper]{article}
\usepackage{amsfonts}
\usepackage{amsmath}
\usepackage{mathtools}
\title{Apuntes}
\author{Matemática 4to}

\begin{document}

\maketitle
\section{Temas}
\subsection{Primer Trimestre}
Factoreo. Conjuntos de números reales. Representación en recta numérica. Intervalos: abiertos, cerrados y semiabiertos. Potencia con exponente fraccionario. Propiedades de potenciación. Operaciones de potencia con exponente fraccionario. Radicales: concepto, semejantes y no semejantes. Extracción de factores de fuera del signo radical por simplificacion y por regla práctica.
\subsection{Segundo trimestre}
Suma algebráica de radicales semejantes y aparentes no semejantes. Multiplicación de radicales de igual y distinto índice. División de radicales de igual y de distinto índice. Binomio conjugado. Racionalización de denominadores.
\subsection{Tercer trimestre}
Función: dominio e imagen, codominio. Funcion lineal. Gráfico de la función. Ecuación de la recta. Distintas formas de la ecuación de la recta. Ecuación de la recta que pasa por un punto. Ecuación que pasa por una recta y es paralelo a una recta. Ecuación de una recta que pasa por un punto y es perpendicular a una recta. Ecuación de segundo grado (parábolas): Definición, elementos, clasificación, resolución y verificación.

\section{Primer trimestre}
\subsection{Factoreo}
Hay 5 casos de factoreo conocidos, su finalidad es poder facilitar las expresiones algebráicas descomponiendolas en factores primos.
\subsubsection{Factor comun}
Se puede observar que en el polinomio, al separar en términos se observen factores que se repitan en todos o algunos de ellos. Es muy importante tener en cuenta el grado de los factores, siempre debe ser el menor posible.\\
Por ejemplo:
\begin{gather*}
    xm^2 + sm - am^3 + sam = m(xm+s-am^2+sa)\\
    2b^2-32m^3=2b^2-2\cdot 16m^3=2(b^2-16m^3)
\end{gather*}

Ejercicio: $3mn^2h+9nm^3-27hm^2$

\subsubsection{Factor común en grupos}

Igual que en el anterior, puedo ver dos o más factores que se repiten en los polinomios y me permiten escribirlos distintos.\\
Por ejemplo:
\begin{align*}
    jk+jt+ak+at+bk+bt &= (jk+jt)+(ak+at)+(bk+bt) \\
    &= j(k+t)+a(k+t)+b(k+t)=(k+t)(j+a+b)\\
    &=(jk+ak+bk)+(jt+at+bt)\\
    &=k(j+a+b)+t(j+a+b)=(j+a+b)(k+t)
\end{align*}

Ejercicio: $5h^2m+8mn^2+10h+16n$

\begin{align*}
5h^2m+8mn^2+10h+16n&= (5h^2m+10h)+(8mn^2+16n)\\
&=(5h^2m+2\cdot 5h)+(8mn^2+8\cdot 2n)\\
&=5h(hm+2)+8n(mn+2)
\end{align*}
\subsubsection{Trinomio cuadrado perfecto}
Es el desarrollo del cuadrado de un binomio. Suele conmunmente asociarse a la fórmula $(a+b)^2=a^2+b^2+2ab$\\
¿De donde surge esa fórmula?:
\begin{align*}
    (a+b)^2=(a+b)(a+b)&=(a\cdot a +a \cdot b + b\cdot a +b\cdot b)\\
    &=(a^2+ab+ba+b^2) =(a^2+b^2+2ab)
\end{align*}

En caso de ser $a$ o $b$ negativas, entonces resulta en $(a-b)^2=a^2+b^2-2ab$\\

Ejercicio: $(7+a^2)^2$\\

Nota: La propiedad de las potencia: potencia de potencia sucede de la siguiente manera:
\[
(a^b)^c=a^{b\cdot c}
\]

\subsubsection{Cuatrinomio cubo perfecto}
Es el desarrollo del cubo de un binomio, similar al caso anterior
\begin{align*}
    (a+b)^3=(a+b)(a+b)(a+b)&=(a\cdot a +a \cdot b + b\cdot a +b\cdot b)(a+b)\\
    &=(a^2+ab+ba+b^2)(a+b) =(a^2+b^2+2ab)(a+b)\\
    &=(a^2a+a^2b+b^2a+b^2b+2aba+2abb)\\
    &=(a^3+a^2b+ab^2+b^3+2a^2b+2ab^2)\\
    &=(a^3+b^3+3a^2b+3ab^2)
\end{align*}

En caso de ser $a$ o $b$ negativos resulta en $(a-b)^3=a^3-b^3-3a^2b+3ab^2$\\

Ejercicio: $(3+a^2x)^3$ 

\begin{align*}
    (3+a^2x)^3&= 3^3+(a^2x)^3+3\cdot 3^2\cdot a^2x+3\cdot 3\cdot(a^2x)^2\\
    &=27+a^6x^3+27a^2x+9a^4x^2
\end{align*}


\subsubsection{Diferencia de cuadrados}
Al identificar dos números que son el cuadrado de otro puedo reescribirlos como el producto de su suma y diferencia. Es decir:

\[
    a^2-b^2=(a-b)(a+b)
\]

Es importante destacar que si multiplico lo del parentesis vuelvo al primer término. Siempre es util identificar las propiedades de potencia para identificar cuadrados.\\
Por ejemplo: 
\begin{align}
    64-m^8&=8^2-(m^4)^2 \\
    &=(8-m^4)(8+m^4)
\end{align}

Ejercicio: $9-b^4$

\subsection{Numeros reales: representación e Intervalos}
Los números reales son aquellos que contienen todos los conjuntos de números anteriormente vistos: los naturales, enteros, racionales e irracionales, ($\mathbb{R}=\left\{\mathbb{N}, \mathbb{Z}, \mathbb{Q}, \mathbb{I}\right\}$), son infinitos y en sus intervalos se encuentran infinitos números.
Corchetes y parentesis representan si el número está incluido o no en ese intervalo\\
Por ejemplo:
\begin{flushleft}
Intervalo abierto: $(2,8)$ tanto el 2 como el 8 no están incluidos en ese intervalo\\
Intervalo semiabierto: $(2,8]$ el 8 si está incluido en el intervalo\\
Intervalo cerrado: $[2,8]$ tanto 2 como 8 están incluidos
\end{flushleft}

En su representación también se guía de parentesis y corchetes sobre los números que están en el intervalo\\


\subsection{Propiedades de la potenciación}

Si bien tienen nombres complicados estas propiedades son bastantes conocidas, hay que tener muy en cuenta cuando pueden usarse y cuando no, ya sea por sumas y restas o multiplicación o división.
\subsubsection{Potencias fraccionarias}
Son aquellas potencias que mediante una fracción representan raices y potencias al mismo tiempo, el denominador hace de índice de la raíz y el numerador de exponente, es decir: 
\[a^{\frac{b}{c}}=\sqrt[c]{a^b}\]

\subsubsection{Potencia 0}

Cualquier número elevado a la potencia $0$, es $1$, es decir $a^0=1$
\[
1000^0=1, \quad 4534^0=1, \quad 12^0=1
\]
\subsubsection{Potencia 1}

Cualquier base con exponente 1 es igual a la misma base, es decir: $a^1=a$
\[
7474^1=7474, \quad 331^1=331, \quad 0^1=0
\]
\subsubsection{Potencia negativa}
al tener una potencia negativa, produce que las fracciones se inviertan para que el exponente sea positivo:
\[
a^{-1}=\left(\frac{1}{a}\right), \quad \left(\frac{r}{t}\right)^{-3}=\left(\frac{t}{r}\right)^{3}=\frac{t^3}{r^3}
\]
\subsubsection{Producto y división de potencias de igual base}

Al encontrarnos dos potencias de igual base, podemos sumar sus exponentes o restarlos dependiendo si se están multiplicando o dividiendo respectivamente. Es decir:
\begin{flushleft}
    Multiplicación: $a^b \cdot a^c=a^{b+c}$\\
    División: $a^b:a^c=\frac{a^b}{a^c}=a^{b-c}$\\
    Ambos: $a^b:a^c \cdot a^d=a^{b-c+d}$
\end{flushleft}

\subsubsection{Potencia de potencia}

Dado una potencia elevada a otra sobre sí, resulta en la multiplicacion de sus exponentes, es decir: 
\[
(a^b)^c=a^{b\cdot c}
\]
Algunos ejemplos:
\[
(c^2)^5=c^10, \quad (12^25)^0=1
\]

\subsubsection{Distribucón de potencia y raíz}

Las potencias y raíces son distribuibles sólamente si hay multiplicación o división

\begin{flushleft}
    División: $(a:b)^c=a^c:b^c$, \quad $\sqrt[n]{\left(\frac{a}{b}\right)}=\frac{\sqrt[n]{a}}{\sqrt[n]{b}}$\\
    Multiplicación: $(a\cdot b)^c=a^c\cdot b^c$, \quad $\sqrt[n]{ab}=\sqrt[n]{a}\cdot \sqrt[n]{b}$\\
    Suma y resta: $(a+b)^c \neq a^c+b^c$, \quad $(a-b)^\frac{1}{2} \neq \sqrt{a}-\sqrt{b}$
\end{flushleft}

\subsection{Fracciones: repaso}

\subsubsection{Suma y resta de fracciones:}
Es importante priorizar que el denominador sea igual para las fracciones a sumar o restarlos, por lo tanto, para igualar denominadores podemos buscar un multiplo común menor ó si el mayor es múltiplo del menor puedo tomar ese como denominador:
\[
    \frac{numerador}{denominador}
\]
\begin{flalign*}
    &(1) \quad \frac{3}{7}+\frac{6}{4}=\frac{12}{28}+\frac{42}{28}=\frac{12+42}{28}=\frac{54}{28}\\
    &(2) \quad \frac{7}{8}-\frac{1}{4}=\frac{7}{8}-\frac{2}{8}=\frac{7-2}{8}=\frac{5}{8}\\
    &(3) \quad \frac{7}{8}-\frac{1}{4} \neq \frac{7-1}{8-4}
\end{flalign*}
\subsubsection{Multiplicación y simplificación de fracciones}
Para simplificar fracciones debo trabajar con una única fracción o puedo entre varias siempre y cuando se esté multiplicando, esta simplificación se dá entre numerador y denominador, jamás entre dos numeradores y dos denominadores:

\begin{flalign*}
    (1) \quad\frac{27}{9}=\frac{3\cdot 9}{9}=\frac{3}{1}=3&&
    (2) \quad \frac{45}{5}\cdot \frac{10}{9}=\frac{5\cdot 9}{5}\cdot \frac{2\cdot 5}{9}=\frac{5}{1}\cdot \frac{2}{1}=10
\end{flalign*}

\subsubsection{División de fracciones}
Al dividir fracciones se "voltea" la segunda fracción y la división pasa a ser multiplicación:
\[
\frac{a}{b}:\frac{m}{n}=\frac{a}{b}\cdot \frac{n}{m}
\]


\subsection{Ejercicios a realizar:}

Ejemplo
\begin{align*}
   (1) \quad \sqrt{24+4\left(\frac{1}{2}\right)^2}\div \left[\frac{39}{8}+\left(\frac{1}{2}\right)^3\right] \cdot (-1)^7+\left(-\frac{1}{3}\right)^2&=\sqrt{24+4\frac{1}{4}} \div \left[\frac{39}{8}+\frac{1}{8}\right] \cdot (-1)+\frac{1}{9}\\
    &=\sqrt{24+1} \div \left[\frac{39+1}{8}\right] \cdot (-1)+\frac{1}{9}\\
    &=\sqrt{25} \div \left[\frac{40}{8}\right] \cdot (-1)+\frac{1}{9}\\
    &=5\div 5\cdot (-1)+\frac{1}{9}\\
    &=-1+\frac{1}{9}\\
    &=\frac{-9+1}{9}=-\frac{8}{9}
\end{align*}

\subsubsection{Ejercicios de factoreo:}
\begin{flalign*}
   & (1)\quad (d^3m-3)^2 \\
   & (2) \quad m^2n5+15n^2h-10n-5n^3k \\
   & (3) \quad 6ac-4ad-9bc+6bd+15c^2-10cd\\
   & (4)\quad x^2-\frac{9}{25}\\
   & (5) \quad (1+m^2p^2)^3
\end{flalign*}

\subsubsection{Ejercicios de Representación e intervalos de números reales}

Representar
\begin{flushleft}
    (1) $[2,6)$\\
    (2) intervalo entre 6 y 9 que incluya ambos números\\
    (3) $(-3,1)$\\
\end{flushleft}

Identificar el intervalo cerrado, abierto y semicerrado.
Verdadero y falso:
\begin{flushleft}
    (1) el tercer intervalo incluye a $-3$
    (2) dos intervalos contienen a $6$
    (3) Hay menos números en el primer intervalo que en el segundo.
\end{flushleft}

\subsubsection{Propiedades de raíz y potencia}
Resolver:
\begin{flalign*}
    & (1) \quad (5^{-2}-4) && (2) \quad (3^2)^{\frac{1}{2}} && (3) \quad \left(\frac{2}{3}\right)^3 && (4) \quad \pi^0
\end{flalign*}

\subsubsection{Ejercicios combinados}
\begin{flalign*}
    &(1) \quad \left(\frac{1-\frac{5}{4}}{\sqrt[3]{-\frac{11}{8}}-2}+\sqrt{\left(\frac{1}{2}\right)^{-4}}\right)^{-1} \\
    &(2) \quad \left(\frac{\left(\frac{3}{5}\right)^4\left(\frac{3}{5}\right)^{-3}+1}{1-\frac{2}{3-\frac{1}{2}}}\right)^{-\frac{1}{3}}
\end{flalign*}

\subsection{Radicación}
\subsubsection{Conceptos:}
Comencemos señalando las partes de un radical:
\begin{center}
    \huge{$\sqrt[b]{a}$} 
\end{center}
\qquad Donde $a$ es el radicando, $b$ el índice y el símbolo es la raíz.\\
Cualquier número o expresión que utiliza una raíz se conoce como radical. El término radical se deriva del latín radix, que significa raíz. El radical puede describir diferentes tipos de raíces para un número. El número escrito antes del radical se conoce como número índice o grado. Este número nos ayuda a decir cuántas veces se multiplicaría el número por sí mismo para obtener el radicando. Este se considera el opuesto de un exponente, al igual que la suma es el opuesto de la resta y la división es el opuesto de la multiplicación
\subsubsection{Raíces y sus signos}
Los radicales también tienen una manera de saber que signo llevan, y está a disposición del radicando y su índice:\\*
\begin{itemize}
    \item Índice par, radicando positivo, positivo y/o negativo: $\sqrt[par]{+a}=\pm b$
    \item Índice impar, radicando positivo, positivo: $\sqrt[impar]{a}=b$
    \item Índice par, radicando negativo, no existe resultado dentro de los reales
    \item índice impar, radicando negativo, negativo: $\sqrt[impar]{-a}=-b$
\end{itemize}

\textbf{Ejercicios:}
\begin{enumerate}
    \item $\sqrt[5]{-32}$ 
    \item $\sqrt{121}$
    \item $\sqrt[6]{64}$
    \item $\sqrt[4]{-81}$
\end{enumerate}
\subsubsection{Semejanzas y no semejanzas}
Dos radicales son semejantes cuando comparten indice y radicando, lo que pueden no compartir sería el coeficiente que los acompaña. Por ejemplo:
\begin{flalign*}
    & (1) \quad \sqrt[b]{55} \quad y \quad 25\sqrt[b]{55} && (2) \quad 2\sqrt[4]{16} \quad y \quad 3\sqrt[4]{16}\\
    & (3) \quad \sqrt[9]{m^3} \quad y \quad 8\sqrt[9]{m^2} && (4) \quad 23\sqrt[2]{25} \quad y \quad 13\sqrt[3]{25}
\end{flalign*}
Donde $1$ y $2$ son semejantes y $3$ y $4$ no cumplen con la semejanza ya sea por índice o radicando, y coeficiente es aquel número que acompaña a las raíces

\subsubsection{Extracción de factores fuera de la raíz}
\begin{center}
    ¿Qué sucede cuando un radicando tiene un factor de exponente mayor o igual al índice de la raíz?
\end{center}
Primero debemos saber identificar todos los factores con sus exponentes que son parte del radicando y posteriormente analizar si es posible "extraerlos" de la raíz.\\*

\textbf{Identificación de factores:} Necesito saber todos los números primos y exponentes de cada letra involucrados en el radicando. ¿Cómo lo hago? Primero puedo factorizar los factores, es decir, aplicar los casos de factoreo y/o descomponer los números en números primos. Ejemplo:\\*

$\sqrt{25(m^3n+m^2h)}$ \quad El radicando en este caso es $25(m^3n+m^2h)$\\*
Primero el número 25 puedo factorizarlo:
\begin{tabular}{c|c}
    25 & 5 \\
    5 & 5 \\
    1&  \\        
\end{tabular}\\
Luego identifico si hay algún caso de factoreo en $m^3n+m^2h$, que es en efecto el primer caso de factoreo: factor común.\\*
\[
    m^3n+m^2h= m^2(mn+h)
\]
\textbf{Reviso índices y exponentes:} Como la raiz es aquella que es opuesta al exponente, entonces si sabemos la potencia del número puede anularse o reducirse, es decir:\\*

Nuestro radical se ve así ahora: $\sqrt{5^2\cdot m^2\cdot (mn+h)}$, además como sabemos por las propiedades de potencia y radicación, la raíz es distribuible en cuanto al producto, por lo tanto:\\*

$\sqrt{5^2}\cdot \sqrt{m^2} \cdot \sqrt{mn+h}$ cuando el índice es igual al exponente puede "cancelarse" lo cual resulta en $5\cdot m \cdot \sqrt{mn+h}$ que sería la solución. Ahora ¿Qué sucede cuando el exponente es mayor al índice? recordemos que los exponentes fraccionarios representan raíces y potencias, lo que se refiere a que pueden simplificarse como fracciones, o dividirse, si hubiese trabajado con 
\[
\sqrt[4]{2^8}=2^{\frac{8}{4}}=2^2=4
\]

En caso de tener un exponente que no es simplificable de ese modo podemos optar por separar en términos simplificables, es decir:
\begin{align*}
    \sqrt[3]{m^4} &= \sqrt[3]{m^3 \cdot m} \qquad \text{Por propiedad de la potencia se puede}\\
    &= \sqrt[3]{m^3} \cdot \sqrt[3]{m}\\
    &= m \cdot \sqrt[3]{m}
\end{align*}

\textbf{Ejercicios:}
Decir si los siguientes radicales son semejantes
\begin{enumerate}
    \item $\sqrt[3]{v^3n}$ y $\sqrt[root]{no^3}$
    \item $2\sqrt[4]{16l}$ y $\sqrt[4]{l^5}$
\end{enumerate}

\section{Segundo trimestre}
\subsection{Suma algebráica}
Que hable de sumas algebráicas hace referencia a que existen tanto números como radicales que resultan en números irracionales, aparte de los racionales que conocemos. Es pertinente aclarar que vamos a hacer uso de los ítems antes vistos:\\
Principalmente se va trabajar con factorización y las propiedades de radicación y potenciación.
\begin{itemize}
    \item \textbf{Semejantes:} como con factorizar letras, consideramos a los radicandos simplificados como factores: $2\sqrt[2]{5}+3\sqrt[2]{5}=\sqrt[2]{5}(2+3)=\sqrt[2]{5}\cdot 5$
    \item \textbf{Aparentes no semejantes:} los simplificamos como se explicó en el tema anterior y de ahí vemos si se puede aplicar la suma algebráica. 
\end{itemize}

\textbf{Ejercicios:}
\begin{enumerate}
    \item $4\sqrt{32}-7\sqrt{8}-3\sqrt{18}$
    \item $5\sqrt[3]{81}-7\sqrt[3]{24}-2\sqrt[3]{375}$
    \item $\sqrt[3]{8xy^3}-\sqrt[3]{125x^3y}-\sqrt[3]{216xy^3}+\sqrt[3]{8x^3y}$
\end{enumerate}
\subsection{Multiplicación de radicales de distinto e igual índice}
Consideremos que el índice de las raíces para poder multiplicar las raíces, no podemos introducir una dentro de la otra si no trabajamos con los mísmos índices ¿Qué sucede entonces si son distintos?
\begin{align*}
    \sqrt[2]{5} \cdot \sqrt[3]{5} &= 5^{\frac{1}{2}} \cdot 5^{\frac{1}{3}} \quad \text{(por propiedad de potencia y raíz)}\\
    &= 5^{\frac{1}{2}+\frac{1}{3}} \quad \text{(por propiedad de multiplicación de potencias de igual base)}\\
    &= 5^{\frac{3}{6}+\frac{2}{6}}=\quad 5^{\frac{5}{6}} \quad \text{(por suma de fracciones)}\\
    &= \sqrt[6]{5^5} \quad \text{(por propiedad de potencia fraccionaria)}
\end{align*}
De otro modo que podemos verlo es buscando un índice común, que sería el mínimo común múltiplo entre los índices, en este caso $6$. Entonces:
\begin{align*}
    \sqrt[2]{5} \cdot \sqrt[3]{5} &= \sqrt[6]{5^3} \cdot \sqrt[6]{5^2} \quad \text{(por propiedad de potencia fraccionaria)}\\
    &= \sqrt[6]{5^3\cdot 5^2}\\
    &= \sqrt[6]{5^{3+2}} \quad \text{(por propiedad de multiplicación de raíces de igual índice)}\\
    &= \sqrt[6]{5^5}
\end{align*}

Lo mismo sucede con multiplicación de radicales de distinta base, es decir:
\begin{align*}
    \sqrt[2]{5} \cdot \sqrt[3]{3} &= 5^{\frac{1}{2}} \cdot 3^{\frac{1}{3}} \quad \text{(por propiedad de potencia y raíz)}\\
    &= \sqrt[6]{5^3} \cdot \sqrt[6]{3^2} \quad \text{(por propiedad de potencia fraccionaria)}\\
    &= \sqrt[6]{5^3\cdot 3^2} \quad \text{(por propiedad de multiplicación de raíces de igual índice)}
\end{align*}
\textbf{Ejercicios:}
\begin{enumerate}
    \item $\sqrt{3}\cdot \sqrt[3]{3}$
    \item $\sqrt[4]{2x}\cdot \sqrt[6]{8x^2}$
    \item $\sqrt[3]{5a^2b}\cdot \sqrt[4]{10ab^3}$
\end{enumerate}
\subsection{División de radicales de igual y distinto índice}
Al igual que en la multiplicación, para dividir radicales de distinto índice debemos trabajar con un índice común, que sería el mínimo común múltiplo entre los índices.\\
Por ejemplo: Si consideramos $\sqrt[2]{5} : \sqrt[3]{5}$ podemos ver que el mínimo común múltiplo entre $2$ y $3$ es $6$, elevamos ambos índices lo que se necesite para llegar a $6$ 

\begin{align*}
    \sqrt[2]{5} : \sqrt[3]{5} &= \sqrt[2\cdot 3]{5^3} : \sqrt[3\cdot 2]{5^2} \quad \text{(por propiedad de potencia y raíz)}\\
    &= \sqrt[6]{5^3} : \sqrt[6]{5^2} \quad \text{(por propiedad de potencia fraccionaria)}\\   
    &= \sqrt[6]{5^3:5^2} \quad \text{(por propiedad de división de raíces de igual índice)}\\
    &= \sqrt[6]{5^{3-2}} \quad \text{(por propiedad de división de potencias de igual base)}\\
    &= \sqrt[6]{5}
\end{align*}
De igual manera para radicales de distinta base:
\begin{align*}
    \sqrt[2]{5} : \sqrt[3]{3} &= \sqrt[2\cdot 3]{5^3} : \sqrt[3\cdot 2]{3^2} \quad \text{(por propiedad de potencia y raíz)}\\
    &= \sqrt[6]{5^3} : \sqrt[6]{3^2} \quad \text{(por propiedad de potencia fraccionaria)}\\   
    &= \sqrt[6]{5^3:3^2} \quad \text{(por propiedad de división de raíces de igual índice)}
\end{align*}
\subsection{Binomio conjugado}

\end{document}