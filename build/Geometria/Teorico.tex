\documentclass[a4paper]{article}
\usepackage{amsfonts}
\usepackage{amsmath}
\usepackage{mathtools}
\usepackage{amssymb}
\title{Teórico}
\author{Geometría I}

\begin{document}
\maketitle
\section{Parcial 1}
\subsection{Axiomas}
\subsubsection{De tipo I: Incidencia}
\begin{enumerate}
    \item El plano es un conjunto infinito $\Pi$
    \item $\exists \mathcal{R} \subseteq \mathcal{P}(\Pi)$ tal que cada $R \in \mathcal{R}$ es un subconjunto propio de $\Pi$ que posee al menos dos elementos
    \item Dados $a,b \in \Pi$ y $a \neq b$, existe una única $R \in \mathcal{R}$ tal que $a,b \in R$ 
\end{enumerate}
\subsubsection{De tipo II: Orden}
\begin{enumerate}
    \item Toda recta en $\Pi$ posee un orden total estricto, ($\mathcal{R^{-1}}$ también lo posee), luego si en una recta $A$ tomamos los puntos $p$ y $q$ con $p \neq q$, existe un orden total estricto en $A$ tal que $p<q$
    \item Dados dos puntos distintos en una recta $A$, existe al menos un punto de $A$ entre ellos y, dado un punto cualquiera, siempre existe un punto que lo precede y otro que le sigue:
    \begin{itemize}
        \item Si $p,q \in A$ con $p<q$ entonces $\exists r \in A$ tal que $p<r<q$
        \item Si $x \in A$ entonces $\exists y,z \in A$ tal que $y<x<z$ para algún orden en $A$
    \end{itemize}
    \item Dada una recta $A$, su complemento $A^c$ está dividido en dos subconjuntos convexos disjuntos, tales que si $a$ pertenece a uno de ellos y $b$ al otro, entocnes $\overline{ab}$ corta a la recta $A$
\end{enumerate}
\subsubsection{De tipo III: Rígidez}
\begin{enumerate}
    \item La transformaciones rígidas del plano son funciones biyectivas del plano en sí mismo que mandan rectas en rectas y semirrectas en semirrectas.
    \item La composición de dos transformaciones rígidas es otra transformación rígida, la inversa de una transformación rígida es también una transformación rígida. 
    \item Si $T$ es una transformación rígida entonces:
    \begin{itemize}
        \item $\overline{ab} \equiv \overline{cd}$ y sucede que $\overline{ab} \subseteq \overline{cd}$ o viceversa, entonces $\overline{ab}= \overline{cd}$
        \item $\hat{aob} \equiv \hat{cod}$  y sucede que $sec(\hat{aob}) \subseteq sec(\hat{cod})$ o viceversa, entonces $\hat{aob} = \hat{cod}$
    \end{itemize}
    \item Dados dos pares SR-SP $(A,\alpha)$ y $(B,\beta)$, existe una única transformación rígidad tal que $T(A,\alpha)=(B,\beta)$, es decir $T(A)=B$ y $T(\alpha)=\beta$
\end{enumerate}
\subsection{Teorema 4}
\subsubsection{Enunciado}
Sea $A$ una recta, $a \in A$ y $p \notin A$ entonces $\overrightarrow{ap} \subseteq A_p$.
\subsubsection{Demostración}
Consideramos en $\overleftrightarrow{ap}$ el orden $<$ tal que $a<p$, quiero ir por el absurdo, supongo $\overrightarrow{ap}\nsubseteq A$, es decir, \textbf{existe $q \in \overrightarrow{ap}$ tal que $q \notin A_p$}.\\
Pero como $A \subset A_p$ entonces $q \in \check{A_p}$, más precisamente $q \in \check{A_p}- A$.\\
Por II.3 sabemos que $\overline{pq} \cap A \neq \varnothing$, pero $a,p,q$ están alineados, entonces $\overline{pq} \subseteq\overleftrightarrow{ap}$.\\\\
\begin{itemize}
    \item $\varnothing \neq \overline{pq} \cap A \subseteq \overleftrightarrow{ap} \cap A =\{a\} \text{ entonces } \overline{pq} \cap A=\{a\}$
    \item $\rightarrow q<a<p \text{ por definición de segmento,}$
    \item $\text{pero } q \in \overrightarrow{ap} \rightarrow a<q \text{ lo cual resulta absurdo}$
\end{itemize}
$\therefore \overrightarrow{ap} \subseteq A_p$

\end{document}