\documentclass[a4paper]{article}
\usepackage{amsfonts}
\usepackage{amsmath}
\usepackage{mathtools}
\title{Practico}
\author{Geometría I}

\begin{document}

\maketitle
\section{Practico 0}
\subsection{Ejercicio 1}
\subsubsection{$A \subseteq B \leftrightarrow A \subseteq A\cap B$}
\textbf{($\Rightarrow$)} Si $A \subseteq B$, entonces $A\subseteq(A\cap B)$, $\forall x \in A$ como $A$ está contenida o es igual a $B$, Por lo tanto $x \in B$, es decir, $x \in A\cap B$.\\
\textbf{($\Leftarrow$)} Si $A \subseteq (A\cap B)$, entonces $A \subseteq B$, Tomo cualquier $x \in A$. Por hipotesis $x \in (A\cap B)$, por lo que $x \in B$ y como el elemento es arbitrario de $A$ resulta en todo $A$ contenida en $B$.

\subsubsection{$A \subseteq B \leftrightarrow A\cup B \subseteq B$}
\textbf{($\Rightarrow$)} Si $A \subseteq B$, entonces como todo elemento de $A$ está en $B$, y la union contiene a los elementos de ambos, entonces todos los elementos de A están tambien en su union con otros conjuntos. Pero como todos los elementos de $A$ están en $B$ resulta en que los elementos de la unión son los elementos de $B$, por lo tanto $A\cup B = B$\\
\textbf{($\Leftarrow$)} Si $A\cup B \subseteq B$, entonces todos los elementos de $A\cup B$ están en $A$ y/o $B$. Entonces tenemos 4 casos:\\
Si $x \in A$ y $x \notin B$ entonces como la unión está contenida en $B$, entonces todo elemento de $A$ está contenido en $B$.\\
Si $x \in A$ y $x\notin B$, entonces no es posible ya que la unión debería estar contenida en $B$. Por lo tanto todos los elementos de $A$ están en B.

\subsubsection{$A \subseteq B \Rightarrow B=A\cup (B\setminus A)$ y $A\cap(B\setminus A)=\emptyset$}
Como estamos hablando de igualdades necesitamos demostrar la doble inclusión entre ambas proposiciones.\\
\textbf{($\subseteq$)} Sea $x \in B$, tenemos dos posibilidades:\\
Si $x \in A$, entonces $x \in A \subseteq A\cup (B\setminus A)$\\
Si $x \notin A$, entonces $x \in B\setminus A \subseteq A\cup (B\setminus A)$\\
En ambos casos $x \in A\cup (B\setminus A)$, por lo cual $B \subseteq A\cup (B\setminus A)$.\\
\textbf{($\supseteq$)} Sea $x \in A\cup (B\setminus A)$, entonces tenemos dos posibilidades, pues $A$ y $B\setminus A$ son disjuntas:\\
Si $x \in A$, entonces $x \in B$ por hipotesis.\\
Si $x \in B\setminus A$, entonces $x \in B$ por definición de diferencia.\\
En ambos casos $x \in B$. Como hay doble contencion, se concluye la igualdad.

\subsubsection{$A\cap (B\cup C)=(A\cap B)\cup(A\cap C)$}
Nuevamente demostramos la doble inclusión.\\
\textbf{($\subseteq$)} Sea $x \in A\cap (B\cup C)$, entonces $x \in A$ y $x \in (B\cup C)$. A su vez, por definicion de union $x \in B$ o $x \in C$. Viendo los dos casos:\\
$x \in A$ y $x \in B$ lo cual es la definicion de intersección $x \in A\cap B$\\
$x \in A$ y $x \in C$ lo cual es la definicion de intersección ($x \in A\cap C$). Resulta en $x$ en alguna de esas intersecciones o ambas, por lo tanto $x \in (A\cap B)\cup(A\cap C)$.\\
\textbf{($\supseteq$)} Sea $x \in (A\cap B)\cup(A\cap C)$, entonces $x \in (A\cap B)$ o $x \in (A\cap C)$. Ambas intersecciones incluyen a $A$, por lo que $x \in A$. Ahora tenemos dos casos:\\
Si $x \in (A\cap B)$, entonces $x \in B$ por lo que $x \in (B\cup C)$ y resulta en $x \in A\cap (B\cup C)$\\
Si $x \in (A\cap C)$, entonces $x \in C$ por lo que $x \in (B\cup C)$ y resulta en $x \in A\cap (B\cup C)$\\
Por doble contención se concluye la igualdad.

\subsubsection{$A\cup (B\cap C)=(A\cup B)\cap(A\cup C)$}
\textbf{($\subseteq$)} Sea $x \in A\cup (B\cap C)$, entonces $x \in A$ o $x \in (B\cap C)$:\\
Si $x \in A$, entonces $x \in (A\cup B)$ y $x \in (A\cup C)$ por lo que $x \in (A\cup B)\cap(A\cup C)$.\\
Si $x \in (B\cap C)$, entonces $x \in B$ y $x \in C$, por lo que $x \in (A\cup B)$ y $x \in (A\cup C)$, por lo que $x \in (A\cup B)\cap(A\cup C)$.\\
\textbf{($\supseteq$)} Sea $x \in (A\cup B)\cap(A\cup C)$, entonces $x \in (A\cup B)$ y $x \in (A\cup C)$, es decir:\\
$x \in A$ lo cual resulta en $x \in A\cup (B\cap C)$\\
$x \notin A$, como $x$ tiene que estar en ambas uniones, entonces al no estar en $A$, $x \in B$ y $x \in C$, por lo que $x \in (B\cap C)$ y resulta en $x \in A\cup (B\cap C)$.\\
Por doble contención se concluye la igualdad.

\subsubsection{$A \subseteq (B\cap C) \rightarrow A \subseteq B$ y $A \subseteq C$}
Sea $x \in A$, por hipotesis $A \subseteq (B\cap C)$, por lo que $x \in (B\cap C)$, es decir, $x \in B$ y $x \in C$. Como cualquier elemento arbitrario de $A$ está en $B$ y $C$, entonces todo $A$ está en $B$ y en $C$.

\subsubsection{$A\cup B \subseteq C \rightarrow A \subseteq C$ y $B \subseteq C$}
Sea $x \in A\cup B$, esto quiere decir que hay 3 casos:\\
Si $x \in A$, y $x \notin B$, entonces todos los elementos de $A$ están en $C$.\\
Si $x \notin A$, y $x \in B$, entonces todos los elementos de $B$ están en $C$.\\
Si $x \in A$ y $x \in B$, entonces todos los elementos de $A$ y $B$ están en $C$.\\
En los 3 casos se ven incluidos los conjuntos $A$ y $B$ en $C$.

\subsubsection{$A \subseteq B \leftrightarrow B^{c} \subseteq A^{c}$}
\textbf{($\Rightarrow$)} Sea $x \in A$, entonces $x \notin A^{c}$. Por hipotesis $A \subseteq B$, por lo que $x \in B$

\section{Práctico 1}
\subsection{Ejercicio 1}
Los axiomas de incidencia son:
\begin{enumerate}
    \item El plano es un conjunto infinito y sus elementos de llaman puntos.
    \item Existe $\mathcal{R} \subseteq \mathcal{P} (\pi)$ tal que cada $R \in \mathcal{R}$ es un subconjunto propio de $\pi$ que posee al menos dos elementos.
    \item Dados $a,b \in \pi$ con $a \neq b$, existe un único $R \in \mathcal{R}$ tal que $a,b \in R$.  
\end{enumerate}
\subsubsection{A}
\textbf{(1)} Sea $\pi = \mathbb{Z} \times \mathbb{Z}$, entendemos por infinito a todo conjunto que no es finito, es decir, $\forall n \in \mathbb{N}$ ninguna función $f:X\rightarrow \{1,\dots ,n\}$ es biyectiva. Más aún conocemos que los naturales son infinitos ya que no puedo crear una función inyectiva entre ese conjunto y $\{1,\dots,n\}$(basta considerar a un subconjunto $\{1,\dots,n+1\}\in \mathbb{N}$), entonces consideramos una función inyectiva con ellos para hablar de un conjunto infinito. Consideramos $f:\mathbb{N}\rightarrow\mathbb{Z} \times \mathbb{Z}$ con $f(n)=(n,0)$, es claro que $f$ es inyectiva y a su vez recorre sólo un subconjunto de $\pi$, por lo tanto $\pi$ es infinito.\\\\
\textbf{(2)} Las rectas están implícitas en este conjunto como los subconjuntos de $\pi$ de dos elementos, es decir, $W\subset \pi$ y $W$ contiene dos pares ordenados de $\mathbb{Z}$. Por definición, son subconjuntos propios de $\pi$ y poseen al menos dos elementos, mas específicamente, dos.\\\\
\textbf{(3)} Dados $a,b \in \pi$ con $a\neq b$, con como está definidas las rectas, se sabe que los subconjuntos tienen sólo dos elementos, por lo tanto no es posible que otra recta aparte de ella misma contenga esos dos elementos. Es decir, si consideramos $a,b,c \in \pi$ si son distintos no hay recta que contengan a ambos, y si existe una recta que los contenga, implica que uno es igual a los otros.
\subsubsection{B}
"Si se cumplen los axiomas de incidencia, entonces las rectas son subconjuntos infinitos del plano".\\
Si bien los axiomas nos aseguran los infinitos puntos del plano, no nos aseguran que no exista una biyección entre los elementos de un subconjunto de un conjunto infinito y un conjunto $\{1, \dots,n\}$, sirviendo como contraejemplo el inciso anterior donde las rectas sólo presentan dos elementos a los cuales puedo definir una biyección para cada recta que recorra sus elementos de modo que $f:R\rightarrow \{1,2\}$. Por lo tanto esta proposición resulta falsa.

\subsection{Ejercicio 2}
Dado $V=\mathbb{R}^3$ espacio vectorial tridimensional sobre los reales. Se considera el plano $\pi$ de subespacios unidimensionales de $V$. Estos serán los puntos. Si $W \subset V$ en un subespacio de $V$ de dimension 2, entonces el conjunto de puntos contenidos en $W$ será llamado una recta. Mostrar que este modelo cumple con los axiomas de incidencia.\\\\

\textbf{(1)} En el enunciado se denota a los puntos como los subespacios unidimensionales de V, es decir, las rectas vectoriales que pasan por el origen ya que hay un único vector que las genera. Con $n\in \mathbb{N}$, sea $<(n,1,0)>$ uno de los vectores generadores de un punto de $\pi$, es decir $X_1=\{(x,y,z) \in \mathbb{R}^3: (x,y,0)=t(n,1,0), t\in \mathbb{R}\}$ es punto de $\pi$ y $X_1\in V$, quiero ver que dados dos puntos $X_1$ y $X_2$ con $X_2=\{(x,y,z)\in \mathbb{R}^3 : (x,y,z)=t(m,1,0), t \in \mathbb{R}^3\}$ si $n\neq m$, entonces los puntos también son distintos, es decir, que son li\\
Partimos de trabajar por el absurdo, consideramos $X_1=X_2$, lo que implica que dado $(1,n,0)\in X_1$ entonces existe $r \in \mathbb{R}$ tal que $r(n,1,0)=(m,1,0)$, esto es:
\begin{equation}
    \left\{
    \begin{array}{l}
        rn=m\\
        r=1\\
        0=0
    \end{array}
    \rightarrow
    \begin{array}{l}
        r=1\\
        n=m\\
        0=0        
    \end{array}
    \right.
\end{equation} 
Lo cual resulta absurdo ya que $n\neq m$. Por lo tanto $X_1 \neq X_2$ de allí se puede demostrar una inyección con los naturales(pensamos en que todos los vectores generadores se ven dados por $f:\mathbb{N} \rightarrow V$ tal que $f(x)=(x,1,0)$), que a su vez sólo son subconjuntos de vectores que generan a $\pi$, por lo que $\pi$ es infinito.\\\\

\textbf{(2)} Las rectas están definidas como los puntos contenidos en $W\subset V$ (subespacios de dimension 2). Cada una de ellas debe ser un subconjunto propio de $\pi$ que posea al menos dos elementos de $V$, es decir 

\subsection{Guía 1:Ejercicio 9}
\subsubsection{Enunciado}
La actividad 5 debe habernos dejado como enseñanza que hay modelos de geomtría en los que se satisfacen los axiomas de incidencia y en los que las rectas no son infinitas. De ello inferimos que la infinitud de las rectas no es una consecuencia lógica de estos axiomas. Así pues, asumamos los axiomas de orden:
\begin{enumerate}
    \item Dar la definición de que un conjunto sea infinito.
    \item Que cada uno elabore una demostración de los siguientes enunciados. Las rectas tienen una cantidad infinita de puntos. Las semirrectas tienen infinitos puntos. Los segmentos tambien poseen una cantidad infinita de puntos.
    \item Compare las estrategias seguidas por cada integrante.
\end{enumerate}
\subsubsection{Solución}
\textbf{(1)} Puntualmente un conjunto $A$ es infinito si \textbf{no} es finito, se entiende que no hay biyección posible entre el conjunto y un conjunto $\{1,\dots,n\}$ con $n\in \mathbb{N}$, puedo armar una inyección $f:\mathbb{N}\rightarrow A$ también para definir la infinitud de A ya que $\mathbb{N}$ es infinito, por no haber una biyección entre el conjunto finito $\{1,\dots, n\}$ y $\mathbb{N}$.
\begin{itemize}
    \item \textbf{Las rectas tienen una cantidad infinita de puntos.}
    \item Sea $r$ una recta,  por definicion de recta tenemos 
    \item ejercicios 3, 5 y 10 con rodrigo.
\end{itemize}
\subsection{Guía 1:Ejercicio 10}
\subsubsection{Enunciado}
Dado $p \in \pi$, denotamos con $H_p$ el \textit{haz de rectas} que pasan por $p$. 
\begin{enumerate}
    \item Demostrar que todo punto del plano pertenece a alguna recta de $H_p$.
    \item No toda recta del plano, es una recta de $H_p$.
    \item Demostrar que los axiomas de orden implican que $H_p$ tiene infinitas rectas.
    \item Mostrar con un ejemplo que sin los axiomas de orden se puede construir un plano que satisfaga los axiomas de incidencia, mas no el item anterior.
\end{enumerate}
\subsubsection{Solución}
\textbf{(1)} Definimos al conjunto $H_p=\{R \in \mathcal{R}: p \in R\}$, queremos ver que dado cualquier punto $q \in \pi$ existe una recta $r \in H_p$ tal que $q \in r$. Por lo que consideramos los siguientes casos:\\
Si $p=q$, entonces cualquier recta que pase por $p$ cumple la condición ya que toda recta de $H_p$ contiene a $p$.\\
Si $p \neq q$, por el axioma 3 de incidencia, existe una única recta $r$ que contiene a ambos puntos ($I.3$), tal recta contiene a $p$ por construcción, así $r \in H_p$ y a su vez $q \in r$ por lo que queda demostrado el enunciado para todo $q \in \pi$.\\\\
\textbf{(2)} Nuevamente, sea $p \in \pi$, quiero probar que existe alguna recta en $\mathcal{R}$ que no contenga a $p$. Tratamos por el absurdo, supongamos que toda recta en el plano contiene a $p$.\\\\
Por $I.2$ conocemos que existen al menos dos puntos distintos en el plano que son parte de una recta, es decir, existen $q \in \pi$ tal que $q \neq p$, y $R$ recta que contenga a estos dos puntos y es subconjunto propio de $\pi$, lo que implica que existe al menos un punto de $\pi$ que no está en $R$. Sean $r \notin R$ un punto de $\pi$ y $N$ la recta determinada por $q$ y $r$, por el axioma $I.3$ dicha recta existe y es única.\\\\
Por lo que asumimos al principio de que toda recta contiene a $p$, entonces $p \in N$, que a su vez por unicidad de la recta $N$ coincide con $R$ pues es la recta generada por $p$ y $q$, resulta así que $N = R$ por lo tanto $r \in R$ lo cual es absurdo por como definí a $r$, este absurdo nace de asumir que todas las rectas en el plano contienen a $p$.\\\\
\textbf{(3)} Por desarrollo en $(9)$ con uso de los axiomas de los dos primeros axiomas de orden pudimos definir que las rectas tienen infinitos puntos y más aún en el inciso anterior probamos la existencia de una recta fuera de $H_p$, 
\section{Práctico 2}
\subsection{Ejercicio 1} Analizar la validez de cada una de las siguientes afirmaciones. En caso de ser verdadera probarla, y si no, dar un contraejemplo que muestre su falsedad.
\subsubsection{(a) La intersección de conjuntos convexos es convexa}
\textbf{Verdadero} Un conjunto $C$ se dice convexo si $\forall a,b \in C$, $a \neq b$ vale que $\overline{ab} \subseteq C$, sea $D$ la intersección de dos conjuntos convexos $A$ y $B$, es decir $D=A\cap B$. Sea $x,y \in D$, entonces $x,y \in A$ y $x,y \in B$, por lo que esos segmentos están contenidos en ambos conjuntos, es decir, $\overline{xy} \subseteq A$ y $\overline{xy} \subseteq B$ esto es por la convexidad de $A$ y $B$. Pero si los puntos de ese segmento pertenecen a ambos conjuntos, entonces pertenecen también a su intersección, es decir, $\overline{xy} \subseteq A\cap B = D$. Por lo que $D$ resulta convexa por ser mis $x,y$ arbitrarios de la intersección de $A$ y $B$.
\subsubsection{(b) La unión de conjuntos convexos es convexa}
\textbf{Falso} Faltaría ver demostrar o refutar el porqué los discos son convexos, podría usar los semiplanos divididos con una recta pero sin la recta que los define, entonces pasar de un semiplano a otro corta la recta\\
Sea $R$ una recta tal que defina los dos semiplanos abiertos $A$ y $B$, es decir, $A \cup B \cup R = \mathbb{R}^2$. Considero los conjuntos $A=\{(x,y) \in \mathbb{R}^2 : y > 0\}$ y $B=\{(x,y) \in \mathbb{R}^2 : y < 0\}$, ambos conjuntos son semiplanos abiertos y convexos en el plano cartesiano y $R$ es el eje $x$. Considero los puntos $a=(0,1) \in A$ y $b=(0,-1) \in B$, el segmento $\overline{ab}$ es la porción del eje $y$ que va desde el punto $(0,-1)$ hasta el punto $(0,1)$, es claro que este segmento no está contenido ni en $A$ ni en $B$ ya que $(0,0)$ es parte del segmento pues pertenece a la intersección de las semirrectas dadas por $(0,1)$ y $(0,-1)$, pero no pertenece a ninguno de los conjuntos, y por lo tanto tampoco en su unión.\\\\
\subsubsection{(c) Si $A$ es convexo entonces $A^{c}$ es convexo}
\textbf{Falso} Sea $A$ un conjunto convexo, considero $A=\{(x,y) \in \mathbb{R}^2 \}$
\subsubsection{(d) Si $A$ y $B$ son convexos, entonces $A-B$ es convexo}
\textbf{Falso} Sea $A=\{(x,y) \in \mathbb{R}^2 : x^2+y^2 < 4\}$ y $B=\{(x,y) \in \mathbb{R}^2 : x^2+y^2 \leq 1\}$, ambos conjuntos son discos en el plano cartesiano, por lo que son convexos. Considero los puntos $a=(0,2) \in A-B$ y $b=(0,-2) \in A-B$, el segmento $\overline{ab}$ es la porción del eje $y$ que va desde el punto $(0,-2)$ hasta el punto $(0,2)$, es claro que este segmento no está contenido en $A-B$ ya que pasa por el conjunto $B$, por lo tanto la resta de conjuntos no es siempre convexa.
\subsubsection{(e) Todo ángulo es convexo}
\textbf{Falso} Sea $\angle AB$ un ángulo con origen $\{o\}$, por definición de ángulo, $A$ y $B$ son semirrectas con origen en $o$. Sea $x,y \in \angle AB$, entonces $x$

\section{Práctico 2: Ejercicios 10 y 13}
\subsection{Ejercicio 10}
\subsubsection{Enunciado}
Dado un polígono convexo $P$, probar que su interior es la intersección de los interiores de todos los ángulos interiores del polígono.
\subsubsection{Solución}
\begin{itemize}
\item \textbf{P es polígono convexo:} Sean $a_1,\dots, a_n \in \Pi$ (con $n\geq 3$) numerados de tal modo que las rectas $\overline{a_1a_2}, \overline{a_2a_3}, \dots, \overline{a_na_1}$ dejen en un mismo semiplano abierto los $(n-2)$ puntos restantes. Se llama pológono convexo de vértices $a_1,\dots,a_n$ a la union de los segmentos $\overline{a_1a_2}, \overline{a_2a_3}, \dots, \overline{a_na_1}$.
\item \textbf{Interior de un polígono convexo:} Sea la intersección de los semiplanos abiertos determinados por las rectas $\overleftrightarrow{a_ia_{i+1}}$ que contienen a los demás puntos, en estos se ignora los lados del polígono.
\item \textbf{Ángulo interior:} del polígono son los ángulos de la forma $\widehat{a_{i-1}a_ia_{i+1}}$, y se los denota por $\hat{a_i}$.
\item \textbf{Interior de un ángulo:} Considerar el sector angular sin los bordes es $sec(\angle AB) - \angle AB =(A_B\cap B_A)- \angle AB$, en el caso de mis lados determinados, $A=\overrightarrow{a_ia_{i-1}}$ y $B=\overrightarrow{a_{i}a_{i+1}}$
\end{itemize}

Quiero ver que los ángulos interiores terminan formando el interior del polígono, es decir, que la intersección de los sectores angulares de los ángulos interiores del polígono es igual al interior del polígono.\\\\
Sea $P$ un polígono convexo de vértices $a_1,\dots,a_n$ e $I_P$ su interior y $sec(\hat{a_i}) -\hat{a}_i$ el interior del ángulo $\hat{a_i}$, quiero ver que:
\begin{equation}
    I_P = \bigcap_{i=1}^{n} (sec(\hat{a_i}) -\hat{a_i}) = \bigcap_{i=1}^{n} int(\hat{a_i})
\end{equation}
\textbf{($\subseteq$)} Sea $x \in I_P$, por definición de interior de un polígono convexo, $x$ pertenece a todos los semiplanos determinados por las rectas $\overleftrightarrow{a_ia_{i+1}}$ que contengan a los otros $(n-2)$ puntos, por lo tanto, esos semiplanos abiertos, contienen a las semirrectas formadas por $\overrightarrow{a_{i+1}a_{i+2}}$ ya que por definición contiene a $a_{i+2}$ (acá hay una excepción con $a_{i+1}$ pues es parte de la recta). Por lo tanto, $x$ pertenece a los semiplanos determinados por $A=\overrightarrow{a_{i-1}a_i}$ y $B=\overrightarrow{a_ia_{i+1}}$, es decir, $x \in \overleftrightarrow{A}_B$ y $x \in \overleftrightarrow{B}_A$ que forman el interior del ángulo $\hat{a_i}$, ya que no pertenece a ninguna de las rectas que definen al polígono, más aún a sus semirrectas, por lo que $x \in sec(\hat{a_i})-\hat{a_i}$. Como $x$ pertenece a todos los interiores de los ángulos interiores del polígono, entonces $x \in \bigcap_{i=1}^{n} int(\hat{a_i})$.\\\\
\textbf{($\supseteq$)} Sea $x \in \bigcap_{i=1}^{n} int(\hat{a_i})$, entonces $x \in int(\hat{a_i})$ para todo $i=1,\dots,n$. Por definición de interior de un ángulo y $A=\overrightarrow{a_ia_{i-1}}$ y $B=\overrightarrow{a_{i}a_{i+1}}$, $x \in \overleftrightarrow{A}_B$ y $x \in \overleftrightarrow{B}_A$ para todo $i=1,\dots,n$. Por lo que $x$ pertenece a todos los semiplanos determinados por las rectas $\overleftrightarrow{a_ia_{i+1}}$ que contengan a los otros $(n-2)$ puntos, es decir, $x \in I_P$.\\\\

Por doble contención se concluye la igualdad.
\subsection{Ejercicio 13}
\subsubsection{Enunciado}
Probar que las dos diagonales de un cuadrilátero convexo se cortan en un único punto y que éste es interior al cuadrilátero.
\subsubsection{Demostración}
\begin{itemize}
    \item \textbf{P es polígono convexo:} Sean $a_1,\dots, a_n \in \Pi$ (con $n\geq 3$) numerados de tal modo que las rectas $\overline{a_1a_2}, \overline{a_2a_3}, \dots, \overline{a_na_1}$ dejen en un mismo semiplano abierto los $(n-2)$ puntos restantes. Se llama pológono convexo de vértices $a_1,\dots,a_n$ a la union de los segmentos $\overline{a_1a_2}, \overline{a_2a_3}, \dots, \overline{a_na_1}$.
    \item \textbf{Diagonales del polígono:} Los segmentos que unen dos vértices no consecutivos
    \item \textbf{Interior de un polígono convexo:} Sea la intersección de los semiplanos abiertos determinados por las rectas $\overleftrightarrow{a_ia_{i+1}}$ que contienen a los demás puntos, en estos se ignora los lados del polígono.
    \item Notar que en el caso del cuadrilátero $i=1,\dots,4$.
    \item Por definición de sector de un polígono convexo, ambas diagonales quedan en él ya que son segmentos que están contenidos en todos los semiplanos convexos que conforman dicho sector, y como es intersección de convexos, el sector poligonal también lo es.
\end{itemize}
Sea un cuadrilátero convexo un polígono de vértices $a_1, a_2, a_3, a_4$ (Todos puntos distintos). Conociendo que ambas diagonales están en el interior del sector polígonal, es decir, $\overline{a_1a_3}$ y $\overline{a_2a_4}$ están totalmente contenidas en el sector poligonal, quiero ver que estos segmentos se cortan en único punto.\\\\
Considero la recta $\overleftrightarrow{a_1a_3}$ que define dos semiplanos y deja a $a_2$ y $a_4$ en dos semiplanos distintos, porque si no fuese así entonces al querer unir los segmentos para nuevamente formar el polígono, todos ellos quedarían dentro del mismo semiplano abierto, pero no así el segmento $\overline{a_1a_3}$ que termina determinado fuera del sector poligonal, lo cual es absurdo ya que las diagonales están en su interior. Absurdo que nace de asumir que $a_2$ y $a_4$ están en el mismo semiplano. \\
Ahora como sé que están en semiplanos distintos por \textbf{II.3} el segmento $\overline{a_2a_4}$ corta a la recta $\overleftrightarrow{a_1a_3}$ en un único punto, ahora ¿Qué asegura que tal intersección sea dentro del interior del polígono?
\subsection{Ejercicio 2}
\subsubsection{Enunciado}
Sea $p$ un punto de la recta $A$ del plano $\Pi$, y sea $q \in \Pi \backslash A$. Probar que la semirrecta opuesta a $\overrightarrow{pq}$ está contenida en $\check{A}_q$
\subsubsection{Demostración}
Sea $A$ la recta que divide el plano en dos semiplanos abiertos disjuntos, dos puntos no están en el mismo semiplano si y solo si el segmento que los une no interseca $A$.\\\\
Por definición de recta, sea $\overleftrightarrow{pq}=\overrightarrow{pq}\cup \check{\overrightarrow{pq}}$. Por definición de semirrecta opuesta, sea un $x \in \check{\overrightarrow{pq}}$, entonces $p \in \overline{xq}$ y a su vez $p \in A$, por lo tanto $\overline{xq}$ interseca a $A$ en $p$, y por \textbf{II.3} $x$ no está en el mismo semiplano que $q$, si no en su semiplano opuesto respecto a $A$, a su vez como mi elección de $x$ es arbitraria en la semirrecta opuesta, entonces toda la semirrecta se encuentra contenida en el semiplano $\check{A}_p$
\subsection{Ejercicio 3}
Sean $a, b, o$ tres puntos no alineados, probar las siguientes afirmaciones:
\subsubsection{(a)}
\section{Práctico 3}
\subsection{Ejercicio 1}
Sea $f: X \rightarrow Y$ biyectiva, $A,B \subseteq X$ y $C,D \subseteq Y$. Probar:
\subsubsection{(a) $f(A\cup B)=f(A)\cup f(B)$}
\textbf{($\subseteq$)} Sea $y \in f(A\cup B)$, entonces existe $x \in A\cup B$ tal que $f(x)=y$. Por definición de unión, $x \in A$ o $x \in B$. Si $x \in A$, entonces $y=f(x) \in f(A)$, y si $x \in B$, entonces $y=f(x) \in f(B)$. En ambos casos, $y \in f(A)\cup f(B)$.\\\\
\textbf{($\supseteq$)} Sea $y \in f(A)\cup f(B)$, entonces $y \in f(A)$ o $y \in f(B)$. Si $y \in f(A)$, entonces existe $x \in A$ tal que $f(x)=y$, y si $y \in f(B)$, entonces existe $x \in B$ tal que $f(x)=y$. En ambos casos, $x \in A\cup B$, por lo tanto, $y=f(x) \in f(A\cup B)$.\\\\
\subsubsection{(b) $f(A\cap B)=f(A)\cap f(B)$}
\subsection{Ejercicio 2}
Sea $T$ una transformación rígida. Porbar que:
\subsubsection{(a) $T$ lleva rectas paralelas en rectas paralelas}
Sea $R$ y $S$ dos rectas paralelas, es decir, $R \cap S = \emptyset$. Quiero ver que $T(R) \cap T(S) = \emptyset$.\\\\
Si $T(R) \cap T(S) \neq \emptyset$, entonces existe $y \in T(R) \cap T(S)$, es decir, $y \in T(R)$ y $y \in T(S)$. Por lo que existen $x_1 \in R$ y $x_2 \in S$ tales que $T(x_1)=y$ y $T(x_2)=y$. Por lo que $T(x_1)=T(x_2)$, y como $T$ es inyectiva, entonces $x_1=x_2$, lo cual es absurdo ya que $R$ y $S$ son disjuntas.\\\\
Por lo tanto, $T(R) \cap T(S) = \emptyset$, es decir, $T(R)$ y $T(S)$ son rectas paralelas.\\\\
\subsubsection{(b) $T$ lleva rectas secantes en rectas secantes}
Sea $R$ y $S$ dos rectas secantes, es decir, $R \cap S = \{p\}$. Quiero ver que $T(R) \cap T(S) = \{T(p)\}$.\\\\
Aplico en la definición de la intersección la transformación rígida y obtengo $T(R \cap S) = T(\{p\})$, además como $T$ es biyectiva y una función, puedo obtener lo siguiente $T(R) \cap T(S) = \{T(p)\}$ y por el axioma \textbf{III.1} manda a las rectas $R$ y $S$ en rectas, por lo tanto, las rectas resultantes terminan siendo secantes.
\subsubsection{(c) $T$ lleva conjuntos convexos en conjuntos convexos}
Sea $C$ un conjunto convexo, es decir, $\forall a,b \in C$, $a \neq b$ vale que $\overline{ab} \subseteq C$. Quiero ver que $T(C)$ es convexo, es decir, $\forall y_1,y_2 \in T(C)$, $y_1 \neq y_2$ vale que $\overline{y_1y_2} \subseteq T(C)$.\\\\
Sea $y_1,y_2 \in T(C)$, entonces existen $x_1,x_2 \in C$ tales que $T(x_1)=y_1$ y $T(x_2)=y_2$. Por lo que $x_1 \neq x_2$, ya que si no fuese así, $y_1=y_2$, lo cual es absurdo.\\\\
Por lo que $\overline{x_1x_2} \subseteq C$, y aplicando la transformación rígida, $T(\overline{x_1x_2}) \subseteq T(C)$. Por lo que $\overline{y_1y_2} \subseteq T(C)$, esto sale del corolario 15 que dice que $T(\overline{ab})=\overline{a'b'}$ con $a'=T(a), b'=T(b)$ ya que $T$ es una transformación rígida, entonces $T(\overline{x_1x_2})=\overline{y_1y_2}$. Y resulta que $T(C)$ es convexo.

\subsection{Ejercicio 4}
\subsubsection{(a) Probar que si $T(\overrightarrow{pq})=\overrightarrow{rs}$ entonces $T(p)=r$ y $T(\check{\overrightarrow{pq}})=\check{\overrightarrow{rs}}$}
Sea $T(\overrightarrow{pq})=\overrightarrow{rs}$, por el corolario 


\end{document}