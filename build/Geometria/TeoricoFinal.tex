\documentclass[a4paper]{article}

%paquetes necesarios
\usepackage[a4paper, top=1.5cm, left=2cm, right=2cm,bottom=1.5cm]{geometry}
\usepackage{amsfonts}
\usepackage{amsmath, scalerel}
\usepackage{mathtools}
\usepackage{amssymb}
\usepackage{multicol}
\usepackage{graphicx}
\usepackage{polynom}
\usepackage{enumitem}
\usepackage{xcolor}

%texto del  documento
\title{Facultad de Matemática, Astronomía, Física y Computación}
\author{Teórico Final de Geomtría I}
\date{Febrero 2026}

\begin{document}
\maketitle
\section{Listado teórico:}
\subsection{\textcolor{blue}{Teorema 4}}
\subsubsection{Enunciado}
Sean $A$ una recta, $a \in A$ y $p \notin A$. Entonces $\overrightarrow{ap} \subset A_p$.
\subsubsection{Demostración}
Consideramos en $\overleftrightarrow{ap}$ el orden $<$ tal que $a<p$, quiero ir por el absurdo, supongo $\overrightarrow{ap}\nsubseteq A$, es decir, \textbf{existe $q \in \overrightarrow{ap}$ tal que $q \notin A_p$}.\\
Pero como $A \subset A_p$ entonces $q \in \check{A_p}$, más precisamente $q \in \check{A_p}- A$.\\
Por \textit{II.3} sabemos que $\overline{pq} \cap A \neq \varnothing$, pero $a,p,q$ están alineados, entonces $\overline{pq} \subseteq\overleftrightarrow{ap}$.\\
\begin{itemize}
    \item $\varnothing \neq \overline{pq} \cap A \subseteq \overleftrightarrow{ap} \cap A =\{a\} \text{ entonces } \overline{pq} \cap A=\{a\}$
    \item $\implies q<a<p \text{ por definición de segmento,}$
    \item $\text{pero } q \in \overrightarrow{ap} \implies a<q \text{ lo cual resulta }\textbf{absurdo}$
\end{itemize}
$\therefore \overrightarrow{ap} \subseteq A_p$
\subsection{\textcolor{blue}{Corolario 5}}
\subsubsection{Enunciado}
Los semiplanos cerrados son convexos.
\subsubsection{Demostración}
Dado un semiplano $A_p$ y sean $r, s \in A_p$ entonces hay 3 casos posibles:
\begin{enumerate}
    \item Si $r,s \in A_p - A$. Entonces por \textit{II.3} tenemos $\overline{rs} \subseteq A_p-A \subseteq A_p$
    \item Si $r,s \in A$. Entonces por ser $A$ una recta, $\overline{rs} \subseteq A \subseteq A_p$
    \item Si $r \in A_p -A$ y $s \in A$. Entonces $\overline{sr} \subseteq \overrightarrow{sr}\subseteq A_p$ por teorema 4
\end{enumerate}
En todos los casos llegamos a $\overline{rs} \subseteq A_p$\\
$\therefore A_p$ es convexo.
\subsection{\textcolor{blue}{Teorema 7}}
\subsubsection{Enunciado}
Sea $\angle AB$ un ángulo con $a \in A$ y $b \in B$. Si R es una semirrecta interior al ángulo, entonces $R$ interseca a $\overline{ab}$ en un único punto.
\subsubsection{Demostración}
Sea $o$ el vértice de $\angle AB$. Si $a=o$ ó $b=o$, el resultado es claro.\\
Consideremos ahora el caso $a,b \neq o$. Quiero ver que $a$ y $b$ están en semiplanos opuestos respecto a $\overleftrightarrow{R}$. Entonces sea $a'\in \check{A}$ con $a\neq o$, sabemos que $b \in A_B$ entonces por teorema 4 $\overrightarrow{a'b} \subseteq A_B$ y como $R \subseteq A_B$ y $R \subseteq B_A$, vale que:
\begin{enumerate}
    \item \textbf{$\overrightarrow{a'b} \cap \check{R}= \varnothing$}, análogamente $a'\in B_{\check{A}}$ entonces por teorema 4, $\overrightarrow{a'b} \subseteq B_{\check{A}}$
    \item \textbf{$\overrightarrow{ba'} \cap R= \varnothing$}
\end{enumerate}
Entonces 
\begin{gather*}
    \overline{a'b} \cap \overleftrightarrow{R} = \overline{a'b} \cap (R \cup \check{R})\\
    = (\overline{a'b} \cap R)\cup(\overline{a'b} \cap \check{R})\\
    = \varnothing \cup \varnothing = \varnothing
\end{gather*}
Deducimos del \textit{II.3} que $a'$ y $b$ están en el mismo semiplano respecto a $\overleftrightarrow{R}$. Como $a \in A$ y $a'\in \check{A}$ resulta que $a$ y $a'$ están en semiplanos opuestos respecto de $\overleftrightarrow{R}$\\
$\therefore a$ y $b$ están en semiplanos opuestos respecto a $\overleftrightarrow{R}$\\\\
Por el \textit{II.3} se deduce que $\overline{ab} \cap R \neq \varnothing$ y además $a, b \in sec(\angle AB)$ que es convexo, entonces $\overline{ab} \subseteq sec(\angle AB)$, luego la intersección $\overline{ab} \cap \overleftrightarrow{R}$ debe estar en $R$ pues los puntos de $\check{R}$ son exteriores a $\angle AB$, por lo tanto $\overline{ab} \cap R \neq \varnothing$.\\\\
Y si hubieran dos puntos de corte de $R$ y $\overline{ab}$, llamémoslos $p$ y $q$, entonces $\overline{ab} \subseteq \overleftrightarrow{pq}=\overleftrightarrow{R}$. Pero $\overline{ab} \subseteq sec(\angle AB)$.
$\therefore \overline{ab} \subseteq R \implies a,b$ son puntos interiores al $\angle AB$, \textbf{absurdo} por lo cual el punto es único.
\subsection{\textcolor{blue}{Teorema 10: Postulado de Pasch}}
\subsubsection{Enunciado}
Si $R$ es una recta que interseca al $\triangle abc$ y no pasa por sus vértices, entonces $R$ interseca a $\triangle abc$ en exactamente dos puntos.
\subsubsection{Demostración}
\begin{figure}[htb]
    \centering
    \includegraphics[scale=0.4]{imagenes/T10.png}
    \label{fig:Postulado de Pasch}
\end{figure}
Supongamos que R corta a $\overline{ac}$ en $p$ con $p \neq a$ y $p \neq c$. $R$ corta a $\overline{ac}$ únicamente en $p$ pues si la intersección tuviera más de un elemento resultaría en $\overleftrightarrow{R}=\overleftarrow{ac}$ lo cual es \textbf{absurdo}. Entonces $a$ y $c$ están en semiplanos opuestos respecto a $R$, es decir, $c \in \check{R}_a$. Ahora tenemos que $b \in R_a$ ó $b \in \check{R}_a$ (y $b \notin R$), entonces:
\begin{itemize}
    \item Si $b \in R_a$, entonces $a,b \in R_a - R$ y por $R_a - R$ convexo resulta $\overline{ab} \cap R=\varnothing$. Además, $c \in \check{R}_a$ y $b \in R_a$ y resulta por \textit{II.3} $\overline{bc} \cap R=\{q\}$ con $q \neq p$ y luego $R$ corta al $\triangle abc$ sólo en $p$ y $q$.
    \item Si $b \in \check{R}_a$, entonces análogamente se ve que $R$ corta al $\triangle abc$ en $\overline{ac}$ y $\overline{ab}$ en dos puntos y no corta al $\overline{bc}$.
\end{itemize}
$\therefore R$ corta al $\triangle abc$ sólo en $p$ y $q$ con $q \in \overline{ab}$.
\subsection{\textcolor{blue}{Teorema 11}}
\subsubsection{Enunciado}
Toda semirrecta con origen en un punto interior de una región poligonal convexa, interseca al polígomo en un único punto.
\subsubsection{Demostración}
\begin{figure}[htb]
    \centering
    \includegraphics[scale=0.6]{imagenes/T11.png}
    \label{fig:Teorema 11}
\end{figure}
Sea $o$ un punto interior al polígono $a_1, \ldots, a_n$. Entonces $o$ es interior a todos los ángulos $\angle a_i$ del polígono. En particular, $\overrightarrow{a_2o}$ es semirrecta interios al $\angle a_1a_2a_3$, y entonces $\overline{a_1a_2}$ y $\overline{a_2a_3}$ se encuentran en semiplanos distintos repecto de $\overleftrightarrow{oa_2}$, por el teorema 7. Luego, $\angle a_1oa_2$ y $\angle a_2oa_3$ son consecutivos. Análogamente, se tiene que $\angle a_{i-1}oa_i$ y $a_ioa_{i+1}$ son ángulos consecutivos para $i=1, \ldots, n$ (con $a_0=a_n$ y $a_{n+1}=a_1$). Por lo tanto, si $A_i = \overrightarrow{oa_i}$ para $i=1, \ldots, n$, entonces las semirrectas $A_i$ satisfacen las hipótesis del teorema 9.\\
$\therefore \Pi = \bigcup_{i=1}^n sec \left(\angle A_iA_{i+1}\right)$\\\\
Sea ahora $p \in \Pi$, $p\neq o$, si $\overrightarrow{op}=A_i$ para algún $i$, entonces $\overrightarrow{op}$ interseca al polígono únicamente en $a_i$. Si $\overrightarrow{op} \neq A_i, \forall i$, sabemos que $p \in \bigcup_{i=1}^n sec(\angle A_iA_{i+1})$ por el teorema 9, por lo tanto existe $i_o$ tal que $p \in sec(\angle A_{i_o}A_{i_o+1})$. Más aún, $p$ es interior al $\angle A_{i_o}A_{i_o+1}$ pues no está en sus lados. Luego $\overrightarrow{op}$ es una semirrecta interior al $\angle A_{i_o}A_{i_o+1}$. Por el teorema 7, el segmento $\overline{a_ia_{i+1}}$ interseca a $\overrightarrow{op}$, por lo que $\overrightarrow{op}$ corta al polígono.\\\\
\textbf{Unicidad:} Si $\overrightarrow{op}$ cortara al polígono en $q$ y $q'(q\neq q')$, consideramos las posiciones de $o, q$ y $q'$ en la semirrecta $\overrightarrow{op}$. Puede ser $o<q<q'$ o bien $o<q'<q$.
\begin{itemize}
    \item En el primer caso, supongamos $q \in \overline{a_ia_{i+1}}$. Entonces $o$ y $q'$ estarían en semiplanos distintos respecto de $\overrightarrow{a_ia_{i+1}}$, lo cual es absurdo pues $o$ y $q'$ están en la región poligonal.
    \item Es análogo
\end{itemize}
Luego, $\overrightarrow{op}$ corta al polígono en un único punto.
\subsection{\textcolor{blue}{Teorema 16}}
\subsubsection{Enunciado}
Sean $R$ una recta, $p \notin R$ y $T$ una transformación rígida. Entonces $T(R_p)=T(R)_{T(p)}$
\subsubsection{Demostración}
Definimos $R'=T(R)$ y $p'=T(p)$. Claramente $R'\subset R'_{p'}$ y $p' \in R'{p'}$. Sea ahora $q \in R_p-R$, con $q \neq p$. Luego $\overline{pq} \cap R= \varnothing$ y aplicando $T$ tenemos: $\overline{p'q'}\cap R' = \varnothing$ donde $q'= T(q)$ por corolario 15. De esto se deduce que $q'\in R'_{p'}$, por lo que vale $T\left(R_p\right) \subseteq R'_{p'}$.\\
Por axioma \textit{III.2}, sabemos que $T^{-1}$ es una transformación rígida y podemos considerar lo probado recién pero para esta transformación y obtenemos: $T^{-1}\left(R'_{p'}\right) \subseteq R_p$. Aplicando $T$: $R'_{p'} \subseteq T\left(R_p\right)$. Resulta entonces $T\left(R_p\right) = R'_{p'}$.
\subsection{\textcolor{blue}{Teorema 20}}
\subsubsection{Enunciado}
Sea $\left(A,\alpha\right)$ un par semirrecta-semiplano, $A$ de origen $o$ y $T$ la transformación rígida tal que $T(A,\alpha)=\left(\check{A}, \check{\alpha}\right)$. Entonces: 
\begin{enumerate}
    \item $T$ es involutiva.
    \item Si $\left(B,\beta\right)$ es otro par de semirrecta-semiplano tal que $B$ tiene origen en $o$, entonces $T\left(B,\beta\right)=\left(\check{B}, \check{\beta}\right)$.
\end{enumerate}
\subsubsection{Demostración}
\begin{enumerate}
    \item \textbf{T involutiva:} $T^2(A,\alpha)=T(T(A,\alpha))=T(\check{A},\check{\alpha})=(\check{T(A)},\check{T(\alpha)})=(\check{\check{A}},\check{\check{\alpha}})=(A,\alpha)$ y por unicidad del axioma \textit{III.4} resulta $T^2=Id$
    \item Tiene dos casos posibles:
    \begin{enumerate}
        \item Si $B=A$ ó $B=\check{A}$ entonces:
        \begin{equation*}
        \left\{
            \begin{array}{llll}
                T(A,\alpha)=(\check{A},\check{\alpha})\\
                T(\check{A},\check{\alpha})=(A,\alpha)\\
                T(A,\check{\alpha})=(\check{A},\alpha)\\
                T(\check{A},\alpha)=(A,\check{\alpha})
            \end{array}
        \right.
        \text{por lo que la afirmación vale}\\
        \end{equation*}
        \item Si $B\neq A$ ó $B\neq \check{A}$ entonces $B \subseteq \alpha$ ó $B \subseteq \check{\alpha}$, consideramos ambos casos análogos entonces trabajamos con $B \subseteq \alpha$\\\\
        Sea $p \in B$ y $p \neq o$ entonces $B=\overrightarrow{op}$ y sea $p':=T(p) \in \check{\alpha}$, por el axioma \textit{II.3} $\overline{pp'}$ corta a la recta $\overleftrightarrow{A}$ en un único punto $a$, lo cual puede estar en $A$ ó $\check{A}$, como son casos análogos, consideramos el primero:\\\\
        Si $a \in A$ entonces $\overline{pp'}\cap A=\{a\}$. Aplicamos $T$ a ambos lados y desarrollamos:
        \begin{equation*}
            T(a)=T(\overline{pp'})\cap T(A)=\overline{p'p}\cap \check{A}
        \end{equation*}
        Como sabemos que $\overline{pp}$'corta a $A$ en un único punto entonces $T(a)=a$ lo cual implica que $a=o$ pues es el único punto fijo y $o \in \overline{pp'}$ lo que implica que $p'\in \check{B}$, entonces $T(B)=T(\overrightarrow{op})=\overrightarrow{op'} = \check{B}$.\\\\
        Sea ahora $q \in \beta$ tal que $q \notin \overleftrightarrow{B}$ por lo probado anteriormente se tiene que $T(\overrightarrow{oq})=\check{\overrightarrow{oq}} \therefore T(q) \in \check{\beta}$ y así $T(\beta)=T(B_q)=\check{\beta}$
    \end{enumerate}
\end{enumerate}
\subsection{\textcolor{blue}{Teorema 25}}
\subsubsection{Enunciado}
Sea $(A,\alpha)$ un par semirrecta-semiplano, $A$ de origen $o$ y sea $T$ la única transformación rígida tal que $T(A,\alpha)=\left(A,\check{\alpha}\right)$. Entonces:
\begin{enumerate}
    \item $T$ es involutiva.
    \item $T(p)=p$ para todo $p$ en la recta $\overleftrightarrow{A}$
    \item Si $B$ es una semirrecta de la recta $\overleftrightarrow{A}$ entonces $T(B,\alpha)=\left(B,\check{\alpha}\right)$.
\end{enumerate}
\subsubsection{Demostración}
\begin{enumerate}
    \item $T^2(A,\alpha)=T(T(A,\alpha))=T(A,\check{\alpha})=(T(A),\check{T(\alpha)})=(A,\check{\check{\alpha}})=(A,\alpha)$. Por unicidad de \textit{III.4} tenemos $T^2=Id$.
    \item Como $T(A)=A$ y $A$ tiene origen en $o$ entonces $T(o)=o$ y sea $p':=T(p)\in A$ por $T(A)=A$ y $T(\overline{op})=\overline{op'}$ entonces por congruencia $\overline{op} \equiv \overline{op'}$ por lo tanto un segmento contiene a otro pero por \textit{III.3} resulta $\overline{op}=\overline{op'}$ por lo que $p=p'$, es decir, $p=T(p)$
    \item Por ser $p$ un punto arbitrario de $A$ entonces la semirrecta $B$ cumple por \textbf{(b)} que $T(B)=B$ además por la propia definición de $T$ resulta $T(\alpha)=\check{\alpha} \therefore T(B,\alpha)=(B, \check{\alpha})$. 
\end{enumerate}
\subsection{\textcolor{blue}{Teorema 28}}
\subsubsection{Enunciado}
Por cada punto del plano pasa una y solo una perperndicular a una recta dada. Es decir, dados $p \in \Pi$ y una recta $A$, existe una única recta $B$ tal que $B \perp A$ y $p \in B$.
\subsubsection{Demostración}
Consideramos dos casos:
\begin{enumerate}
    \item $p \notin A$
    \begin{itemize}
        \item \textbf{Existencia:} Sea $p':=S_A(p)$ que es distinto de $p$ y sea $B=\overleftrightarrow{pp'}$ entonces $p \in B$ y $B \perp A$ por el Teorema 26.
        \item \textbf{Unicidad:} Supongamos que $C$ es una recta tal que $p \in C$ y $C\perp A$. Por el teorema 27 $C$ es invariante por $S_A$ y como $p \in C$ luego $p':=S_A(p) \in C$ entonces $C=\overleftrightarrow{pp'}$ pero esa es $B \therefore B=C$ 
    \end{itemize}
    \item $p \in A$
    \begin{itemize}
        \item \textbf{Existencia:} Sea $q \notin A$ y $q'=S_A(q)$ entonces por el teorema 26 $\overleftrightarrow{qq'}\perp A$. Si $\overleftrightarrow{qq'} \cap A=\{p\}$ tomamos $B=\overleftrightarrow{qq'}$, en cambio si $\overleftrightarrow{qq'}\cap A=\{o\}$, con $o\neq p$, sea $m$ el punto medio de $\overline{op}$ y consideramos $B=S_m(\overleftrightarrow{qq'})$. Sea $q''=S_m(q) \implies S_m(\angle qop)=\angle q''po \therefore \angle qop \equiv \angle q''po$. Como $\angle qop$ es recto, resulta que $\angle q''po$ es recto y esto dice que $B \perp A$ y $p \in B$
        \item \textbf{Unicidad:} Supongamos que $B$ y $C$ son rectos tales que $B \perp A$ y $C\perp A$ y $p \in B\cap C$.\\\\
        Sea $b \in A$ con $b \neq p$ entonces $\overrightarrow{pb}$ es una semirrecta de A y $S_B(\overrightarrow{pb})=\check{\overrightarrow{pb}}=S_C(\overrightarrow{pb})$ pues $A$ es invariante por $S_B$ y $S_C$ ya que es perpendicular a ambos.\\\\
        Sea $\alpha$ un semiplano determinado por $A$ y consideremos $q \in \alpha \cap B$, $r \in \alpha \cap C$, con $q, r \notin A$. Luego, $S_B(q)=q$ y $S_C(r)=r$ entonces $S_B(\alpha)=S_C(\alpha)=\alpha$ y así $S_B(\overrightarrow{pb}, \alpha)=(\check{\overrightarrow{pb}}, \alpha)=S_C(\overrightarrow{pb}, \alpha) \implies S_B=S_C$ Por axioma \textit{III.4}. Esto dice que $B=C$, pues $B$ y $C$ son los conjuntos de puntos fijos de $S_B$ y $S_C$.  
    \end{itemize}
\end{enumerate}
\subsection{\textcolor{blue}{Teorema 32}}
\subsubsection{Enunciado}
Todo ángulo tiene una y solo una bisectriz
\subsubsection{Demostración}
Sea $o$ el vértice del ángulo $\angle AB$ 
\begin{itemize}
    \item \textbf{Existencia:} Sea $T$ la transformación rígida tal que $T(A,A_B)=(B,B_A)$, si $B':=T(B)$ entonces $T(\angle AB)=\angle BB'$ por lo tanto $\angle AB=\angle BB'$. Como $B'\subseteq B_A$ y $A\subseteq B_A$ por el teorema 18 resulta $B'=A$. Sean $a \in A$, $a\neq o$, $a':=T(a)\in B$ y $a'':=T(a')\in A$.\\\\
    $T(\overline{oa})=\overline{oa'}$ y $T(\overline{oa'})=T(\overline{oa''})$, luego $\overline{oa}\equiv \overline{oa'} \equiv \overline{oa''}$ por teorema 18 $\overline{oa}=\overline{oa''}$. Por lo tanto, $a=T(a')=a''$. Entonces $T(\overline{aa'})=\overline{a'a}$, esto nos dice $T(m)=m$ donde $m$ es el punto medio de $\overline{a'a}$. Dado que $m$ es punto interior al $\angle AB$ resulta $R=\overrightarrow{om}$ semirrecta interior al $\angle AB$ y $T(\angle AR)=\angle BR$ por lo cual son congruentes. Además como $T(R, R_A)=(R,R_B)=(R,\check{R_A})$ resulta $T=S_R$
    \item \textbf{Unicidad:} Sea $R'$ una semirrecta interior al $\angle AB$ tal que $\angle AR'\equiv \angle BR'$. Por teorema 31 se tiene $S_{R'}(A)=B$ y $S_{R'}(A_B)=\left(S_{R'}(A)\right)_{S_R(B)}=B_A$. Entonces $S_{R'}=T=S_R$ lo que implica $\overleftrightarrow{R}=\overleftrightarrow{R'}$.\\\\
    Por ser los conjuntos de puntos fijos de $S_R$ y $S_{R'}$, como $o$ es el origen de $R$ y $R'$, tenemos $R'=R$ ó $R'= \check{R}$, pero como ambos son semirrectas interiores debe ser $R'=R$
\end{itemize}
\subsection{\textcolor{blue}{Teorema 34}}
\subsubsection{Enunciado}
Si dos ángulos de un triángulo son congruentes los lados opuestos a tales ángulos tambíen lo son. Recíprocamente, si dos lados de un triángulo son congruentes los ángulos opuestos a tales lados son congruentes.
\subsubsection{Demostración}
\begin{enumerate}
    \item Consideremos primero el $\triangle abc$ tal que $\angle bac \equiv \angle abc$. Sea $M=M_{\overline{ab}}$ la mediatriz del $\overline{ab}$ entonces $S_M(a)=b$, $S_M(b)=a$ y así $S_M(\overrightarrow{ab})=\overrightarrow{ba}$. Sea
    \begin{gather*}
        c'=S_M(c) \therefore S_M(\overrightarrow{ac})=\overrightarrow{bc'}\\
        S_M(\angle bac)=\angle abc'\therefore \angle bac \equiv abc'
    \end{gather*}
    Luego $\angle abc \equiv abc'$. Además $\overrightarrow{bc}$ y $\overrightarrow{bc'}$ se encuentran en el mismo semiplano respecto de $\overleftrightarrow{ab}$ pues $M \perp \overleftrightarrow{ab}$ y $S_M(\overleftrightarrow{ab}_c)=\overleftrightarrow{ab}_c$. Por teorema 18, resulta $\overrightarrow{bc}=\overrightarrow{bc'}$, o sea, $S_M(\overrightarrow{ac})=\overrightarrow{bc}$. Como $S_M$ es involutiva, $S_M(\overrightarrow{bc})=\overrightarrow{ac}$, ahora:
    \begin{align*}
        \overrightarrow{ac} \cap \overrightarrow{bc}=&c\\
        S_M(\overrightarrow{ac}) \cap S_M(\overrightarrow{bc})=&S_M(c)\\
        \overrightarrow{bc} \cap \overrightarrow{ac}=&S_M(c) \quad \therefore S_M(c)=c
    \end{align*}
    $\therefore S_M(\overline{ac})=\overline{bc}$ con lo cual $\overline{ac}\equiv \overline{bc}$
    \item Consideramos $\triangle abc$ tal que $\overline{ac} \equiv \overline{bc}$. Sea $B$ la bisectriz del $\angle acb$ entonces $S_B(\overrightarrow{ca})=\overrightarrow{cb}$. Sea $a':=S_B(a) \in \overrightarrow{cb}$ y $S_B(\overline{ca})=\overline{ca'}$ entonces $\overline{ca}\equiv \overline{ca'}$. Por hipótesis $\overline{bc} \equiv \overline{ca} \implies \overline{bc} \equiv \overline{ca'}$ y por teorema 18 vale que $b=a'=S_B(a)$ y $S_B(b)=a$. Finalmente $S_B(\angle cab)=\angle cba \implies \angle cab \equiv \angle cba$
\end{enumerate}
\subsection{\textcolor{blue}{Teorema 36}}
\subsubsection{Enunciado}
Una recta interseca a una circunferencia en a lo sumo dos puntos.
\subsubsection{Demostración}
Supongamos que $p$, $q$ y $r$ son tres puntos alineados pertenecientes a una circunferencia de centro $o$, con $q \in \overline{pr}$.Como $\overline{op}\equiv \overline{oq}$, pues $p$ y $q$ están en la circunferencia entonces $o \in M_{\overline{pq}}$ por corolario 35. Como $\overline{oq} \equiv \overline{or}$ entonces $o \in M_{\overline{qr}}$.\\\\
Sean $m_1, m_2$ puntos medios de $\overline{pq}$ y $\overline{qr}$ respectivamente, entonces los puntos medios de $\overline{pq}$ y $\overline{qr}$ están en las semirrectas opeustas de $\overleftrightarrow{pr}$ con respecto a $q$. Luego $m_1\neq m_2$, como $m_1 \in M_{\overline{pq}}$ y $m_2 \in M_{\overline{qr}}$, resulta $M_{\overline{pq}}\neq M_{\overline{qr}}$ y ambas son perpendiculares a $\overleftrightarrow{pr}$ por $o$ lo que contradice la unicidad de las rectas perpendiculares que pasan por un punto.
\subsection{\textcolor{blue}{39}}
\subsubsection{Enunciado}
En todo triángulo un ángulo exterior es mayor que cada ángulo interior no adyacente.
\subsubsection{Demostración}
\begin{figure}[htb]
    \centering
    \includegraphics[scale=0.6]{imagenes/t39.png}
    \label{fig:Teorema 39}
\end{figure}
Sea $d \in \check{\overrightarrow{ba}}$. Probemos que $\angle cbd>\angle acb$. Sean $m$ el punto medio de $\overline{bc}$ y $a'=S_m(a)$. $S_m(\angle acb)=\angle a'bc \therefore \angle acb \equiv \angle a'bc$. Veamos que $a'$ es interior al $\angle cbd$:
\begin{itemize}
    \item $m \in \overline{aa'} \implies m$ y $a'$ están en el mismo semiplano respecto de $\overleftrightarrow{ab}$
    \item $m \in \overline{bc'} \implies m$ y $c$ están en el mismo semiplano respecto de $\overleftrightarrow{ab}$
\end{itemize}
Entonces $a'$ y $c$ están en el mismo semiplano respecto de $\overleftrightarrow{ab} \therefore a'\in \overleftrightarrow{ab_c}$. Ahora, $a$ y $a'$ están en distintos semiplanos respecto de $\overleftrightarrow{bc} \therefore a'\in \check{\left(\overleftrightarrow{bc_a}\right)}$. Luego $a'\in \overleftrightarrow{ab_c}\cap \check{\left(\overleftarrow{bc_a}\right)}=sec(\angle cbd)$. Claramente, $a'\notin \angle cbd \therefore a'$ es interior al $\angle cbd$.
\begin{equation*}
    S_m(\angle acb)=\angle a'bc \therefore \angle acb \equiv \angle a'bc
\end{equation*} 
Como $\overrightarrow{ba'}$ es interior al $\angle cbd$ tenemos que $\angle cbd>\angle acb$. Falta ver $\angle cbd>\angle cab$\\\\
Sea $d' \in \check{\overrightarrow{bc}}$. Por ser opuestos por el vértice, tenemos $\angle cbd \equiv \angle abd'$. Por lo recién probado, vale: $\angle abd'>\angle cab \therefore \angle cbd > \angle cab$
\subsection{\textcolor{blue}{Teorema 41}}
\subsubsection{Enunciado}
En todo triángulo, a mayor lado se opone mayor ángulo y recíprocamente a mayor ángulo se opone mayor lado.
\subsubsection{Demostración}
\begin{enumerate}
    \item Supongamos $\overline{ac}>\overline{ab}$, queremos probar que $\angle abc>\angle acb$
\end{enumerate}
\end{document}