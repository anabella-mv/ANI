\documentclass[a4paper]{article}
\usepackage{amsfonts}
\usepackage{amsmath}
\usepackage{mathtools}
\title{Apuntes}
\author{Matemática 4to}

\begin{document}

\maketitle
\section{Practico 0}
\subsection{Ejercicio 1}
\subsubsection{$A \subseteq B \leftrightarrow A \subseteq A\cap B$}
\textbf{($\Rightarrow$)} Si $A \subseteq B$, entonces $A\subseteq(A\cap B)$, $\forall x \in A$ como $A$ está contenida o es igual a $B$, Por lo tanto $x \in B$, es decir, $x \in A\cap B$.\\
\textbf{($\Leftarrow$)} Si $A \subseteq (A\cap B)$, entonces $A \subseteq B$, Tomo cualquier $x \in A$. Por hipotesis $x \in (A\cap B)$, por lo que $x \in B$ y como el elemento es arbitrario de $A$ resulta en todo $A$ contenida en $B$.

\subsubsection{$A \subseteq B \leftrightarrow A\cup B \subseteq B$}
\textbf{($\Rightarrow$)} Si $A \subseteq B$, entonces como todo elemento de $A$ está en $B$, y la union contiene a los elementos de ambos, entonces todos los elementos de A están tambien en su union con otros conjuntos. Pero como todos los elementos de $A$ están en $B$ resulta en que los elementos de la unión son los elementos de $B$, por lo tanto $A\cup B = B$\\
\textbf{($\Leftarrow$)} Si $A\cup B \subseteq B$, entonces todos los elementos de $A\cup B$ están en $A$ y/o $B$. Entonces tenemos 4 casos:\\
Si $x \in A$ y $x \notin B$ entonces como la unión está contenida en $B$, entonces todo elemento de $A$ está contenido en $B$.\\
Si $x \in A$ y $x\notin B$, entonces no es posible ya que la unión debería estar contenida en $B$. Por lo tanto todos los elementos de $A$ están en B.

\subsubsection{$A \subseteq B \Rightarrow B=A\cup (B\setminus A)$ y $A\cap(B\setminus A)=\emptyset$}
Como estamos hablando de igualdades necesitamos demostrar la doble inclusión entre ambas proposiciones.\\
\textbf{($\subseteq$)} Sea $x \in B$, tenemos dos posibilidades:\\
Si $x \in A$, entonces $x \in A \subseteq A\cup (B\setminus A)$\\
Si $x \notin A$, entonces $x \in B\setminus A \subseteq A\cup (B\setminus A)$\\
En ambos casos $x \in A\cup (B\setminus A)$, por lo cual $B \subseteq A\cup (B\setminus A)$.\\
\textbf{($\supseteq$)} Sea $x \in A\cup (B\setminus A)$, entonces tenemos dos posibilidades, pues $A$ y $B\setminus A$ son disjuntas:\\
Si $x \in A$, entonces $x \in B$ por hipotesis.\\
Si $x \in B\setminus A$, entonces $x \in B$ por definición de diferencia.\\
En ambos casos $x \in B$. Como hay doble contencion, se concluye la igualdad.

\subsubsection{$A\cap (B\cup C)=(A\cap B)\cup(A\cap C)$}
Nuevamente demostramos la doble inclusión.\\
\textbf{($\subseteq$)} Sea $x \in A\cap (B\cup C)$, entonces $x \in A$ y $x \in (B\cup C)$. A su vez, por definicion de union $x \in B$ o $x \in C$. Viendo los dos casos:\\
$x \in A$ y $x \in B$ lo cual es la definicion de intersección $x \in A\cap B$\\
$x \in A$ y $x \in C$ lo cual es la definicion de intersección ($x \in A\cap C$). Resulta en $x$ en alguna de esas intersecciones o ambas, por lo tanto $x \in (A\cap B)\cup(A\cap C)$.\\
\textbf{($\supseteq$)} Sea $x \in (A\cap B)\cup(A\cap C)$, entonces $x \in (A\cap B)$ o $x \in (A\cap C)$. Ambas intersecciones incluyen a $A$, por lo que $x \in A$. Ahora tenemos dos casos:\\
Si $x \in (A\cap B)$, entonces $x \in B$ por lo que $x \in (B\cup C)$ y resulta en $x \in A\cap (B\cup C)$\\
Si $x \in (A\cap C)$, entonces $x \in C$ por lo que $x \in (B\cup C)$ y resulta en $x \in A\cap (B\cup C)$\\
Por doble contención se concluye la igualdad.

\subsubsection{$A\cup (B\cap C)=(A\cup B)\cap(A\cup C)$}
\textbf{($\subseteq$)} Sea $x \in A\cup (B\cap C)$, entonces $x \in A$ o $x \in (B\cap C)$:\\
Si $x \in A$, entonces $x \in (A\cup B)$ y $x \in (A\cup C)$ por lo que $x \in (A\cup B)\cap(A\cup C)$.\\
Si $x \in (B\cap C)$, entonces $x \in B$ y $x \in C$, por lo que $x \in (A\cup B)$ y $x \in (A\cup C)$, por lo que $x \in (A\cup B)\cap(A\cup C)$.\\
\textbf{($\supseteq$)} Sea $x \in (A\cup B)\cap(A\cup C)$, entonces $x \in (A\cup B)$ y $x \in (A\cup C)$, es decir:\\
$x \in A$ lo cual resulta en $x \in A\cup (B\cap C)$\\
$x \notin A$, como $x$ tiene que estar en ambas uniones, entonces al no estar en $A$, $x \in B$ y $x \in C$, por lo que $x \in (B\cap C)$ y resulta en $x \in A\cup (B\cap C)$.\\
Por doble contención se concluye la igualdad.

\subsubsection{$A \subseteq (B\cap C) \rightarrow A \subseteq B$ y $A \subseteq C$}
Sea $x \in A$, por hipotesis $A \subseteq (B\cap C)$, por lo que $x \in (B\cap C)$, es decir, $x \in B$ y $x \in C$. Como cualquier elemento arbitrario de $A$ está en $B$ y $C$, entonces todo $A$ está en $B$ y en $C$.

\subsubsection{$A\cup B \subseteq C \rightarrow A \subseteq C$ y $B \subseteq C$}
Sea $x \in A\cup B$, esto quiere decir que hay 3 casos:\\
Si $x \in A$, y $x \notin B$, entonces todos los elementos de $A$ están en $C$.\\
Si $x \notin A$, y $x \in B$, entonces todos los elementos de $B$ están en $C$.\\
Si $x \in A$ y $x \in B$, entonces todos los elementos de $A$ y $B$ están en $C$.\\
En los 3 casos se ven incluidos los conjuntos $A$ y $B$ en $C$.

\subsubsection{$A \subseteq B \leftrightarrow B^{c} \subseteq A^{c}$}
\textbf{($\Rightarrow$)} Sea $x \in A$, entonces $x \notin A^{c}$. Por hipotesis $A \subseteq B$, por lo que $x \in B$

\section{Práctico 1}
\subsection{Ejercicio 1}
Los axiomas de incidencia son:
\begin{enumerate}
    \item El plano es un conjunto infinito y sus elementos de llaman puntos.
    \item Existe $\mathcal{R} \subseteq \mathcal{P} (\pi)$ tal que cada $R \in \mathcal{R}$ es un subconjunto propio de $\pi$ que posee al menos dos elementos.
    \item Dados $a,b \in \pi$ con $a \neq b$, existe un único $R \in \mathcal{R}$ tal que $a,b \in R$.  
\end{enumerate}
\textbf{(1)} Sea $\pi = \mathbb{Z} \times \mathbb{Z}$, entendemos por infinito a todo conjunto que no es finito, es decir, $\forall n \in \mathbb{N}$ ninguna función $f:X\rightarrow \{1,\dots ,n\}$ es biyectiva. Más aún conocemos que los naturales son infinitos, basta crear una inyección con ellos para hablar de un conjunto infinito. Consideramos $f:\mathbb{N}\rightarrow\mathbb{Z} \times \mathbb{Z}$ con $f(n)=(n,0)$, es claro que $f$ es inyectiva, por lo tanto $\pi$ es infinito.\\\\
\textbf{(2)} Las rectas están implícitas en este conjunto como los subconjuntos de $\pi$ de dos elementos. Por definición, son subconjuntos propios de $\pi$ y poseen al menos dos elementos, mas específicamente, dos.\\\\
\textbf{(3)} Dados $a,b \in \pi$ con $a\neq b$, con como está definidas las rectas, se sabe que los subconjuntos tienen sólo dos elementos, por lo tanto no es posible que otra recta aparte de ella misma contenga esos dos elementos.

\subsection{Ejercicio 2}
Dado $V=\mathbb{R}^3$ espacio vectorial tridimensional sobre los reales. Se considera el plano $\pi$ de subespacios unidimensionales de $V$. Estos serán los puntos. Si $W \subset V$ en un subespacio de $V$ de dimension 2, entonces el conjunto de puntos contenidos en $W$ será llamado una recta. Mostrar que este modelo cumple con los axiomas de incidencia.\\\\
\textbf{(1)} $n \rightarrow R_n=<(1,n,0)>=\{(x,y,z)=t(1,n,0) \forall t \in \mathbb{R}\}$\\
$(x,y,z) \in R_n$ si y solo si $(x,y,z)=(t,tn,o)$\\
\begin{equation}
    \left\{
    \begin{array}{l}
        x=t\\
        y=tn\\
        z=0
    \end{array}
    \leftrightarrow
    \begin{array}{l}
        z=0\\
        y=nx
    \end{array}
    \right.
\end{equation}
Suponiendo $n_1\neq n_2$, $R_{n_1}=<(1,n_1,0)>$ y $R_{n_2}=<(1,n_2,0)>$.\\
$k_1(1,n_1,0)+k_2(1,n_2,0)=(0,0,0)$ Los vectores $(1,n_1,0)$ y $(1,n_2,0)$ son li.
\begin{equation}
    \begin{array}{c}
        k_1+k_2=0 \Rightarrow k_2=-k_1\\
        n_1k_1+n_2(-k_1)=0\\
        k_1(n_1-n_2)=0        
    \end{array}
\end{equation}

\textbf{(2)} Las rectas están definidas como los puntos contenidos en $W\subset V$ (subespacios de dimension 2). Cada una de ellas debe ser un subconjunto propio de $\pi$ que posea al menos dos elementos de $V$ 

\subsection{Guía 1:Ejercicio 9}
\subsubsection{Enunciado}
La actividad 5 debe habernos dejado como enseñanza que hay modelos de geomtría en los que se satisfacen los axiomas de incidencia y en los que las rectas no son infinitas. De ello inferimos que la infinitud de las rectas no es una consecuencia lógica de estos axiomas. Así pues, asumamos los axiomas de orden:
\begin{enumerate}
    \item Dar la definición de que un conjunto sea infinito.
    \item Que cada uno elabore una demostración de los siguientes enunciados. Las rectas tienen una cantidad infinita de puntos. Las semirrectas tienen infinitos puntos. Los segmentos tambien poseen una cantidad infinita de puntos.
    \item Compare las estrategias seguidas por cada integrante.
\end{enumerate}
\subsubsection{Solución}
\textbf{(1)} Puntualmente un conjunto $A$ es infinito si \textbf{no} es finito, se entiende que no hay biyección posible entre el conjunto y un conjunto $\{1,\dots,n\}$ con $n\in \mathbb{N}$, puedo armar una inyección $f:\mathbb{N}\rightarrow A$ también para definir la infinitud de A ya que $\mathbb{N}$ es infinito, por no haber una biyección entre el conjunto finito $\{1,\dots, n\}$ y $\mathbb{N}$.\\\\
\textbf{(2)}
\begin{itemize}
    \item \textbf{Las rectas tienen una cantidad infinita de puntos.}\\
    \item Sea $r$ una recta,  por definicion de recta tenemos 
    \item ejercicios 3, 5 y 10 con rodrigo.
\end{itemize}
\end{document}