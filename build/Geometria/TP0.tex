\documentclass[a4paper]{article}
\usepackage{amsfonts}
\usepackage{amsmath}
\usepackage{mathtools}
\title{Apuntes}
\author{Matemática 4to}

\begin{document}

\maketitle
\section{Practico 0}
\textbf{1.(a) $A \subseteq B \longleftrightarrow A \subseteq A\bigcup B$}
Que $A$ esté contenida en $B$ quiere decir que para todo elemento $x \in A$ resulta que $x \in B$, esto resulta en la definición de la intersección, por lo tanto $\forall x \in A$ resultan en $B$ y por consiguiente en la intersección de ambos cjtos
\textbf{1.(b) $A \subseteq B \longleftrightarrow A\cup B \subseteq B$}

Como todo elemento de A está en B, $\forall x \in A$ resulta que $x \in B$, si están en la intersección, más aún se encuentra en la union, por lo tanto como la union resulta en la union de todos los elementos de A y B, entonces bastaría ver que los elementos de A están en B, lo cual es verdadero por proposicion.

\end{document}